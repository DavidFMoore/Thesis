\section{Parameterizing the Polarized Sky}
\label{sec:sim_params}

\subsubsection*{Source Positions}
Source positions are distributed uniformly over the sphere. A single source's altitude $\theta$ is
drawn from a distribution in which $\cos\theta$ is uniform on $[0,1]$. A source's azimuthal angle
$\phi$ is drawn independently from $\cos\theta$, uniform on $[0,2\pi)$. This choice of source
position distributions conserves number of sources per unit area across the sky, and is equivalent
to drawing both direction cosines, $(l,m) = (\sin\theta\cos\phi, \sin\theta\sin\phi)$, from a
uniform distribution on $[-1,1]$.

\subsubsection*{Flux Counts}
In order to achieve realistic source fluxes and source counts, we base the distributions from which
we draw various parameters on previous radio surveys. For the source fluxes, we aim to agree with
the VLA Large Sky Survey \cite[henceforth, called
VLSS]{VLSS}\nomenclature[Zv]{VLSS}{VLA Large Sky
Survey} and the Sixth Cambridge Survey
\cite[henceforth, called 6C]{Hales1988}\nomenclature[Zs]{6C}{Sixth Cambridge Survey} surveys, summarized alongside a full polarization
survey in Table \ref{tab:surveys}.
\begin{table}
  \begin{center}
    \begin{tabular}{c c c c c}
      {\bf Survey Name} & {\bf Frequency} & {\bf Survey Area} & {\bf Full Stokes?} & Citation\\ 
      \hline\hline
      VLSS &  74 MHz & $\delta > -30^\circ$ &  No & \citet{VLSS} \\
      6C   & 151 MHz & 0.82 sr              &  No & \citet{Hales1988} \\
      NVSS & 1.4 GHz & $\delta > -30^\circ$ & Yes & \citet{NVSS} \\
    \end{tabular}
  \end{center}
  \caption[Summary of three, low-frequency surveys]{\label{tab:surveys} Summary of three,
  low-frequency surveys.}
\end{table}
We can take 6C source counts at face value, since it was measured in the PAPER band at 151 MHz.
However, we must extrapolate the VLSS source counts from the observed 74 MHz into the paper band. We
perform this extrapolation, following \citet{Cohen2004}, by applying a spectral index of -0.79 to the
amplitude of the source counts.

Above some limiting flux $S_{min}$, the differential number counts in ($dN/dS$) found in the 6C survey 
may be characterized by a double power law, turning over at some knee flux $S_0$:
\begin{equation}
  \dfdx{N}{S} = \begin{cases}
    4000\ S_0^{-0.76}S^{-1.75}\ {\rm Jy}^{-1}{\rm sr}^{-1} & S_{min} \le S < S_0 \\
    4000\ S^{-2.81}\ {\rm Jy}^{-1}{\rm sr}^{-1} & S_0 \le S
  \end{cases}.
\end{equation}
Following the 6C results, we choose the turning point $S_0$ to be 0.88 Jy. The number of sources
simulated (20,531) is chosen by the size of the PAPER beam at 151 MHz (0.76 sr) and a flux range
over which to integrate. In general, this operation can be expressed by the integral
\begin{equation}
  N = \Omega \int_{S_{min}}^{S_{max}} \dfdx{N}{S}\D{S},
\end{equation}
where $S_{min}$ and $S_{max}$ are the limiting fluxes of the survey, and $\Omega$ is the survey's field of
view. We choose to include sources in between 100 mJy and 10 Jy. This choice provides a reasonable
dynamic range of sources. Below the lower limit, the 6C sources are unreliable due to
signal-to-noise issues and confusion\footnote{Below a certain flux, I expect to
find more than one source per resolution element --- this uncertainty in the number of sources per
pointing adds to the variance of my measurement.}, and above 10 Jy, we expect that sources may be 
easily identified and removed.

By setting our lower cutoff too high, are we omitting much of the power contained in our
measurement? If we were to blindly extrapolate the 6C source counts below the lower limit of the catalog, we
would add a negligible amount of power. Integrating $S^2\ dN/dS$ down to some minimum flux estimates
the contribution of the sources above that flux to the total variance of the flux. Performing this
operation to the 6C source counts, we find that we are including $\sim70\%$ of the total variance.
Extending the minimum flux would indeed add more power to the simulation, but it would not
drastically alter these results.

The VLSS source counts follow a single power law, given by 
\begin{equation}
  \dfdx{N}{S} = 4865\ S^{-2.3}\ {\rm Jy}^{-1}{\rm sr}^{-1},
\end{equation}
where the spectral index of -0.79 has been applied. For these, we choose minimum and maximum fluxes
of 0.8 Jy and 100 Jy, respectively, rejecting sources well below the lower limit of the catalogue,
and providing a reasonable dynamic range for the included sources. Integrating over the PAPER beam
provides 11,262 sources.

Qualitatively, the source counts for these two surveys differ in two ways. The 6C survey yields
more, dimmer sources, where the VLSS survey yields fewer, brighter sources. By examining the
difference in polarized power from these two source counts, we may answer the question ``Is $Q\to I$
leakage due mostly to a few, bright sources, or is it due to a forest of unresolved, dim sources?''

It is worth noting the robustness of these two source counts with respect to independent
measurements --- both agree with the results of a recent survey from the Murchison Widefield Array
\cite{Williams2012}, an instrument similar in many regards to PAPER.

\subsubsection*{Spectral Indices}
All sources are assigned a spectral index, which is drawn from a normal distribution with mean -0.8,
and standard deviation 0.1. This roughly agrees with the findings of \citet{Helmboldt2008}.

\subsubsection*{Polarized Fractions}
Instead of drawing polarized sources from a measured polarized flux distribution, we simply
down-weight the total intensity by some polarized fraction ($\Pi$), chosen to reflect the studies 
of \citet{Tucci2012}. We sample the $\Pi$ from a log-normal distribution whose mean is $2.01\%$ and
whose standard deviation is $\log(3.845\%)$. Because the log-normal distribution has no upper bound,
and it is unreasonable to find sources with a high polarized fraction\footnote{In fact, it is
impossible to measure $\Pi > 1$!}, we truncate the distribution at $30\%$. As we will investigate in
Section \ref{sec:updates}, this upper limit is considerably higher than what has been measured
at 150 MHz. Following the aforementioned study, we do not impose any correlation between source flux
and polarization fraction.

It has been noted that, among other effects, bandwidth depolarization causes the polarized fraction
to decrease at lower frequencies \cite{Law2011}. This, alongside the GMRT measurements
\cite{Pen2009}, indicates that these distributions, taken at 1.4 GHz, may overestimate the
distribution at 150 MHz. We neglect this instinct that we are overestimating the polarized flux,
taking the 1.4 GHz measurements at face value, since the mean polarization fraction can be thought
of as a scale factor to the overall power spectrum.

\subsubsection*{Polarization Angles}
The polarization angle of each source is chosen to be uniformly sampled on $[0,\pi)$, which assumes no correlation in the polarization angles of individual extragalactic sources. Section \ref{sec:Consistency} 
will investigate the validity of this claim.

\subsubsection*{Rotation Measures}

We draw our distribution of rotation measures on the map presented in \citet{Oppermann2012}. To
mimic the effects of depolarization due to a finite spacial resolution \cite{Law2011}, we apply a
low-pass filter to the rotation measure map. Projecting the map into a spherical harmonic basis, we
keep only those modes below the resolution of our simulated instrument. In the case of this
simulation, we choose to keep only $\ell \le 100 = 2\pi |\vec{b}/\lambda|$. This averages the
polarization vectors in much the same way as a synthesized beam, and its effect is to essentially
remove outliers in the rotation measure distribution, to which instruments like PAPER may not be
sensitive. We then randomly draw rotation measures from the empirical cumulative distribution
function of rotation measures, computed from the filtered, \citet{Oppermann2012} maps. Aside from low-pass
filtering, no spatial information from the data is used. Section \ref{sec:Consistency} briefly
discusses the negligible consequences of spatially correlating rotation measures.

\figuremacroW{SimParams.eps}{1}{fig:sim_params}{
  Distributions of simulated parameters. (Top Left) Euclidean normalized source counts for the sources produced 
  in simulations $A$ (blue) and $B$ (black). Over plotted in cyan and gray, respectively, are the
  analytical distributions from which they are drawn. (Top Right) Distribution of polarized
  fractions used, with the log-normal distribution over-plotted in gray. (Bottom) Empirical
  distribution of rotation measures, generated from a spatially low-pass-filtered map
  \cite{Oppermann2012}. 
}{Distributions of simulation parameters
$S$, $\Pi$, and $\Phi$.}
Histograms of the empirical distributions of rotation measure, polarized fraction, and source counts may be
found in Figure \ref{fig:sim_params}. Over-plotted on all are the distributions from which they are
drawn. Figure \ref{fig:sim_polflux} shows the empirical distribution of polarized flux, using the
NVSS and 6C surveys --- these distributions should be convolutions of the power law source counts
and the log-normal polarized fraction. These distributions qualitatively agree with the total power 
source counts: the 6C survey produces dimmer sources, while NVSS produces fewer.
\figuremacroW{SimPolFlux.eps}{0.6}{fig:sim_polflux}{
  Euclidean-normalized, differential source counts for polarized flux in simulations $A$ (blue)
  and $B$ (black).
}{Distribution of polarized flux in two
simulations}

We calculate visibilities for a single 30m, east-west baseline, corresponding to the most common spacing in
the maximum-redundancy PAPER array \cite{PAPERSensitivity, Parsons2014}. The choice of baseline
orientation is arbitrary, and since we are only modelling point sources, the choice of baseline
length will only set the horizon limit of the power spectrum. Since the delay affected by a rotation
measure is independent of a choice of baseline (Equation \ref{eq:bad_k}), choosing a relatively
short baseline length will isolate smooth-spectrum foregrounds at lower $\tau$ and highlight Faraday leakage.

The full measurement equation for the visibility with linear polarization $p$ ($\mathcal{V}_p$) used 
in this simulation is
\begin{align}
  \mathcal{V}_p = \sum_{j=1}^{N_{src}}
    A_p(l_j, m_j, \nu)
    S_j^{150}
    \left(\frac{150\ {\rm MHz}}{\nu}\right)^{\alpha_j}
    e^{ -2\pi i\nu(ul_j + vm_j)}
    \left(1 \pm \Pi_j e^{-2i(\Phi_j\lambda^2 + \chi_j)} \right),
    \label{eq:sim_vis}
\end{align}
where each source $j$ is assigned a flux ($S_j$), a polarized fraction ($\Pi_j$), a spectral index
($\alpha_j$), a position $(l_j,m_j)$, rotation measure ($\Phi_j$), polarization angle ($\chi_j$),
and is weighted by the model primary beam in that linear polarization ($A_p$). To include both $I$
and $Q$ emission, $xx$ visibilities receive the $+$, while $yy$ visibilities recieve the $-$. They
are then summed and differenced to yield $\mathcal{V}_I$ and $\mathcal{V}_Q$. A sample $Q$ visibility 
is shown in Figure \ref{fig:sample_Q_sim}.
\figuremacroW{SampleQSim}{0.6}{fig:sample_Q_sim}{
  The real part of a sample $Q$ visibility, given by Equation \ref{eq:sim_vis}, and generated using
  the parameters shown in Figure \ref{fig:sim_params}. 
}{Sample simulated $Q$ visibility.}

We choose not to include the parallactic rotation of $Q$ and $U$ (see Section
\ref{sec:Polarimetry}), implying that the $Q$ we label for this simulation is fixed to topocentric,
azimuth and altitude coordinates. This choice clarifies equations and allows for an ease of
understanding which would be obfuscated by writing equatorially defined $Q$ and $U$.

\begin{table}
  \begin{center}
    \begin{tabular}{c c c c}
      {\bf Label} & {\bf Source Counts} & {\bf $N_{src}$} & {\bf Rotation Measure Distribution}\\ 
      \hline\hline
      A & 6C   & 20,531 & Oppermann \\
      B & NVSS & 11,262 & Oppermann \\
      C & 6C   & 20,531 & 2$\times$Oppermann \\
    \end{tabular}
  \end{center}
  \caption[Simulation treatments]{\label{tab:sims} Three simulation treatments used in this section.}
\end{table}

Finally, we take advantage of our built-in tunable parameters and present three treatments of the
simulation, summarized in Table \ref{tab:sims}. Simulation $A$ serves as a baseline measurement,
with reliable $6C$ source counts and the conservative, low-pass filtered Oppermann rotation measure
distribution. Simulation $B$ uses NVSS source counts, asking is $Q\to I$ leakage is dominated by a
few, bright sources, rather than the forest of dim sources in 6C. Simulation $C$ uses the 6C source
counts, but doubles the Oppermann rotation measures, asking how large rotation measures affect $Q\to
I$ leakage.
