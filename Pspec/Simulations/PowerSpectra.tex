\section{Results}
\label{sec:sim_results}

\figuremacroW{SimulationResults}{0.9}{fig:sim_power_spectra}{
  Power spectra for the three treatments of the simulation discussed in Section
  \ref{sec:sim_params}. From top to bottom, the rows are treatments A, B, and C. The left column
  shows the $I$ power spectrum, highlighting $Q\to I$ leakage. The right column shows the $Q$ power
  spectrum. Three redshift bins are shown in all plots: $z=11.13$ (cyan), $z=9.77$ (black), and
  $z=7.05$ (blue). Error bars show 95\% confidence intervals of several iterations of the
  simulation. For a point of reference in the left column, a fiducial EoR model \cite{Lidz2008} is plotted in grey.   
}{Power spectra for three treatments of the simulation}

Figure \ref{fig:sim_power_spectra} the power spectra of several renderings of simulations 
$A$, $B$, and $C$. We interpret the power spectrum of $\mathcal{V}_I$ outside the horizon 
as the amount of polarized leakage corrupting the EoR signal ($Q\to I$ leakage), and the 
power spectrum of $\mathcal{V}_Q$ is our best representation of the polarized signal. These 
plots show the median power in each $k$ bin for 100 realizations of the simulation, with error bars
showing the 1-$\sigma$ extend of the bandpowers for these realizations. These power spectra confirm
the prediction made in Section \ref{sec:onesource} that $\lambda^2$ phase wrapping extends the
foreground cutoff \cite[][e.g.]{PAPERSensitivity, Pober2013} to higher delay bins, corrupting some of the
most sensitive regions of $k$ space for 21cm EoR analysis. They also demonstrate the prediction that
high-redshift bins will be most affected.

\figuremacroW{DeltakGoodK}{0.6}{fig:k3pk_goodk}{
  $A$ (cyan), $B$ (blue), $C$ (black)
}{$\Delta^2(k)$ at $k\approx 0.2\ h{\rm Mpc}^{-1}$ vs. redshift for three simuations}
The severity of the leakage can be inferred from the power in the most EoR-sensitive $k$-bins which lie
outside the horizon for small baselines ($0.2\ h{\rm Mpc}^{-1} \le k \le 0.3\ h{\rm Mpc}^{-1}$). Figure 
\ref{fig:k3pk_goodk} shows $\Delta^2(k)$ in these bins as a function of redshift. The leaked power 
ranges in the hundreds of $\rm{mK}^2$ to thousands, increasing from high frequency/low
redshift to low frequency/high redshift. These simulations are about an order of magnitude above the
level of the expected 21cm signal \cite{Lidz2008}. If we may take this simulation as an accurate
prediction of the low-frequency sky's polarized emission, these results imply that na\"{i}vely
adding $\mathcal{V}_{xx}$ and $\mathcal{V}_{yy}$, formed with an approximately 10\% asymmetric
primary beam, incorporates enough bias from polarized leakage to completely obscure the 21cm signal.
The levels of leakage in our simulations demand a strategy to model and remove point sources.

We note that simply forming $\mathcal{V}_I$ will also remove a negligible component of the EoR
signal via the same mechanism. In a sense, the $Q\to I$ leakage can be thought of as a rotation of
power between the two Stokes parameters. Hence, for precision measurements of the EoR signal, this
simple estimate may not be ideal. However, the effects of the $I\to Q$ leakage is small (compare the
levels of $\mathcal{V}_Q$ and high $k$-modes of $\mathcal{V}_I$) and should not provide a
significant hiderance to detection.

Were a power spectrum computed from only one linearly polarized visibiltity ($xx$, for instance),
all polarized power would corrupt the measurement. We have chosen to suppress the polarized leakage
by adding the linearly-polarized visibilities $xx$ and $yy$. The leakage is dependent on the
difference of the two beams (see discussion in Section \ref{sec:BeamLeakage}), and by having beams
that are at most 10\% different suppresses the signal by around two to three orders of magnitude.
Correcting for the beam-weighting in the image domain can further suppress the leakage, but errors
in the beam model will introduce leakage in much the same manner. Hence, the constraint of having to
suppress polarized leakage by four orders of magnitude causes the need for an accurate primary beam
model to around the 1\% level in the case of imaging, or symmetric at the 1\% level if visibilities
are used directly.

We conlude this discussion by noting the large variance in simulted power. The results shown are the
mean bandpowers  in $\Delta^2(k)$ for several realizations of the simulation. Taking so many
realizations into account essentially maps out the posterior distribution of the $\Delta^2(k)$
bandpowers. The $2\sigma$ width covers nearly two orders of magnitude, which indicates that the
level of $Q\to I$ leakage is highly sensitive to the exact parameters drawn in any realization.
The actual level of leakage measured will thus be highly dependent on a choice of field, and on
the actual distributions of polarized fluxes and rotation measures.
