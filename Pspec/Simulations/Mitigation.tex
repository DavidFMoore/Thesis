\section{Mitigating Leakage}
\label{sec:sim_mitigation}

Section \ref{sec:sim_results} predicts an excess polarized signal due only to point sources of
around $10^4\ {\rm mK}^2$ at $k\sim 0.15\ h{\rm Mpc}^{-1}$ for most treatments of the simulation.
While the exact levels of these predictions may be subject to some error, the need certainly arises
for some removal scheme. This removal must suppress power from polarized foregrounds by around four
orders of magnitude in the power spectrum.

\figuremacroW{DeltakGoodK_rm.eps}{0.6}{fig:rm_srcs}{
}{Figure \ref{fig:k3pk_goodk}, with sources perfectly removed}
To investigate the effects of modelling and removing polarized sources, we rerun the simulation,
excluding the brightest polarized sources. Figure \ref{fig:rm_srcs} shows the median value of
several simulations of the $k$-bin nearest $0.25\ h{\rm Mpc}^{-1}$, having removed the brightest
1000, 2000, 5000, and 10,000 sources. These limits in numbers of sources correspond to unpolarized 
flux-limits of 1300, 900, 460, and 240 mJy, respectively. Polarized flux limits are roughly $2\%$ of
these. We remove these sources from treatment A of
the simulation, which includes around 21,000 sources. Despite having removed nearly one-third of
the sources, the leaked power still exceeds $10\ {\rm mK}^2$, the expected level of the 21cm EoR
power spectrum.

To remove enough flux to consistently fall below the expected EoR signal, we need to remove a large 
majority of polarized point sources. We recomputed the simulations with a lower minimum flux (60
mJy), expecting a similar result, but found that we increased the power in this $k$-bin by only one
or two ${\rm mK}^2$. For total power to fall below $10\ {\rm mK}^2$, more sources required removal.
We exclude further investigation of this analysis for three reasons. First, current measurements do
not constrain $dN/dS$ to the levels necessary to accurately model such low-flux sources. Second,
including lower-flux sources does not significantly affect the result that the result that the
expected polarized power spectrum will be of the order of $10^4$-$10^6\ {\rm mK}^2$. Third, the
variance in power from one simulation to the next was large enough that the two treatments of the
simulation --- even with 10,000 sources removed --- could not be considered significantly different.

The onerous levels of source-removal suggest that a different mitigation scheme be considered.
Future instruments may take polarization into consideration in their design. Leakage can be
mitigated with more circular beams, and circular feeds avoid the $Q\to I$ leakage entirely. Even
with existing data, rotation measure synthesis \cite{BrentjensDeBruyn} could potentially provide the
ability to separate sources with distinct rotation measure structure to be separated from the EoR
signal.
