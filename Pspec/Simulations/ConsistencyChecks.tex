\section{Consistency Tests}
\label{sec:Consistency}

Now that we have presented the power spectra, we must check two
things: first, that the two-dimensional $C_\ell$ power spectrum generated by the simulations agrees
with current measurements \cite{Bernardi2009}, and second, that the assumption that uncorrelated
polarization angles is valid.

\subsection{Two-Dimensional Power Spectrum and Diffuse Emission}

\figuremacroW{SimCell.eps}{0.6}{fig:Cell}{
  Black points show the $C_\ell$ power spectrum from \citet{Bernardi2009}. The blue line shows the
  mean $C_\ell$ of several simulations. The shaded, cyan region shows $2\sigma$ limits of the
  distributions for each bin in $\ell$. This shows the consistency between our simulations and
  recent measurements. Treatment A is used for these simulations.
}{Comparison of simulated $C_\ell$ power spectrum with recent measurements}
Figure \ref{fig:Cell} shows the distribution of two-dimensional power spectrum over several
simulations, plotted alongside the $C_\ell$ measurements from \citet{Bernardi2009}. We see
qualitatively that our simulation well obeys the upper limits imposed by the Bernardi measurement.
This agreement helps validate our results.

The estimates of power in Section \ref{sec:sim_results} are dependent on the relative strengths of
diffuse, polarized emission and polarized point sources. We have taken care to agree with
measurements of all polarized emission, but those measurements are uncertain above $\ell\sim300$. We
interpret them as an upper limit. In the limiting case where diffuse emission is the only component
to the polarized sky, this leakage could be suppressed by measuring with a longer baseline, which in
turn measures a lower $\ell$ or $k_\perp$. We have chosen a 30m baseline, which corresponds to
$\ell\approx200$. This choice of baseline length is relatively short for interferometers at these
wavelengths, but falls at the high end of the \citet{Bernardi2009} measurements.

Including additional diffuse emission in the simulation would certainly increase the total power in the
simulation for low $\ell$, but the frequency structure would remain qualitatively the same as point
sources. As we will show in the following section, the correlation of rotation measures and
polarization angles that could be introduced by an extended structure will not significantly affect
the power spectrum. For this reason, we can consider the polarized sky as having two components with
nearly identical footprints in the line-of-sight direction: diffuse and point-like. Both components
will exhibit similar frequency structure, so choice of baseline length will set the relative
weightings of these components. \citet{Bernardi2009} briefly discuss some of the implications of
their measurement of extended structure to the three-dimensional power spectrum in their conclusion,
which agrees with our analysis of point-like structure. We will discuss the qualitative differences
between diffuse and point-like emission in Section \ref{sec:applicable}.

\subsection{Correlating Polarization Vectors}

The analysis of Section \ref{sec:sim_results} neglects known spatial correlations of the rotation
measure distribution \cite{Kronberg2011}. Furthermore, the random drawing of polarization angles
could have a cancelling effect on the visibilities. This neglect could potentially suppress our
estimation of polarized leakage into the power spectrum. 

To investigate these possible effects, we choose rotation measures from the Oppermann map
\cite{Oppermann2012}, with a pointing center at the Galactic south pole --- a reasonable field for
EoR analysis. We then set all polarization angles to zero, maximally correlating polarization
vectors, while still including information of the polarized sky. All other simulation parameters are
identical to treatment A of Table \ref{tab:sims}.

\figuremacroW{CorrelatePspec.eps}{0.6}{fig:corr_sim}{
  A comparison of power spectrum measurements for a treatment of the simulation with correlated
  polarization angles (black), and those from treatment A (gray). As in Figure
  \ref{fig:sim_power_spectra}, the left panel shows the $I$ power spectrum, and the right panel shows the
  $Q$ power spectrum. Three redshift bins are shown, each denoted with a different line style: 9.73
  (solid), 8.33 (dashed), and 7.25 (dot-dashed). The results of simulation A with this simulation
  show that correlating polarization vectors does not affect the power spectrum.
}{Simulation with correlated polarization vectors}
Figure \ref{fig:corr_sim} compares the results of this treatment  with simulation A from Table
\ref{tab:sims}. The power spectrum of this treatment agreees with simulation A at all redshifts and
values of $k$, for both polarizations. This agreement indicates that the spatial correlation of
polarization vectors do not significantly affect the power spectrum. Thus, the assumption in Section
\ref{sec:sim_params} are spatially uncorrelated does not affect the results of these simulations.
