\chapter{Power Spectra} \label{chap:PowerSpectra}

To directly measure the level of polarized power described and simulated in the previous two
sections, we turn to data taken during the EoR2011 observing campaign defined in Table
\ref{tab:deploy}. While the primary objective of this campaign was to measure the mostly unpolarized
signature of neutral Hydrogen in the early universe \cite{Parsons2014, Jacobs2014}, it could not
reach the sensitivity levels required to characterize or even detect the power spectrum. Hence, it
is a useful study for characterizing foregrounds. All four polarization products were correlated specifically to
characterize the level of polarized power which could corrupt 21cm EoR power spectrum measurements.

\figuremacroW{UVCoverage.eps}{1.0}{fig:uv_coverage}{
  (Top Panel) Antenna positions, referenced to the top, left antenna. (Bottom Panel) $uv$-coverage
  for the entire array in black. The $uv$-coverage of the subset of antennae used for this analysis 
  is shown in cyan. To give a sense of scale between the $u$- and $v$-axes, concentric circles with
  radii of $2\lambda$, $5\lambda$, $10\lambda$, $20\lambda$, $50\lambda$, and $100\lambda$ are shown.
  Since the power spectrum is computed for each integration, there is no Earth-rotation synthesis.
}{Antenna positions and $uv$-coverage}
As as brief reminder of Section \ref{sec:deploy} and Table \ref{tab:deploy}, this data was
taken during Winter of 2011 and Spring of 2012, spanning eighty-two nights of observations.
PAPER's configuration for this season was in an $8\times4$ grid. Since the antennae are
arranged in a redundant grid, we can label the subsets of redundant baselines by their grid
spacings. For instance, a baseline composed of two adjacent antennae in the same row can be written
$(0,1)$. Similarly, a baseline composed of two adjacent antennae in the same column can be written
as $(1,0)$. For the results presented here, and in the two sister papers to this work,
\cite{Jacobs2014, Parsons2014}, only baseline types $(0,1)$, $(1,1)$, and $(-1,1)$ are considered. 
The row-spacings in this configuration were 30m, chosen to reduce the extent of the foreground wedge 
without incurring the any antenna-to-antenna cross talk. The column spacings were 4m, maximizing 
the redundancy between baselines $(0,1)$ and $(1,1)$ or $(-1,1)$. Figure \ref{fig:uv_coverage} shows
a map of the antenna spacing, as well as the $uv$-coverage of the array. The four-degree offset of
the columns from true north is due to the projection of the tangent plane to the Earth at the
array location to UTM plane 34.

Data was taken continuously from 6pm SAST until 6am SAST each night during this campaign. To remove
effects of the sun, we only consider data when the sun is below -5$^\circ$ in altitude. We restrict
the number of nights due to some systematic errors which corrupt the data after April 1, 2012
--- Julian Date 2456018. A catastrophic event occurred on this date,\footnote{Because I observed
this data remotely, I can only guess, but my money is on a lightning strike.} and most of the data
taken after it was unusable. 82 nights of data survived quality checks and are used for this
analysis. 

\figuremacroW{FOV.eps}{1.0}{fig:fov}{
  Effective integration time per pointing (Equation \ref{eq:teff}), as a function of position on the sphere.
}{Effective integration time for each pointing on the sphere}
We focus our efforts on the range in LST from 1h00m until 8h00m, which maximizes the total
integration time available, but minimizes the effects of systematics. Since PAPER is a drift-scan
array, this sets both the pointing and the field of view. Figure \ref{fig:fov} shows a map of effective 
integration time per pointing, defined as 
\begin{equation}
  t_{eff}(\alpha,\delta) = \sum_{i} t_{int} A(\alpha,\delta,t_i)
  \label{eq:teff}
\end{equation}
where the sum extends over the each integration in the season. This metric is defined to give the total 
integration time when integrated over position on the sphere. The total field of view surveyed is 2.39 sr. 

Finally, we restrict the final analysis to two bands, though the full range of frequencies is used 
throughout much of the analysis. We label the lower one Band I, and the upper, Band II. Band II 
corresponds with that used for the results in \citet{Parsons2014}, and Band I is chosen to correspond 
to the lowest band in \citet{Jacobs2014}. Many of the observational parameters defining these two bands 
are presented in Table \ref{tab:obsparams}.

\begin{table}\begin{center}
  \begin{tabular}{c c c c c c c}
    Band & $\nu_0$ [MHz] & $\Delta\nu$ [MHz] & $z$ & $A_{eff}\ [{\rm m}^2]$ & $T_{sys}$ [K] & $\mathcal{A}_-/\mathcal{A}_+$ \\
    \hline\hline
    I  & 126 & 7.9 & 10.3  & 4.47 & 836 & $3.3\times10^{-3}$ \\
    II & 164 & 9.4 &  7.66 & 5.80 & 505 & $2.2\times10^{-2}$
  \end{tabular}
  \caption[Observational parameters]{\label{tab:obsparams} Observational parameters for the two sets of power spectra
  presented. Given are the central frequency $\nu_0$, the effective bandwidth $\Delta\nu$, the
central redshift of observation $z$, the effective area of the antennae $A_{eff}$, and the ratio
which paramterizes $Q\to I$ leakage, $\mathcal{A}_-/\mathcal{A}_+$ (Section \ref{sec:BeamLeakage}).}
\end{center}\end{table}
This Chapter outlines the processing and analysis of the polarization properties of this
data. Section \ref{sec:data} describes the analysis and quality checks this data underwent; Section
\ref{sec:results} presents the power spectra of these data, and finally, Section \ref{sec:updates}
gives the physical properties of polarized point sources that these data imply.

\section{Data Processing}\label{sec:data}

\subsection{RFI Excision}
We begin with an excision of RFI from the raw data, a three step process. First, we flag known
frequency channels containing nearly constant RFI --- for example, the 137 MHz bin contains the
continuous signal from a constellation of communications satellites. Next, we difference the data in
time and frequency, flagging the data which produces $6\sigma$ outliers. Finally, we remove a
fiducial foreground model, the process of finding this model is described in Section
\ref{sec:rm_fg}, and flag $4\sigma$ outliers of the residuals. The flags generated from
this process were used in the production of Figure \ref{fig:RFI}. A single set of flags is generated
for all times and frequencies for the entire array for each night of data taking.

\subsection{Compression}\label{sec:compress}

The volume of raw data generated in the EoR2011 season exceeds 10 TB, which is unwieldy for 
the level of computation required. While the relatively short integration times of 10s and
relatively narrow channel widths of 50 kHz are useful for reducing the attrition of data due to RFI
excision, these rates highly oversample both the frequency structure of foregrounds and EoR signal
and the temporal structure of anything tied to the sky. To remedy the abundance of oversampled ---
and thus redundant --- data, we employ a compression technique, first described in
\citet{Parsons2014}, which critically samples the data in both time and frequency.

This compression algorithm hinges on two results from Section \ref{sec:DDR}:
\begin{enumerate}
  \item The delay of a smooth-spectrum point source is restricted to the range $\tau \le |b|/c$,
    where $|b|$ is the baseline length.
  \item Similarly, the fringe-rate of a source is restricted to the range $ -(\nu b_E/c) \omega_\oplus
    \cos\delta_0 \le f \le (\nu b_E/c) \omega_\oplus$, where $b_E$ is the east-west component of the
    baseline, $\omega_\oplus$ is the angular speed of the Earth's rotation, and $\delta_0$ is the
    latitude of the array.
\end{enumerate}
These two properties of drift-scan interferometers allow us to set limits on the fringe rate and
delay at which celestial emission can enter the signal --- this in turn allows us to set minimum
integration times and channel widths which preserve that emission. 

To set this minimum sampling rate in time, we inspect the maximum fringe rate, $ (\nu
b_E/c)\omega_\oplus$. We define the delay rate as the frequency-integrated fringe rate --- this
allows us to simultaneously compute this alongside the delay. Hence, the maximum delay rate allowed
by celestial emission is $(b_E/c)\omega_\oplus$. The maximum delay rate across the entire PAPER
array occurs in the 210m east-west baselines (between the leftmost and rightmost columns in Figure
\ref{fig:uv_coverage}): 9.6 mHz. The Nyquist-Shannon sampling theorem dictates that an sampling time of
33s can completely describe this structure.

Though the maximum delay can be set as low as the horizon, we intend to preserve supra-horizon modes
containing high-$k_{||}$ EoR modes. However, we achieve maximum sensitivity to these modes on the
shortest baselines, and use the longest only for foreground characterization. Hence, we set the
limit in delay to the horizon limit of the longest baseline in the array. This requires a sampling
rate in frequency of 713 kHz. Setting the maximum delay allows supra-horizon modes to enter into the
visibilities of short-baselines, including cosmological modes up to $0.38\ h{\rm Mpc}^{-1}$.

\figuremacroW{ddr_compression.eps}{0.6}{fig:compress}{
  Delay / Delay rate transform of one days' worth of raw PAPER visibilities from a 30m baseline. The
  relatively fine sampling in frequency and time result in large ranges of delay and delay rate
  (respectively). A dashed, cyan box shows the skypass filter in delay, and the magenta, in delay
  rate. The fluxscale, in $\log_{10}({\rm Jy})$, is shown on the right. The skypass filters are
  designed to preserve all smooth-spectrum, celestial emission for the entire PAPER array, with
  baselines ranging from 30m (shown) up to 300m. For this short baseline, sky emission is contained
  within a relatively small range around 0 delay and 0 delay rate.
  Figure credit: \citet{Parsons2014}
}{Range of skypass filters for delay/delay rate compression, from \citet{Parsons2014}}
Figure \ref{fig:compress} shows the extent of the skypass filters --- defined as one on the
intervals that contain emission (shown in the preceding discussion) and zero elsewhere. The skypass
filters are shown atop the delay/delay rate transform of a visibility, confirming the claims about
foreground signal's extent in these directions made in the text.

One serious hurdle to overcome in this compression process is the spectral and temporal structure
introduced by nonuniform RFI flagging. To accommodate the scattering of signal into high delay/delay
rate bins, we first deconvolve the data by the sampling function using a variation on the CLEAN
algorithm, discussed in Section \ref{sec:rm_fg}. The difference between this implementation of the
algorithm and the foreground-removal strategy discussed in Section \ref{sec:rm_fg} is that we add
the CLEAN components back into the residual spectra. This is the most computationally costly step.

Once the CLEAN deconvolution has been performed, we simply decimate the data, re-sampling the data at
the rates described in the preceding paragraphs. While we could sample each baseline type with its
own integration time and channel width, we set the limits based on the longest baselines --- this
both ensures a conservative application of this new procedure and allows for ease of data analysis
and storage.  

We implement the compression algorithm each night on the data using a 35 node computer cluster
located on site. This allows us to perform all preprocessing steps up until this point, including
compression, in real time as the data is taken. This algorithm reduces both the data rate and data 
volume by a nearly factor of twenty, reducing storage costs and required computational power.

\subsection{Crosstalk Removal} 

For our purposes, crosstalk may be defined as a additive offset to the visibilities, which is stable
on long timescales. To remove crosstalk, we simply subtract the nightly average of each baseline
from each integration of that baseline.

\subsection{Calibration}

Calibration is a two step process. First, we solve for the antenna-based gains and delays which
enforce redundancy among redundant baselines. This procedure is described in greater detail in
Section \ref{sec:redcal}. We treat the $xx$ and $yy$ polarizations of the array separately in this
analysis, linking the two calibrations with a cross-polarization delay and the assumption that all
calibration terms are antenna-dependent. Next, we solve for the remaining four calibration terms 
--- an overall flux scale for the $x$ and $y$ polarizations, and the delay of fiducial baselines ---
by fitting visibilities to a model of Pictor A \cite{Jacobs2013b}. 

We compute the calibration parameters using a relatively small amount of data --- for two hours
when Pictor A is overhead during a single day. We then apply these calibration terms to the
entire seasons' data. \citet{JacobsPHD} and \citet{Parsons2014} have shown that calibration terms
remain constant for long timescales, and we take advantage of the remarkable stability of PAPER's
calibration\footnote{To me, that PAPER's calibration terms are stable for these long timescales is
  one of the more magical things about the instrument. I've even successfully applied calibration 
  terms from one observing season to another!} to ease the computational burden of calibration. We
could solve for all calibration terms on smaller timescales but we have found that this only causes
slight improvements to the variance in our data. In practice, the errors caused by such a cavalier
calibration effort can be absorbed into the uncertainty of the data, causing a roughly $5\%$ 
increase in $T_{sys}$.

\subsection{Foreground Removal}\label{sec:rm_fg}

The final step before averaging multiple days is to remove a foreground model from the raw
visibilities. Rather than removing a number of previously-identified sources from the data 
--- this leaves us vulnerable to calibration errors as well as errors in the primary beam 
model --- we employ a non-parametric method to remove foregrounds modelled on each visibility 
itself. Since most foreground sources are smooth spectrum and can be modelled as point sources, we
model them as delta functions in delay (See Section \ref{sec:DDR}). Rather than fitting for antenna 
gains and delays, a primary beam model, and the source flux at each frequency, we only fit for 
a single parameter: the flux in a given delay mode of a visibility.

To fit these fluxes, we use a version of the CLEAN algorithm \cite{Hogbom1974}, reduced to one
dimension, and extended to allow complex flux values. We find the peak in the delay
spectrum of a single integration for one baseline, and subtract the kernel of the sampling function,
weighted by the flux of that peak, from the delay spectrum. We iterate this process until the
variance of the residual spectrum is $10^{-8}$ times that of the original spectrum. We
restrict the algorithm to peaks found within the horizon limits, described in Section \ref{sec:DDR},
which enforces that smooth-spectrum foreground sources be removed. This procedure both removes
foregrounds and deconvolves from the spectral sampling function created from the flagging of RFI.

In the limit where all spectral bins contain data, this is simply a notch filter which nulls
inter-horizon delay modes.

\subsection{Averaging Multiple Days}\label{sec:LSTbin}

As a final excision of spurious signals (most likely due to RFI), for each day, we flag outlying
measurements in each bin of LST and frequency. We use measurements of $T_{sys}$ outlined in the Section
\ref{sec:Tsys} to estimate the variance in each bin, flagging $3\sigma$ outliers.

If the data followed a complex normal distribution, consistent with pure, thermal noise, then we
would expect this procedure to flag one measurement per frequency/LST bin, causing a slight
miscalculation of statistics post flagging. Most notably, this causes an underestimate in the
variance of the power spectrum. To counteract this effect, we calculate the ratio of the variance of
a normal distribution truncated at $\pm3\sigma$ to the variance of its parent distribution
($97.3\%$). Henceforth, we will increase all errors in the power spectrum by a factor of
$1.03\approx1/97.3\%$ to accommodate for this error. 

We compute the mean of the RFI-removed data for each bin of LST and frequency, creating a dataset
comprised of a single, fiducial day. We continue analysis on this averaged dataset. 

\subsection{Final Processing}

After visibilities are averaged in LST, a final round of crosstalk removal is performed. Again, we
simply subtract the daily average from the data. Much of the crosstalk lies beneath the sensitivity
level of only one day's worth of data, and appears in the LST-averaged dataset. Recomputing the
mean with the increased sensitivity of an averaged dataset allows for a more accurate removal.

In the penultimate processing step, we pass the data through a second low-pass filter in time. Sections
\ref{sec:DDR} and \ref{sec:compress} describe the celestial limits of the fringe rate for drift-scan
arrays ($f$) as $b_E\omega_\oplus\cos\delta_0 \le f \le b_E\omega_\oplus$, where $b_E$ is the
east-west component of the baseline, $\omega_\oplus$ is the angular velocity of the Earth's
rotation, and $\delta_0$ is the latitude of the array. We filter the data in time using a boxcar
filter, defined as one on $0\le f\le b_E\omega_\oplus$ and zero elsewhere. While this filter does
null some celestial emission (roughly the area between the south celestial pole and the horizon),
its effect is small, since the primary beam heavily attenuates these areas of the sky. We null these
fringe rates as an additional step of cross-talk removal.

Finally, we rotate the linearly polarized visibilities into Stokes visibilities, defined in
Equation \ref{eq:def_stokes_vis}.

\subsection{System Temperature} \label{sec:Tsys}

Alongside the calculation of statistics for binning in LST and frequency, we take advantage of the
nightly redundancy as a check on the data. Since PAPER is a tracking array, measurements taken at
the same LST on different nights should be totally redundant. This redundancy allows us to measure
the system temperature via fluctuations in signal in the same LST bin from day to day.

First, we compute the variance in each frequency and LST bin over all nights of data
($\sigma_{Jy}^2(\nu,t)$), and convert this variance into a measurement of the system temperature
$T_{sys}$\nomenclature[Rt]{$T_{sys}$}{System Temperature}. This measurement is totally independent of
the following power spectrum analysis, and can be used to quantify the level of systematic and
statistical uncertainty in the power spectra. It compliments measurements of $T_{sys}$ in
\citet{Parsons2014} and \citet{Jacobs2014}. The variance computed in each LST/frequency bin is
converted into a system temperature in the usual fashion:
\begin{equation}
  T_{sys}(\nu,t) = \frac{A_{eff}}{k_B}\frac{\sigma_{Jy}}{\sqrt{2\Delta\nu t_{int}}},
  \label{eq:tsys}
\end{equation}
where $A_{eff}$ is the effective area of the antenna (see Figure \ref{fig:beam}), $k_B$ is the
Boltzmann constant, $\Delta\nu$ is the channel width, and $t_{int}$ is the integration time of the
LST bin.

\figuremacroW{Tsys_all.eps}{1.0}{fig:tsys}{
  System temperature in Kelvin as a function of LST and frequency $\nu$, calculated by Equation
  \ref{eq:tsys}. Black boxes enclose the range in LST and $\nu$ used to compute the power spectra.
}{System temperature as a function of LST and $\nu$.}
Figure \ref{fig:tsys} shows the measured system temperature for each frequency and LST bin collected
during the EoR2011 observing season. To further summarize our data's variance, we can average
$T_{sys}(\nu,t)$ over the time- and frequency-axes. The frequency-averaged system temperature is
computed as 
\begin{equation}
  \langle T_{sys}\rangle(t) \equiv \frac{\int_{\Delta\nu}
  W(\nu)T_{sys}(\nu,t)\D{\nu}}{\int_{\Delta\nu}W(\nu)\D{\nu}},
  \label{eq:tsys_avg}
\end{equation}
where $W(\nu)$ is the spectral window function, and the integral is computed over the frequency band
$\Delta\nu$. For our analysis, we use a Blackman-Harris window function \cite{Harris1978}, chosen to
maximally suppress sidelobe levels. A similar expression may be written for the time-axis, where our
window function is simply the number of redundant samples in each frequency channel.

\figuremacroW{Tsys_lst.eps}{0.6}{fig:tsys_avg}{
  (Top Panel) band-averaged system temperature (Equation \ref{eq:tsys_avg}) as a function of LST for
  Bands I and II in black and blue, respectively. The shaded grey region indicates the range in LST
  used to compute the power spectra. (Bottom Panel) Time-averaged system temperature, averaged over
  LST 1h00m until 8h00m. The shaded grey and blue regions show the spectral window functions for
  Bands I and II, respectively. 
}{$T_{sys}$, averaged in frequency and time}
Figure \ref{fig:tsys_avg} shows the system temperature averaged over frequency and LST ranges used
to compute the power spectra. $T_{sys}$, averaged in both frequency and time for both bands are
reported in Table \ref{tab:obsparams}

\section{Power Spectra}\label{sec:results}

Now that the data are processed and averaged, we begin computing power spectra by a two-step
process. First, we remove off-diagonal covariances from the correlation matrix of two baselines ---
this results in a power spectrum for each integration time of the fiducial, averaged day and for
each baseline pair. Next, we bootstrap multiple times and baseline pairs to characterize the
statistics of the distributions of power spectra.  

\subsection{Covariance Removal}\label{sec:covariance}

Even after having undergone several layers of RFI excision and crosstalk removal, the data still
show large covariances between delay bins. These covariances dominate the averaged power spectrum
despite varying wildly between baseline pairs. We remove them via the covariance removal strategy
described in Appendix C of \citet{Parsons2014}.

This strategy essentially diagonalizes the average covariance matrix between all baseline pairs.
Since we expect all redundant measurements to see the same sky, and we expect all $k$-bins of the
power spectrum to be independent, then we expect sky signal to appear in the diagonal elements of
the covariance matrix between two delay-transformed visibilities. By measuring the full covariance 
matrix of all baseline pairs, we can estimate the instrumental systematics which would leak signal
from one $k$-bin to another. By inverting the mean covariance matrix of all baseline pairs and
dotting this into a delay-transformed visibility, we remove our best guess at the covariances in
that visibility. 

Once we estimate and remove covariances from the delay-transformed visibilities, we may proceed with
power spectrum estimation. We direct the reader to \citet{Parsons2014} for a more detailed discussion. 

\subsection{Results}

\figuremacroW{PQ_vs_t.eps}{1.0}{fig:pq_vs_t}{
  Power spectra, in units of $mK^2\ (h{\rm Mpc}^{-1})^3$ shown for each $k_{||}$ and LST measured.
  $k_{||}$ modes within the horizon for the 30m baselines used are masked. The left panel show $I$,
  the middle panel, $P_P=P_Q+P_U$, and the right panel shows $P_V$. An excess of power in $P_P$
  below LST 4h30m could indicate polarized emission. The excess at $k_{||} \approx 0.35\ h{\rm
  Mpc}^{-1}$ between LST 6h00m and 8h00m could be generated by a 170 mJy source with rotation measure
  between 42 and 59 ${\rm m}^{-2}$.
}{Power spectra in $I$, $P$, and $V$, vs. LST and $k_{||}$}
The covariance removal described in the previous section projects the delay-transformed
visibilities into a basis in which the covariance between two redundant baselines is diagonal, and
then computes the power spectrum from the projected delay spectra. This procedure produces an
estimate to the power spectrum for each LST bin and baseline type. To measure the uncertainties in
the time-dependent power spectra, we bootstrap over groups of redundant baselines. Figure
\ref{fig:pq_vs_t} shows the linearly polarized power spectra from unpolarized emission ($I$),
linearly polarized emission ($P = Q+iU$), and circularly polarized emission ($V$). as functions of
LST, computed via this bootstrapping.

There are two features in $P_P$ worth noting. First is the excess of emission at $0 \lesssim k_{||}
\lesssim 0.2\ h{\rm Mpc}^{-1}$, between right ascension 1h00m and 4h30m. That the excess survives
the LST averaging over 82 days indicates that it is fixed to the sky, and that it exceeds its
corresponding $k$-bins in $P_I$ indicates that it is polarized. Furthermore, it roughly corresponds
with the diffuse, polarized power shown in \citet{Bernardi2013}, which takes its minimum value at 
around LST of 5h00m.

The second feature of Figure \ref{fig:pq_vs_t} that we will comment on is the track of excess power
from fight ascension 6h00m to 8h00m, at $k_{||} \sim 0.35\ h{\rm Mpc}^{-1}$. This type of excess
power could be generated by a polarized point source. This stripe satisfies the two criteria put forth to
indicate the excess at lower right ascension could be polarized emission, and it appears to be
localized in $k_{||}$, a feature of polarized point sources behind a single Faraday screen.

What properties of a polarized point source would be necessary to generate the excess shown? To
answer this question, we describe the source spectrum with three parameters: a flux $S$, a
geometrical delay $\tau_g$, and a rotation measure $\Phi$. The power spectrum of this point sources
$P_1(k)$, taken from Equation \ref{eq:def_vis2pk} becomes 
\begin{equation}
  P_1(k) \approx \left(\frac{\lambda^2}{2k_B}\right)^2\frac{X^2Y}{\Omega}S^2
    \delta\left(k_{||} - \dfdx{k_{||}}{\eta}\tau_g - k_{leak}(\Phi)\right),
\end{equation}
where $dk/d\eta$ is the linear conversion from delay into $k_{||}$ ($1/Y$ of Equation \ref{eq:def_Y}),
and $k_{leak}(\Phi)$ is the $k_{||}$ mode most infected by rotation measure $\Phi$ (Equation
\ref{eq:bad_k}). All other terms agree with Equation \ref{eq:def_vis2pk}.The bandwidth in the
denominator of Equation \ref{eq:def_vis2pk} is removed to properly normalize the delta function.
Note the important result from Section \ref{sec:onesource} that a rotation measure component to a
spectrum \emph{adds} to the normal position of a source in $k_{||}$.

The potential source peaks in power at right ascension of 6h52m at a $k_{||}$ value of $0.345\ h{\rm
Mpc}^{-1}$. Its power peaks at $4.62\times10^8\ {\rm mK}^2(h^3{\rm Mpc}^{-3})$. A point source whose
right ascension is between 6h00m and 8h00m, with an apparent flux of 173 mJy, whose rotation measure
is between 42 and 59 m$^{-2}$ satisfies the requirements to generate this type of emission. To
confirm the existence of such a source would require follow-up observations with an imaging array,
but the excess power at that location in the $P_P$ spectrum is well-described by such a source.

\figuremacroW{Pspec_2x2_37_70.eps}{1.0}{fig:pspec2x2_lo}{
  Spherically averaged power spectra for the four Stokes parameters: $I$ in the top left panel. $Q$ 
  in the top right, $U$ in the lower left, and $V$ in the lower right. The data from Band I are shown. 
  Error basrs show the 98\% confidence intervals derived from bootstrapping over all samples in LST 
  and all redundant baselines. Dashed, cyan lines show the theoretical level of thermal fluctuations, 
  with $T_{sys}$ calculated in Section \ref{sec:Tsys}. 
}{Power spectra for all Stokes parameters in Band I}
\figuremacroW{Pspec_2x2_110_149.eps}{1.0}{fig:pspec2x2_hi}{
  Same as figure \ref{fig:pspec2x2_lo}, for Band II.
}{Power spectra for all Stokes parameters in Band II}
Figure \ref{fig:pspec2x2_lo} shows the power spectra of the four Stokes parameters  in Band I, and
Figure \ref{fig:pspec2x2_hi} sohws that of Band II. Sensitivity limits using the $T_{sys}$ computed
in Section \ref{sec:Tsys} and the sensitivity calculations of \citet{PAPERSensitivity} and
\citet{Pober2014} are shown in dashed, cyan lines. The noise level of these power spectra are
computed by examining the spread of bootstrapped power spectra, where we bootstrap over both
redundant baselines and LST samples. These noise levels are considerably higher than the predictions
from the $T_{sys}$ of Section \ref{sec:Tsys}. The increase in $T_{sys}$ is most
likely due to calibration errors and systematics not targeted by the covariance removal.

The level of leakage predicted from the arguments of Section \ref{sec:BeamLeakage} show that in the
lowest $k_{||}$ bins, the $I$ power spectrum cannot be dominated by $Q\to I$ leakage. The levels of
polarized leakage in $P_I$, to an order of magnitude, are $10^3\ {\rm mK}^2$ in Band I, and $10^2\
{\rm mK}^2$ in Band II. These levels are well below the systematics which dominate the lowest
$k_{||}$ bins of the $I$ power spectrum, and are also well below the levels of $Q\to I$ leakage 
predicted by the simulations in Section \ref{sec:sim_results}. We will investigate this 
further in Section \ref{sec:updates}. 

\subsection{Ionospheric Effects}
Daily changes in the Faraday depth of the Earth's ionosphere could potentially attenuate polarized
signal. As the total electron content (TEC)\nomenclature[Zt]{TEC}{Total Electron Content of the
ionosphere} varies, it modulates the incoming polarized signal by some Faraday depth that which 
is a function of both the local TEC of that time, and the strength of the Earth's magnetic field. 
Though we assume visibilities are redundant in LST, they do have slight variations due to the 
variable TEC of the ionosphere. Thus, averaging in LST could result in some attenuation of signal.

To quantify this, we first assume that the ionospheric TEC is constant over the PAPER beam, and
assume that from day to day, the Faraday depth of the ionosphere is a random variable, based on the
TEC measurements of \citet{Datta2014}. For now, we neglect the day-to-day correlations, though we
can check the effects of any correlation later.

We begin by writing the LST-averaged visibility as the Faraday depth-weighted sum of otherwise
redundant visibilities:
\begin{equation}
  V' = \frac{1}{N}\sum_i e^{-2\Phi_i\lambda^2}V,
\end{equation}
where $\Phi_i$ is the ionospheric Faraday depth from day $i$ and $V$ is the redundant component of
the visibilities. Using this expression, we compute the magnitude of the rotated visibilities, which 
is proportional to the power spectrum,
\begin{equation}
  P' = |V'|^2 = \frac{1}{N^2}\left(\sum_{i,j}e^{-2i(\Phi_i-\Phi_j)\lambda^2}\right)|V|^2.
\end{equation}
When $i=j$, the term in parentheses becomes one, and the $i,j$ component of thes sum is the
conjugate of the $j,i$ component. This allows us to rewrite the sum in terms of the $i=j$ component,
and the $j>i$ components, now written as cosines:
\begin{equation}
  \sum_{i,j} e^{-2i(\Phi_i-\Phi_j)\lambda^2} 
    = N+2
    \sum_{i>j}\cos\left\{2(\Phi_i-\Phi_j)\lambda^2\right\}.
  \label{eq:ionosphere}  
\end{equation}
In the limit where all values of $\Phi_i$ are equal, the second term becomes $N(N-1)/2$, the number
of $i,j$ pairs with $i>j$. This produces the desired result that with no daily fluctuations in
ionospheric Faraday depth, there is no effect on the signal. In the limit of totally uncorrelated
data (i.e. $\langle\Phi_i\Phi_j\rangle\propto\delta_{ij}$), this term takes its minimum value,
maximally attenuating the measured power spectrum. Hence, considering uncorrelated $\Phi_i$ gives
the worst-case scenario.

To estimate the level of ionospheric attenuation, we estimate the attenuation factor in Equation
\ref{eq:ionosphere}. Using typical TEC values of $6\times10^{16}\ {\rm m}^2$ and a typical value 
for the Earth's magnetic field at the PAPER site\footnote{{\tt
http://www.ngdc.noaa.gov/geomag/magfield.shtml}}, we calculate a rotation measure for each day, and
estimate the attenuation factor, and estimate the attenuation factor for eighty-two days, as the data
was averaged over that period of time.

This procedure yields a distribution of values for the attenuation factor which peaks at $88\%$.
Considering that we have neglected day-to-day correlations of the TEC, which increases this factor,
decreasing the level of attenuation, we assume that attenuation due to the ionosphere is negligible
and do not adjust our results for it.

\section{Updated Polarization Fractions}\label{sec:updates}

\subsection{Scaling the Simulations}

\figuremacroW{CompareSim.eps}{0.6}{fig:compare_sim}{
  Top row: measured and simulated power spectra for $I$ (left) and Q (right). Measured power
  spectra are in black, simulated values are in blue (median value), cyan ($68\%$ confidence
  interval), and light cyan ($95\%$ confidence interval). Simulations are generated as in
  Chapter \ref{chap:Simulations}, with a mean polarized fraction of $2.01\%$. Bottom row: Same as
  the top row, for Band II.
}{A comparison of simulations and measurements.}
Figure \ref{fig:compare_sim} compares the measured $Q$ and $I$ power spectra to those simulated
in Chapter \ref{chap:Simulations}. Since the measured values consistently disagree with the
simulations, we can constrain the input parameters to the simulations, beginning with a simple
scaling relation:
\begin{equation}
  P_k = x^2S_k,
  \label{eq:def_x}
\end{equation}
where $P_k$\nomenclature[Rp]{$P_k$}{Power spectrum in the $k^{{\rm th}}$ bin} is the measured $Q$ 
power spectrum in the $k^{{\rm th}}$ bin,
$S_k$\nomenclature[Rs]{$S_k$}{Simulated power spectrum in
the $k^{{\rm th}}$ bin} is the simulated power spectrum in that bin, and
$x^2$\nomenclature[Rx]{$x$}{Scale factor between the simulated and measured power spectra} is the 
scale factor between the two. We choose to use a scale factor of $x^2$ rather than $x$ in order to
facilitate the interpretation of $x$ as an adjustment to the mean polarized fraction of point
sources, as we will soon discuss.

For the duration of this section, we will approximate the measured power spectrum $\hat{P}_k$ as
normal, random variables, 
\begin{equation}
  \hat{P}_k \sim \mathcal{N}(P_k, \sigma_k^2),
\end{equation}
with mean $P_k$ and variance $\sigma_k^2$, derived from the distribution of bootstrapped power
spectra.

We find the scale factor $x$ that best fits Equation \ref{eq:def_x}, and then interpret its physical
meaning. The likelihood of drawing a simulated power spectrum
$\mathcal{S}$\nomenclature[Rs]{$\mathcal{S}$}{Set of all bandpowers in a simulated power spectrum} by
a factor $x^2$ given the measured data
$\mathcal{D}$\nomenclature[Rd]{$\mathcal{D}$}{Set of all
bandpowers in a measured power spectrum} is 
\begin{equation}
  P(x,\mathcal{S}|\mathcal{D}) 
  \propto \exp\left\{-\frac{1}{2}\sum_k\frac{|x^2S_k - P_k|^2}{\sigma_k^2}\right\},
\end{equation}
where the sum extends over all available values of $k$.

By marginalizing over $\mathcal{S}$, we find the likelihood of $x$:
\begin{equation}
  P(x|\mathcal{D}) \propto \int P(x,\mathcal{S}|\mathcal{D})P(\mathcal{S})\D{\mathcal{S}}.
  \label{eq:likelihood}
\end{equation}
Here, $P(\mathcal{S})$ is the joint probability of all $k$-bins of the simulation, i.e. $P(S_0,
\dots, S_n)$ for $k$-bins labelled 0 to $n$, and $\D{\mathcal{S}}$ denotes the $n$ values of
$\mathcal{S}$ over which we integrate. We compute the integral in Equation \ref{eq:likelihood} 
by the Monte Carlo technique, sampling $\mathcal{S}$ from different instances of the simulation. 
This encapsulates both the probability distribution functions of each $S_k$ and the covariances 
between $k$-bins in $\mathcal{S}$. To insulate the result from potentially damaging effects of 
the foreground removal (Section \ref{sec:rm_fg}), we do not consider $k$-bins within the horizon 
in this integral. 

To find the most likely value of $x$ which would produce the measured power spectrum, we turn to
Bayes theorem, $P(\mathcal{D}|x) \propto P(x)P(x|\mathcal{D})$, where $P(\mathcal{D}|x)$ is the
posterior distribution of $\mathcal{D}$ and $P(x)$ is our prior on $x$. Since $x^2$ is a scale
factor, we choose to use Jeffrey's prior in $x^2$, which sets $P(x) \propto 1/x$.

\figuremacroW{Px.eps}{0.6}{fig:Px}{
  Posterior distributions for the data, given scale factor $x$. Blue shows that from Band I; cyan,
  from Band II; and black shows the joint posterior from both bands. Moments of these distributions
  are summarized in Table \ref{tab:x}.
}{Posterior distribution of $\mathcal{D}$ given $x$.}
\begin{table}\begin{center}
  \begin{tabular}{ c c c c }
    Band & $\bar{x}$ & $\sigma_x$ & Implied Mean Polarized Fraction 
    \\
    \hline\hline
    I & 0.99 & 0.002 & $2.0\times10^{-3}$
    \\
    II & 0.247 & 0.002 & $5.0\times10^{-3}$
    \\
    Both & 0.108 & 0.001 & $2.2\times10^{-3}$
  \end{tabular}
  \caption{\label{tab:x}Moments of $P(\mathcal{D}|x)$.}
\end{center}\end{table}
Measurements from the different bands can be summarized into a joint posterior by simply computing
the product of the posterior of each band. This assumes that each band is independent, a reasonable
assumption given the high level of noise in the measured power spectra. Figure \ref{fig:Px} shows
the posterior distributions of $\mathcal{D}$ given $x$ for Bands I and II, alongside the joint posterior. 
The moments of the three distributions are summarized in Table \ref{tab:x}.

\subsection{Why is $x$ Related to the Polarized Fraction?}

As mentioned in the previous section, we interpret the scale factor $x$ as an adjustment to the mean
polarized fraction of point sources. The simulations parameterize each point source with a polarized
fraction $p$, an unpolarized flux $f$, and a rotation measure $\Phi$. All sources are given a
spectrum $pf\exp\{-2i\Phi\lambda^2\}$, where $\lambda^2$ is the squared wavelength. This is a
simplified account of Equation \ref{eq:sim_vis}, but encapsulates the relevant quantities for this
discussion. Since the source counts are well measured at these frequencies \cite{Hales1988}, and
rotation measures are also well-measured \cite{Oppermann2012} and independent of frequency, we
regard the distributions of these two quantities to be fixed. Hence, any constraints we place on
these simulations can be considered as updates to the distribution of polarized fraction.

The amplitude of the power spectrum in the simulations ($S_k$) can be expressed in terms of the
source fluxes $f_i$, and the polarized fractions $p_i$.,
\begin{equation}
  S_k \sim \sum_{i,j}p_ip_jf_if_j \equiv \bar{p}^2\sum_{i,j}\pi_i\pi_jf_if_j,
\end{equation}
where we have defined $\pi_i \equiv p_i/\bar{p}$ as the ratio of a single polarization fraction
$p_i$ to the mean, $\bar{p}$. Hence, the simulated power spectra are proportional to the mean
polarized fraction squared, $\bar{p}^2$, and we can interpret the scale factor $x$ as the fractional
change in the mean polarized fraction.

Table \ref{tab:x} gives the implied mean polarized fraction of point sources for the two bands and
the joint posterior. In these simulations, we drew polarized fractions from a distribution with a
mean of around $2\%$, and now we can set a limit about an order of magnitude lower. 

\figuremacroW{CompareSim_MLE.eps}{0.6}{fig:compare_sim_mle}{
  Same as Figure \ref{fig:compare_sim}, but using a polarized fraction distribution scaled by
  $x=0.108$, the maximum likelihood value of $x$ using both bands.
}{Figure \ref{fig:compare_sim}, with a mean polarized fraction of $2.2\times10^{-3}$.}
Figure \ref{fig:compare_sim} shows updated simulations using the implied polarization fraction of
the joint posterior from Figure \ref{fig:Px}: $2.2\times10^{-3}$. The simulated $Q\to I$ leakage now
lies between 10 and 100 ${\rm mK}^2$ in Band II, around the expected level of the 21cm EoR power
spectrum at redshift 7.

\subsection{On the Applicability of the Simulations}\label{sec:applicable}

We now turn our attention to a qualitative discussion of the diffuse emission found in
\citet[][B13]{Bernardi2013} and \citet[][J14]{Jelic2014}, and how applicable the simulations are to
the polarized emission found in those measurements, which is largely characterized as being diffuse
and as having low rotation measures. We will argue that the simulations may apply to both these two
measurements, and also that the simulations are a valid point of comparison to the measurements made
in Section \ref{sec:results}. This discussion builds on Chapter \ref{chap:Simulations}, and justifies
the use of the simulation as a tool for understanding the power spectra presented in Section
\ref{sec:results}.

Both B13 and J14 show diffuse, weakly polarized emission found at relatively low rotation measure
($|\Phi| \lesssim 25\ {\rm m}^{-2}$. These measurements differ from the input sources of the
simulations in two ways: in the choice of rotation measures included, and in the spatial
correlation of power. We will discuss these in turn.

The simulations sample rotation measures from the entire \citet{Oppermann2012} map, rather than
restricting to a particular field of view. Two effects may arise from such a generality. First, the
inclusion of large rotation measures could scatter power to larger $k$ in the simulations than in
reality, and second, uncorrelated polarization vectors in the simulation could have a depolarizing
effect. This concern was addressed in Section \ref{sec:Consistency} by computing the simulation with
all rotation measures drawn from a pointing at the galactic south pole (coincidentally, the B13
field), with maximally correlated polarization vectors. The result of this test was identical to the
random drawing (Figure \ref{fig:corr_sim}). The reproduction of power spectra between the two
simulations indicates that rotation measure is not a dominant factor in determining the shape or the
amplitude of the power spectrum of polarized emission. 

The rotation measures in B13 and J14 are considerably lower than those sampled in the simulations.
Since there is overlap between our field and the B13 field, our measurements include lower
rotation measures than sampled in the simulation as well. A comparison with our field and the
\citet{Oppermann2012} maps shows that this is due simply to our choice of pointing. Again, the
results of Section \ref{sec:Consistency} show that this does not significantly affect the power
spectrum.

There is another more subtle difference between the Faraday depths of diffuse emission and point
sources. Since we expect diffuse emission mostly to be generated from within our galaxy, we expect
it to be emitting from within the magnetized, ionized plasma that rotates its polarization vector.
As discussed in \citet{Jelic2010}, this creates both a depolarizing effect and structure in the
rotation measure spectrum of the source. 

The simulations account for this by distributing sources on smaller scales than the resolution
element of the array.\footnote{I'm trying very hard not to use the word ``synthesized beam'' here.
  We are looking at the power spectrum on a single baseline --- there is no synthesis, so a
``synthesized beam'' doesn't really make sense.} Since each source is assigned an independent
rotation measure, and many sources are placed within the inverse baseline length ($\theta \sim 1/u$),
then the visibility averages over many rotation measures and polarization angles per pointing. This
has the same effect as a polarized source emitting from within an ionized, magnetized plasma ---
different lines of sight summed within the same resolution element of an array produce a complex
Faraday depth spectrum, and they also add incoherently.

Next, we address the spatial correlation of emission. The simulations assume an isotropic placement
of point sources. Projecting the fringe onto the simulated sky selects the modes correlated on the
baseline scale (in our case, $3^\circ$), so the simulation represents any power correlated on those
$3^\circ$ scales --- that we model it as a series of point sources in many ways is irrelevant, since
it does not affect this fact. Figure \ref{fig:Cell} shows the agreement of these simulations with
existing angular power spectral measurements, which indicates that the level of the power correlated
on $3^\circ$ scales in the simulations agrees with real, diffuse emission. 

Since the simulations produce angular power spectra that agree with measurements, and address
spectral differences between polarized point sources and diffuse emission, we take the simulations
as a good reference for all types of polarized emission. Hence, for the $15\lambda$ baselines we
simulate and measure, any adjustments to the simulation can be considered as adjustments to our
understanding of the polarized sky. 

\subsection{Comparison to Other Measurements}

As mentioned before, most measurements of the polarized sky at meter wavelengths detect large
amounts of diffuse polarized emission, compared to the relatively few point sources.
\citet[][abbreviated in this section as B13]{Bernardi2013}, in a 2400 square degree, detect a single
point source, PMN J0351-2744, whose polarized flux is 320 mJy. Since this was the only source
detected in this survey, they claimed that the polarized fraction of sources must be bounded at
$2\%$.

We ask if our measurements agree with the B13 detection and upper bound. First, we compute the
probability of detecting a source whose polarized fraction is greater than $2\%$ in the context of
our measurements. We scale the log-normal distribution of polarization fractions in \citet{Tucci2012}
by the maximum likelihood value of $x$ from the joint distribution of both bands (Table
\ref{tab:x}). Integrating this distribution above $2\%$  yields the probability of detecting a
polarized point source above $2\%$. That probability is $5\%$. While this implies that $2\%$ cannot
be counted as a strict upper limit, as B13 imply, it does roughly agree with their sstatement that
detecting sources this polarized is unlikely. Thus, we can relax their strict upper limit to a
$2\sigma$ upper bound in the polarization fraction.

\figuremacroW{NgtP.eps}{0.6}{fig:NgtP}{
  Number counts of polarized sources, from a simulation with mean polarized
  fraction of $2.2\times10^{-3}$, derived from the power spectra in Figures \ref{fig:pspec2x2_lo}
  and \ref{fig:pspec2x2_hi}. This is a convolution of the unpolarized source counts
  \cite{Hales1988}, and the polarized fraction distribution \cite{Tucci2012}, scaled by the
  maximum-likelihood value of $x$ (Table \ref{tab:x}).
}{Polarized source counts predicted from our measurement of the mean polarized fraction}
Second, we ask if our updated simulations can produce the occurrence of sources like PMN J0351-2744.
Figure \ref{fig:NgtP} shows the simulated, integrate source counts from the updated simulation.
These source counts imply that one source with a polarized flux of 320 mJy occurs roughly every 1700
square degrees. These number counts are in close agreement with the detection of one source of this
strength in 2400 square degrees.

This data does have the sensitivity to detect sources like PMN J0351-2744, but its location amidst
other polarized emission provides difficulty isolating it, as can be seen in Figure
\ref{fig:pq_vs_t}.

\subsection{Closing Remarks}

Though PAPER in its grid configuration is incapable of creating the high dynamic-range images needed
to isolate polarized point sources, there are several hints in the data indicating the presence of
polarized foregrounds. The power we described in the previous sections is consistent with
the general properties of diffuse, polarized emission described by other measurements
\cite[][e.g.]{Pen2009,Bernardi2009,Bernardi2013,Jelic2014}. Follow-up observations with arrays more
suited for imaging will be necessary to fully detect and characterize this emission.

Even with the much reduced polarized fraction inferred from this emission, the implied level of $Q\to I$ 
leakage exceeds the expected level of the 21cm EoR power spectrum \cite[][e.g.]{Morales2006,
Lidz2008}. This excess presents a challenge for ongoing and future observations. There are two
mitigation strategies. First, polarized point sources may be identified and subtracted, as in
\citet{Geil2011}. Subtracting to the requisite levels will require highly accurate models for an
unreasonable number of sources, as we showed in Section \ref{sec:sim_mitigation}. The second mitigation
strategy will involve the design of future instruments, limiting $Q\to I$ leakage. Engineering a
symmetric primary beam could limit instrumental polarization, and is one of the drivers of the
design of the HERA array.

