\section{Polarization at Meter Wavelengths} \label{sec:FaradayRotation}

Nearly all celestial emission at meter wavelengths comes from the synchrotron radiation of
electrons. Synchrotron emission is natively polarized, so it stands to reason that all emission
at radio frequencies is polarized. These polarized, radio sources yield a wealth of astrophysics in
their own right, but since we are interested in detecting the 21cm EoR power spectrum, we focus on
polarized sources' impact on its detection.

This section will discuss the provenance of polarized sources, give a summary of the current
understanding of the polarized sky at meter wavelengths, and then discuss how they can affect
efforts to measure the 21cm EoR power spectrum.

\subsection{Why is Synchrotron Emission Polarized?}

To discuss the cause of polarized synchrotron emission, we review how synchrotron emission comes
about. This discussion will summarize the appropriate chapters of \citet{Tools} and
\citet{RybickiAndLightman}, but also loosely follows the original paper, \citet{Westfold1959}\footnote{
  In my opinion, an interesting read in how it differs from modern astrophysical calculations}. 

We begin by recalling a few basics of Larmour precession. As an electron takes its helical path
about a magnetic field, it emits a dipole radiation pattern perpendicular to both its acceleration
(pointed toward the center of its helix) and the magnetic field. In the nonrelativistic limit, this
is simply cyclotron emission.  As the speed of the electron approaches $c$, the Larmour dipole
pattern elongates in the direction of the electron's motion, forming a thin beam of emission
preceding the electron. 

An observer off bore-sight of the magnetic field will see a series of pulses in time from a single 
synchrotron electron --- these pulses in the delta-train arrive at the Lorentz-boosted cyclotron frequency, 
\begin{equation}
  \nu_c \propto \gamma^3\omega_B \approx \left[1-\left(\frac{v}{c}\right)^2\right] \frac{eB}{m_ec},
  \label{eq:synchrotron}
\end{equation}
where $\nu_c$ is the observed frequency of the synchrotron emission, $\omega_B \equiv eB/\gamma mc$
is the cyclotron frequency, and $\gamma$ is the Lorentz factor. Hence, each synchrotron electron
emits radiation whose frequency is a function of that electron's energy.

We extend this notion to an ensemble of electrons. Since each contributes emission whose frequency
is proportional to the energy of the electron, the total spectrum will be proportional to the energy
spectrum of the ensemble of electrons.

The same notion can be extended to the polarization properties of synchrotron emission. One can
write the full Larmour formula for the relativistic ensemble of electrons (as in the three sources
cited), and find that the polarized fraction of synchrotron emission, 
$p$\nomenclature[Rp]{$p$}{Polarized fraction}, exactly written in terms of modified Hankel functions 
\cite{Westfold1959}, can be written for power-law energy distributions for the electrons, $N(E)\propto E^{-n}$, 
simply as
\begin{equation}
  p = \frac{n+1}{n+7/3}.
  \label{eq:polfrac_theory}
\end{equation}
The direction of polarization is perpendicular to both the magnetic field and the line of sight.

Finding $n$ in Equation \ref{eq:polfrac_theory} from the spectral slope of the galaxy implies a
polarized fraction of around 75\%. This is a couple orders of magnitude higher that what is
measured, which we will discuss later in this section. 

\subsection{Faraday Rotation}

The next most important process in radio polarimetry is the rotation of the polarization vector
through an ionized, magnetized plasma.\footnote{Faraday rotation got its start in the optical,
  though we mostly see it in the radio since we measure the $E$-field directly in radio. Faraday
rotation is much like interferometry in that sense (remember Michelson and Morley).}

An easy way to understand Faraday rotation is through birefringence. If a polarized wavefront is
incident on a Faraday screen, a thin layer of electrons with a constant magnetic field through them,
then one circular polarization --- the one which opposes cyclotron motion of electrons in that
magnetic field --- is slowed with respect to the other.

Integrating through the Faraday screen, we find that the phase difference between the two
polarizations can be written like a column density: 
\begin{equation}
  \Delta\varphi = \frac{e^3}{(m_ec^2)^2}\lambda^2\int n_e(s)B_{||}(s)\D{s}
  \equiv \lambda^2\Phi,
  \label{eq:def_faraday_depth}
\end{equation}
which defines the rotation measure $\Phi$\nomenclature[G]{$\Phi$}{Rotation measure}. In this
expression, $m_e$ and $e$ are the mass and charge of the electron, $n_e(s)$ is the electron density
along the line of sight, and $B_{||}$ is the component of the magnetic field that aligns with the
line of sight. The integral extends from the observer to the emitting source. Since a Faraday
screen rotates the polarization vector of a field, it affects Stokes parameters $Q$ and $U$ by
rotating them by $2\Phi\lambda^2$: 
\begin{equation}
  (Q + iU)_{meas} = e^{-2i\Phi\lambda^2}(Q+iU)_{int},
  \label{eq:def_faraday}
\end{equation}
where the subscripts $meas$ and $inc$ refer to the measured and incident polarization states,
respectively. 

Measurements of the Faraday depth of sources can be used to characterize galactic magnetic fields
\cite[][e.g.]{Beck2013}. To measure the rotation measure of sources, we can either fit a line in
$\lambda^2$ to the polarization angle, or exploit the Fourier relationship between $\Phi$ and
$\lambda^2$ in Equation \ref{eq:def_faraday} \cite{BrentjensDeBruyn}. A method to characterize the Faraday depths 
rotation the polarization vectors of a source is the topic of discussion in Chapter \ref{chap:DRMT}.
Faraday rotation of sources is also mechanism which gives polarized sources spectral structure,
which has led to the studies in the remainder of this thesis.

\subsection{Methods of Depolarization}

We will show in Section \ref{sec:updates} that the mean polarized fraction of point sources at 150 MHz is less than
$3\times10^{-3}$, a far cry from the 75\% predicted from the spectral slope of our own
galaxy.\footnote{This is the primary result of this thesis.} The question for us to answer now is why
there is such a disparity between these two polarized fractions. The remainder of this section will
discuss methods of depolarization, which we split into two groups: instrumental, and intrinsic.

The first form of instrumental depolarization, called beam depolarization, arises from the minimum
angular resolution element of an array. In general, galactic magnetic fields can be turbulent, so
the polarization angle varies with position. A large synthesized beam sums many of these randomized
polarization vectors, yielding a large amount of depolarization. The extent of this attenuation is
dependent on the size of the synthesized beam, the angular distribution, and alignment of polarization
vectors at a pointing.

\figuremacroW{BandwidthDepolarization.eps}{0.6}{fig:bw_depol}{
  Attenuation factor from Equation \ref{eq:attenuate_bandwidth} versus bandwidth for a band centered
  at 150 MHz, and a distribution of sources drawn from the \citet{Oppermann2012} maps.
}{Depolarization as a function of bandwidth} 
The second method of instrumental polarization comes from integrating in frequency, and is called
bandwidth depolarization. Since all polarized emission passing through magnetic fields undergoes
Faraday rotation, it all gains spectral structure given by Equation \ref{eq:def_faraday}. As we
measure this emission, we must integrate over some bandwidth $\Delta\nu$. The rotation within the
band yields an attenuation given by
\begin{equation}
  P_{meas} = \int W(\nu)P(\nu)\D{\nu},
\end{equation}
where $P(\nu)$ is the unaveraged power, $P_{meas}$ is the measured power within the band, and
$W(\nu)$\nomenclature[Rw]{$W(\nu)$}{Window function in frequency} is the window function of the band. If
we assume the window is a tophat function centered at frequency $\nu_0$, with a width $\Delta\nu$,
and if we also assume that the intrinsic power follows a flat-spectrum polarized source, $P(\nu) =
P_0\exp\{-2i\lambda^2\Phi\}$, then we can expect an attenuation rate of 
\begin{equation}
  \frac{P_{meas}}{P_0} =
  \int \mathbb{P}(\Phi)
  \left[\int_{\nu_0-\Delta\nu/2}^{\nu_0+\Delta\nu/2}e^{-2i\lambda^2\Phi}\D{\nu}\right]
  \D{\Phi},
  \label{eq:attenuate_bandwidth}
\end{equation}
where $\mathbb{P}(\Phi)$ is the probability distribution of Faraday depths. Figure
\ref{fig:bw_depol} shows the level of bandwidth depolarization from Equation
\ref{eq:attenuate_bandwidth} with a distribution of Faraday depths drawn from the
\citet{Oppermann2012} maps. Using the channel widths we use for the results in Chapter
\ref{chap:PowerSpectra}, we expect nearly all polarization to be preserved by this mechanism. Using the
entire bandwidth available to PAPER, around 80 MHz, attenuates the signal to around 10\% of its
intrinsic power. This effect strengthens at lower frequencies, since the $2\Phi\lambda^2$ phase-wrapping
grows faster as the frequency decreases. 

These two methods of depolarization are instrumental, meaning they can be mitigated by the correct
design choices in an instrument. The next two forms of depolarization are intrinsic, meaning they
arise by galactic physics. No data analysis methods or instrument design can remove these sources of
depolarization.

The primary form of intrinsic depolarization is due to turbulent magnetic fields. The expression
linking spectral slope with polarization fraction (Equation \ref{eq:polfrac_theory}) assumes a
constant magnetic field driving the electron's acceleration. This assumption is not true in 
general, but rather, galactic magnetic fields usually vary as a function of position, causing
emission from different parts of the magnetic field to have different polarization angles. This 
effect causes the polarization in a line of sight to have emission from many different photons with different
polarization angles. This amounts to an attenuation of polarized power. In the extreme case, a
totally spatially random magnetic field, the net polarization is zero. 

The secondary form of intrinsic depolarization is due to a polarized emitter's position inside a
magnetized, ionized plasma. This effect is discussed at length in \citet{Jelic2014}, and we will
briefly discuss it here. If an emitter lies within the plasma which Faraday-rotates its emission,
different photons will travel through different lengths of the plasma, which by Equation
\ref{eq:def_faraday_depth} gives each a different Faraday depth. Many photons with different
rotation measures developed this way may scatter into one line of sight, yielding a signal containing many
Faraday-rotated components. Once this line of sight is summed in the instrument, the ensemble
average of the polarized signal is attenuated by a factor proportional to
$\exp\{-4\Delta\Phi\lambda^4\}$, where $\Delta\Phi^2$ is the variance of Faraday depths available in
this line of sight \cite{Tools}.

These methods of depolarization, both instrumental and intrinsic, are discussed in
\citet{Gaensler2001}, and \citet{Landecker2000}, to give two examples. The latter synthesizes
information regarding depolarization into the notion of a polarization horizon. The polarization
horizon is the maximum distance of a polarized source measurable by a given instrument --- polarized
emission produced beyond this horizon will be subjected to a level of instrumental polarization so
great as to attenuate the signal below any reasonable detection threshold. Instruments like PAPER have polarization
horizons on the order of around 10-20 kpc \cite{Bernardi2004}.
