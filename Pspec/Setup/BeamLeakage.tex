\section{Beam Leakage}
\label{sec:BeamLeakage}

Since the 21cm EoR power spectrum is largely unpolarized, we ask the question, why care about
polarized emission? In short, Faraday-rotated, polarized sources will exhibit the spectral structure
reserved solely for the EoR and noise (Section \ref{sec:StatusQuo}). Any $Q\to I$ leakage will
damage prospects of making a clean detection of the 21cm EoR power spectrum. Hence, we begin our
investigation of $Q\to I$ leakage.

The two most prominent ways in which polarized sky emission can leak into an interferometric
estimate of Stokes $I$ are leakages due to non-orthogonal and rotated feeds, and beam ellipticity
--- an asymmetry in the two linear polarizations of a primary beam which causes unpolarized signals
to appear polarized, and vice versa. The first is a well-understood question, discussed at length in
the series of papers by Hammaker, Bregman, and Sault \cite{HBS1,HBS2,HBS3,HBS4,HBS5}. This type of
leakage can be corrected by the proper linear combination of visibilities. Hence, we will omit
discussion of this, and focus entirely on the latter issue. Unlike misaligned feeds, Beam leakage 
cannot be calibrated away. To begin, we will examine the contents of a visibility and relate them 
to the intrinsic Stokes parameters.

We begin by expanding the terms for the $I$ and $Q$ visibilities, from Equation
\ref{eq:def_stokes_vis}, noting that they are not exact representations of the Stokes parameters
they approximate:
\begin{gather}
  \mathcal{V}_{I} = \mathcal{V}_{xx} + \mathcal{V}_{yy} 
    = \int \left(A_{xx} + A_{yy}\right)Ie^{-2\pi i \vec{b}\cdot\hat{s}}\D{\Omega}
    + \int \left(A_{xx} - A_{yy}\right)Qe^{-2\pi i \vec{b}\cdot\hat{s}}\D{\Omega};
  \\
  \mathcal{V}_{Q} = \mathcal{V}_{xx} - \mathcal{V}_{yy} 
    = \int \left(A_{xx} + A_{yy}\right)Qe^{-2\pi i \vec{b}\cdot\hat{s}}\D{\Omega}
    + \int \left(A_{xx} - A_{yy}\right)Ie^{-2\pi i \vec{b}\cdot\hat{s}}\D{\Omega}.
\end{gather}
$\mathcal{V}_I$ is the Fourier transform of $I$, weighted by the sum of the $xx$ and $yy$ beams
(call it $A_+$), and that of $Q$, weighted by the differenced beam ($A_-$). $\mathcal{V}_Q$ is
symmetric to $\mathcal{V}_I$. Na\"{i}ve addition of the Stokes visibilities clearly produces a
mechanism for $Q\to I$ leakage.

If we allowed ourselves the ability to image visiblities, we could simply add a linear combintation
of images to acheive ``pure'' Stokes parameters, but if we restrict ourselves to visibilities, this
process, now a convolution, is impossible without a densely-sampled $uv$-plane. The requirement of
densely-sampled $uv$-plane is often not satisfied, especially for the array used for the duration of 
this thesis certainly falls into this category. More discussion on 
the benefits and consequences of this effect can be found in Section \ref{sec:SparseUV}.

Having established this particular mechanism for $I\to Q$ leakage, we now ask how much of $Q$'s power is
contained in the power of $\mathcal{V}_I$, which yeilds an estimate of the fraction of our power
spectra are corrupted by beam leakage.

To find this estimate, let us first make a three assumptions: 
\begin{enumerate}
  \item The flux contained in $I$ and $Q$ are Gaussian, random fields.
  \item The fields $I$ and $Q$ are uncorrelated.
  \item The varaiance of $Q$ is some fraction $p^2$ that of $I$, and $p \ll 1$.
\end{enumerate}
Despite the obvious exceptions to these three assumptions, we proceed with our estimate of the
leakage power, sacrificing accuracy for an analytic solution, and thus, intuition. The second
assumption allows us to separate the contributions from $I$ and $Q$ when squaring $\mathcal{V}_I$,
allowing us to write 
\begin{equation}
  \left|\mathcal{V}_I\right|^2 = \int |A_+|^2\D{\Omega} \star P_i + \int|A_-|^2\D{\Omega}\star P_Q, 
\end{equation}
where $P_I$($P_Q$) is the power spectrum of $I$($Q$), and $\star$ denotes a convolution. The first
assumption sets $P_I$ and $P_Q$ to be constant, since the power spectrum of a Gaussian, random field is
flat, and the third assumption allows us to write the simple expression for the power spectrum of
$\mathcal{V}_I$,
\begin{equation}
  \left|\mathcal{V}_I\right|^2 \propto \int |A_+|^2\D{\Omega} + p^2\int|A_-|^2\D{\Omega};
\end{equation}
and similarly for $Q$, 
\begin{equation}
  \left|\mathcal{V}_Q\right|^2 \propto \int |A_-|^2 \D{\Omega} + p^2\int |A_+|^2\D{\Omega}.
\end{equation}
We now ask what the ratio $|\mathcal{V}_Q|^2/|\mathcal{V}_Q|^2$ is --- this allows us to answer the
question, what fraction of what I measure in $Q$ is present in what I measure for $I$. Defining $\mathcal{A}_\pm$ to be the power spectrum of $A_\pm$, we can write this ratio as 
\begin{equation}
  \frac{\left|\mathcal{V}_Q\right|^2}{\left|\mathcal{V}_I\right|^2}
    = \frac{\mathcal{A}_+ + p^2\mathcal{A}_-}{p^2\mathcal{A}_+ + \mathcal{A}_-}
    \approx \frac{\mathcal{A}_-}{\mathcal{A}_+},
    \label{eq:beam_metric}
\end{equation}
where the approximation is the first-order Taylor expansion of the ratio in $p$, which has been
measured to be much smaller than one at higher frequencies \cite{Tucci2012}, and been
measured to decrease at lower frequencies \cite[][e.g.]{Bernardi2009}.

\figuremacroW{BeamPlusMinus.eps}{1}{fig:beam_sum_diff}{
  (Right Panel) Summed PAPER beam \cite{Pober2012}, $A_{xx} + A_{yy}$, at 164 MHz, normalized to peak at two.
  (Left Panel) Differenced PAPER beam, $A_{xx} - A_{yy}$, where each component $A_p$ is normalized
  to peak at one. 
}{Sum and Difference of PAPER Beam}
We can apply these parameters to the measured PAPER beam \cite{Pober2012}, whose sum and difference
are shown in Figure \ref{fig:beam_sum_diff}. We find the metric for
leakage at 164 MHz to be
\begin{equation}
  \frac{\mathcal{A}_-}{\mathcal{A}_+} = 2.1\times10^{-3},
\end{equation}
--- this is roughly the fractional level of contamination we'd expect in the $I$ power spectrum. In
fact, this metric can be a function of frequency --- the PAPER beam is most matched at 150 MHz, and
we would expect the leakage to be least at that frequency, and increase as we approach the edges of
the band. The measured values, shown in Figure \ref{fig:beam_leakage}, confirm our intuition,
showing that the minimum leakage occurs at roughly the central frequency of 156 MHz, where the PAPER
antenna's impedence matching is optimized.
\figuremacroW{BeamLeakage.eps}{0.6}{fig:beam_leakage}{
  Fractional beam leakage, defined by Equation \ref{eq:beam_metric}, as a function of frequency for
  the PAPER beam. This metric of leakage takes its minumum value at 156 MHz, near where the PAPER
  antenna is optimized for impedence matching. 
}{Metric of beam leakage vs. frequency}
