\section{Recent Observations}\label{sec:PolStatus}
The polarized sky at meter wavelengths is relatively uncharted
territory, especially in the southern hemisphere.\footnote{\emph{coeli incogniti}} 
Significant efforts have been made to characterize the polarized intensity of individual sources,
the power spectrum of polarized emission, and the structure of
magnetic fields of both our galaxy and extragalactic objects. Here, we will briefly describe the
status of polarimetry at the time of writing this thesis. Many of these measurements were taken at
1.4 GHz, and we must extrapolate their properties down to 150 MHz.

\figuremacroW{WollebenMap.eps}{1.0}{fig:drao}{
  Polarized intensity in galactic coordinates from the DRAO survey \cite{Wolleben2005}. We present
  this to show the relative levels of polarized emission rather than the exact intensity. Lighter
  regions have a higher polarized intensity than darker regions. 
}{Map of polarized intensity from the DRAO survey, from \citet{Wolleben2005}}
There are two nearly complete surveys of the polarized sky: the NRAO VLA Sky 
Survey\nomenclature[Zn]{NVSS}{NRAO VLA Sky Survey} \cite[NVSS,][]{NVSS} and a survey using the Dominion 
Radio Astrophysical Observatory of Canada\nomenclature[Zd]{DRAO}{Dominion Radio Astrophysical
Observatory of Canada} \cite[DRAO,][]{Wolleben2005}, both at 1.4 GHz. The former constructed a catalog of
polarized point sources, alongside their rotation measures. The latter, with its shorter baselines,
was more able to measure the extended polarized emission mostly due to our own galaxy. Both the
Very Large Array (VLA)\nomenclature[Zv]{VLA}{Very Large Array} and DRAO are located in the northern
hemisphere, so much of the southern sky blocked by the Earth. Figure \ref{fig:drao} shows the map of
polarized intensity derived from the DRAO survey\cite{Wolleben2005}. The polarized intensity in this map does 
not match well with unpolarized intensity (see \citet{Haslam} or \citet{dOCMap} for examples). For 
example, the north galactic spur is highly polarized in these maps\footnote{Mostly due to its 
close proximity to us}, but the galactic plane is unpolarized due to turbulent magnetic fields. 

A more recent measurement from \citet{Bernardi2013} surveys 2400 square degrees, a stripe centered
at the PAPER latitude, so within the PAPER field of view. These observations, alongside 
\citet{Wolleben2005} and other measurements \cite[][e.g.]{Jelic2014}, indicate that qualitatively, 
the majority of polarized emission is contained in diffuse structures. \citet{Bernardi2013} does 
measure one point source, PMN J0351-2744, whose flux at 184 MHz is 320 Jy, and whose rotation measure is 
$+33\ {\rm m}^{-2}$. The discovery of this single source will be used to constrain the mean polarized 
flux at these frequencies in the analysis of Section \ref{sec:updates}. 

To better characterize the relative levels of diffuse and point-like polarized emission, groups have
measured its angular power spectrum. The first relevant upper limits were provided by
\citet{Pen2009} using the Giant Metrewave Radio Telescope
(GMRT)\nomenclature[Zg]{GMRT}{Giant
Metrewave Radio Telescope}. They found, for spherical harmonic multipoles $200\le \ell \le 5000$,
and upper limit of around $C_\ell \lesssim 100\ {\rm mK}^2$. More recent work by
\citet{Bernardi2010} detected polarized power at the same level at $\ell \lesssim 1000$ using the
Westerbork Synthesis Radio Telescope (WSRT)\nomenclature[Zw]{WSRT}{Westerbork Synthesis Radio
Telescope}, with no significant detection of power above $\ell$ of 1000. \citet{Bernardi2010} did
not detect emission directly attributable to polarized point sources. 

\figuremacroW{CellCompare.eps}{0.6}{fig:cell_compare}{
  Recent measurements of low-frequency polarized power spectra. The thick black points show the
  \citet{Bernardi2010} measurements of a field around 3C196, and the solid, magenta line shows the
  upper limit of \citet{Pen2009}. The Haslam map at 408 MHz \cite{Haslam}, scaled by a polarization
  fraction of $0.3\%$ is shown by thin, blue points, and a power-law extrapolation is shown with a
  dotted line above $\ell = 200$. This fraction was chosen to agree with the low-$\ell$ points in
  the Bernardi measurement. At high-$\ell$, the upper limits do not constrain the level of polarized
  emission.
}{A comparison of recent polarized power spectrum measurements}
Figure \ref{fig:cell_compare} gives a summary of the low-frequency measurements of polarized power
spectra. In addition to the \citet{Pen2009} and \citet{Bernardi2010} measurements, It also 
shows the angular power spectrum of \citet{Haslam}, scaled by a mean
polarized fraction of $0.3\%$, which roughly causes it to agree with the two measurements. This
scaling requires a large degree of depolarization of the synchrotron emission from an ordered
magnetic field --- the assumption we used in calculating the expression for the expected
polarization fraction for synchrotron radiation earlier in this section (Equation
\ref{eq:polfrac_theory}.

A recent simulation \cite{Jelic2010} attempted to further constrain the problem by fully simulating
a full-Stokes realization of galactic synchrotron emission over a $10^\circ\times10^\circ$ field of
view. They present a realistic spectrum of the mean temperature of polarized emission, but do not
extend their analysis into the power spectrum. They also predict a polarization fraction
from diffuse emission much higher than the limited available measurements allow.

\figuremacroW{Oppermann.eps}{1.0}{fig:oppermann}{
  Map of Faraday depths, from \citet{Oppermann2012}, shown in galactic coordinates.
}{Map of Faraday depths, from \citet{Oppermann2012}}
Finally, we turn to measurements of Faraday depth. The most comprehensive study is a meta-analysis
by \citet{Oppermann2012}, which synthesizes a full-sky map of Faraday depth (Figure
\ref{fig:oppermann}) from existing measurements, mostly from the NVSS polarization survey
\cite{NVSS, Taylor2009}. The quadrupole pattern of rotation measure indicates a large-scale ordering
of the galactic magnetic field \cite{Kronberg2011,Pshirkov2011}, though turbulence of the magnetic
field within the plane diminishes this effect. A recent study \cite{Oppermann2014} finds that the
dominant contribution to the rotation measure of a source comes from magnetic fields within our galaxy.
