\section{Sparse $uv$ Sampling and Wide-field Polarimetry}
\label{sec:SparseUV}

One advantage of the delay spectrum approach (Section \ref{sec:DelaySpectrum}) is that it relaxes
the requirement of gridding in the $uv$-plane. Each baseline is assigned a position in the
$uv$-plane ab initio, and visibilities from similar baselines may be coherently added without
imaging. This allows for sparse sampling in the $uv$-plane without damaging effects from sidelobes
or missing data, problems other methods may experience. Since the delay spectrum rotates a
power-spectrum estimate into the native coordinate system of an interferometer, there are no
inherently missing frequency-data. \citet{PAPERSensitivity} present the sensitivity benefits of a sparse,
redundant array configuration, but other techniques aim to uniformly sample the $(u,v,\nu)$
cube, mitigating the systematic effects of computing a Fourier Transform across unevenly sampled data.

An obvious disadvantage of having sparesely-sampled data is poor imaging. Not only does sparse
sampling provide a highly-irregular synthesized beam, but it also limits the available information
for a full reconstruction of the image. Without adjacent $uv$-samples, a full, accurate
deconvolution by a wide beam simply has inssufficient information. As seen in Section
\ref{sec:BeamLeakage}, the inability to correct for beam effects will provide a significant source
of systematic error via polarized leakage.

By choosing to wield the full power of the delay spectrum approach and redundant sampling, an
observer is forced to add visibilities with no beam weighting. The beam information supplied by
adjacent $uv$-samples simply does not exist, and without transforming into the image plane, is
unrecoverable. Hence, the imperative to investigate the implications of a lack of beam-weighting,
the na\"{i}ve construction of the $I$ visibility, arises.

Together, redundant sampling and the delay spectrum approach give a 21cm EoR experiment incentive to
add raw visibilities, subjecting it to potential leakage. An elliptical primary beam givees a
mechanism whereby polarized emission can corrupt an estimate of the total power. To what degree does
polarized emission corrupt an estimate of the 21cm EoR signal?
