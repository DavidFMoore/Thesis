\chapter{Simulations}
\label{chap:Simulations}

To better grasp the effects of $Q\to I$ leakage into the 21cm signal, we generate several
random realizations of the sky, each consisting of many polarized point sources. Each source passes
through a Faraday screen with some rotation measure, chosen from a distribution based on current
measurements. Next, we simulate that source for a single baseline. Finally, we calculate the power
spectrum measured by that visibility. Only one visibility needs to be simulated, because the
delay-spectrum approach makes use of the fact that each baseline measures the 21cm EoR with a range
of $k$-modes determened by the baseline length, orientation, and bandwidth.

Rather than creating an exact simulation of the physical sky, we create a simulation whose
statistical properties are physically motivated. This choice reflects a desire for simple, easliy
tunable parameters for the simulation. In that same spirit, we model all sources simply as point
sources with a Poisson angular distribution. The simulation's primary concern  with the spectral information
of polarized foregrounds allows us to justify neglecting the angular terms. This is equivalent to
assuming for all the relevant $k$-modes, $k_{||} \gg k_{\perp}$. Emphasis on the $k_{||}$, or
line-of-sight, spectral modes also motivates our decision to model the sky as nuemerous point
sources. For a more detailed discussion of diffuse polarized emission, we direct the reader to
\citet{Jelic2010}.

\section{Parameterizing the Polarized Sky}
\label{sec:sim_params}

\subsubsection*{Source Positions}
Source positions are distributed uniformly over the sphere. A single source's altitude $\theta$ is
drawn from a distribution in which $\cos\theta$ is uniform on $[0,1]$. A source's azimuthal angle
$\phi$ is drawn independently from $\cos\theta$, uniform on $[0,2\pi)$. This choice of source
position distributions conserves number of sources per unit area across the sky, and is equivalent
to drawing both direction cosines, $(l,m) = (\sin\theta\cos\phi, \sin\theta\sin\phi)$, from a
uniform distribution on $[-1,1]$.

\subsubsection*{Flux Counts}
In order to achieve realistic source fluxes and source counts, we base the distributions from which
we draw various parameters on previous radio surveys. For the source fluxes, we aim to agree with
the VLA Large Sky Survey \cite[henceforth, called
VLSS]{VLSS}\nomenclature[Zv]{VLSS}{VLA Large Sky
Survey} and the Sixth Cambridge Survey
\cite[henceforth, called 6C]{Hales1988}\nomenclature[Zs]{6C}{Sixth Cambridge Survey} surveys, summarized alongside a full polarization
survey in Table \ref{tab:surveys}.
\begin{table}
  \begin{center}
    \begin{tabular}{c c c c c}
      {\bf Survey Name} & {\bf Frequency} & {\bf Survey Area} & {\bf Full Stokes?} & Citation\\ 
      \hline\hline
      VLSS &  74 MHz & $\delta > -30^\circ$ &  No & \citet{VLSS} \\
      6C   & 151 MHz & 0.82 sr              &  No & \citet{Hales1988} \\
      NVSS & 1.4 GHz & $\delta > -30^\circ$ & Yes & \citet{NVSS} \\
    \end{tabular}
  \end{center}
  \caption[Summary of three, low-frequency surveys]{\label{tab:surveys} Summary of three,
  low-frequency surveys.}
\end{table}
We can take 6C source counts at face value, since it was measured in the PAPER band at 151 MHz.
However, we must extrapolate the VLSS source counts from the observed 74 MHz into the paper band. We
perform this extrapolation, following \citet{Cohen2004}, by applying a spectral index of -0.79 to the
amplitude of the source counts.

Above some limiting flux $S_{min}$, the differential number counts in ($dN/dS$) found in the 6C survey 
may be characterized by a double power law, turning over at some knee flux $S_0$:
\begin{equation}
  \dfdx{N}{S} = \begin{cases}
    4000\ S_0^{-0.76}S^{-1.75}\ {\rm Jy}^{-1}{\rm sr}^{-1} & S_{min} \le S < S_0 \\
    4000\ S^{-2.81}\ {\rm Jy}^{-1}{\rm sr}^{-1} & S_0 \le S
  \end{cases}.
\end{equation}
Following the 6C results, we choose the turning point $S_0$ to be 0.88 Jy. The number of sources
simulated (20,531) is chosen by the size of the PAPER beam at 151 MHz (0.76 sr) and a flux range
over which to integrate. In general, this operation can be expressed by the integral
\begin{equation}
  N = \Omega \int_{S_{min}}^{S_{max}} \dfdx{N}{S}\D{S},
\end{equation}
where $S_{min}$ and $S_{max}$ are the limiting fluxes of the survey, and $\Omega$ is the survey's field of
view. We choose to include sources in between 100 mJy and 10 Jy. This choice provides a reasonable
dynamic range of sources. Below the lower limit, the 6C sources are unreliable due to
signal-to-noise issues and confusion\footnote{Below a certain flux, I expect to
find more than one source per resolution element --- this uncertainty in the number of sources per
pointing adds to the variance of my measurement.}, and above 10 Jy, we expect that sources may be 
easily identified and removed.

By setting our lower cutoff too high, are we omitting much of the power contained in our
measurement? If we were to blindly extrapolate the 6C source counts below the lower limit of the catalog, we
would add a negligible amount of power. Integrating $S^2\ dN/dS$ down to some minimum flux estimates
the contribution of the sources above that flux to the total variance of the flux. Performing this
operation to the 6C source counts, we find that we are including $\sim70\%$ of the total variance.
Extending the minimum flux would indeed add more power to the simulation, but it would not
drastically alter these results.

The VLSS source counts follow a single power law, given by 
\begin{equation}
  \dfdx{N}{S} = 4865\ S^{-2.3}\ {\rm Jy}^{-1}{\rm sr}^{-1},
\end{equation}
where the spectral index of -0.79 has been applied. For these, we choose minimum and maximum fluxes
of 0.8 Jy and 100 Jy, respectively, rejecting sources well below the lower limit of the catalogue,
and providing a reasonable dynamic range for the included sources. Integrating over the PAPER beam
provides 11,262 sources.

Qualitatively, the source counts for these two surveys differ in two ways. The 6C survey yields
more, dimmer sources, where the VLSS survey yields fewer, brighter sources. By examining the
difference in polarized power from these two source counts, we may answer the question ``Is $Q\to I$
leakage due mostly to a few, bright sources, or is it due to a forest of unresolved, dim sources?''

It is worth noting the robustness of these two source counts with respect to independent
measurements --- both agree with the results of a recent survey from the Murchison Widefield Array
\cite{Williams2012}, an instrument similar in many regards to PAPER.

\subsubsection*{Spectral Indices}
All sources are assigned a spectral index, which is drawn from a normal distribution with mean -0.8,
and standard deviation 0.1. This roughly agrees with the findings of \citet{Helmboldt2008}.

\subsubsection*{Polarized Fractions}
Instead of drawing polarized sources from a measured polarized flux distribution, we simply
down-weight the total intensity by some polarized fraction ($\Pi$), chosen to reflect the studies 
of \citet{Tucci2012}. We sample the $\Pi$ from a log-normal distribution whose mean is $2.01\%$ and
whose standard deviation is $\log(3.845\%)$. Because the log-normal distribution has no upper bound,
and it is unreasonable to find sources with a high polarized fraction\footnote{In fact, it is
impossible to measure $\Pi > 1$!}, we truncate the distribution at $30\%$. As we will investigate in
Section \ref{sec:updates}, this upper limit is considerably higher than what has been measured
at 150 MHz. Following the aforementioned study, we do not impose any correlation between source flux
and polarization fraction.

It has been noted that, among other effects, bandwidth depolarization causes the polarized fraction
to decrease at lower frequencies \cite{Law2011}. This, alongside the GMRT measurements
\cite{Pen2009}, indicates that these distributions, taken at 1.4 GHz, may overestimate the
distribution at 150 MHz. We neglect this instinct that we are overestimating the polarized flux,
taking the 1.4 GHz measurements at face value, since the mean polarization fraction can be thought
of as a scale factor to the overall power spectrum.

\subsubsection*{Polarization Angles}
The polarization angle of each source is chosen to be uniformly sampled on $[0,\pi)$, which assumes no correlation in the polarization angles of individual extragalactic sources. Section \ref{sec:Consistency} 
will investigate the validity of this claim.

\subsubsection*{Rotation Measures}

We draw our distribution of rotation measures on the map presented in \citet{Oppermann2012}. To
mimic the effects of depolarization due to a finite spacial resolution \cite{Law2011}, we apply a
low-pass filter to the rotation measure map. Projecting the map into a spherical harmonic basis, we
keep only those modes below the resolution of our simulated instrument. In the case of this
simulation, we choose to keep only $\ell \le 100 = 2\pi |\vec{b}/\lambda|$. This averages the
polarization vectors in much the same way as a synthesized beam, and its effect is to essentially
remove outliers in the rotation measure distribution, to which instruments like PAPER may not be
sensitive. We then randomly draw rotation measures from the empirical cumulative distribution
function of rotation measures, computed from the filtered, \citet{Oppermann2012} maps. Aside from low-pass
filtering, no spatial information from the data is used. Section \ref{sec:Consistency} briefly
discusses the negligible consequences of spatially correlating rotation measures.

\figuremacroW{SimParams.eps}{1}{fig:sim_params}{
  Distributions of simulated parameters. (Top Left) Euclidean normalized source counts for the sources produced 
  in simulations $A$ (blue) and $B$ (black). Over plotted in cyan and gray, respectively, are the
  analytical distributions from which they are drawn. (Top Right) Distribution of polarized
  fractions used, with the log-normal distribution over-plotted in gray. (Bottom) Empirical
  distribution of rotation measures, generated from a spatially low-pass-filtered map
  \cite{Oppermann2012}. 
}{Distributions of simulation parameters
$S$, $\Pi$, and $\Phi$.}
Histograms of the empirical distributions of rotation measure, polarized fraction, and source counts may be
found in Figure \ref{fig:sim_params}. Over-plotted on all are the distributions from which they are
drawn. Figure \ref{fig:sim_polflux} shows the empirical distribution of polarized flux, using the
NVSS and 6C surveys --- these distributions should be convolutions of the power law source counts
and the log-normal polarized fraction. These distributions qualitatively agree with the total power 
source counts: the 6C survey produces dimmer sources, while NVSS produces fewer.
\figuremacroW{SimPolFlux.eps}{0.6}{fig:sim_polflux}{
  Euclidean-normalized, differential source counts for polarized flux in simulations $A$ (blue)
  and $B$ (black).
}{Distribution of polarized flux in two
simulations}

We calculate visibilities for a single 30m, east-west baseline, corresponding to the most common spacing in
the maximum-redundancy PAPER array \cite{PAPERSensitivity, Parsons2014}. The choice of baseline
orientation is arbitrary, and since we are only modelling point sources, the choice of baseline
length will only set the horizon limit of the power spectrum. Since the delay affected by a rotation
measure is independent of a choice of baseline (Equation \ref{eq:bad_k}), choosing a relatively
short baseline length will isolate smooth-spectrum foregrounds at lower $\tau$ and highlight Faraday leakage.

The full measurement equation for the visibility with linear polarization $p$ ($\mathcal{V}_p$) used 
in this simulation is
\begin{align}
  \mathcal{V}_p = \sum_{j=1}^{N_{src}}
    A_p(l_j, m_j, \nu)
    S_j^{150}
    \left(\frac{150\ {\rm MHz}}{\nu}\right)^{\alpha_j}
    e^{ -2\pi i\nu(ul_j + vm_j)}
    \left(1 \pm \Pi_j e^{-2i(\Phi_j\lambda^2 + \chi_j)} \right),
    \label{eq:sim_vis}
\end{align}
where each source $j$ is assigned a flux ($S_j$), a polarized fraction ($\Pi_j$), a spectral index
($\alpha_j$), a position $(l_j,m_j)$, rotation measure ($\Phi_j$), polarization angle ($\chi_j$),
and is weighted by the model primary beam in that linear polarization ($A_p$). To include both $I$
and $Q$ emission, $xx$ visibilities receive the $+$, while $yy$ visibilities recieve the $-$. They
are then summed and differenced to yield $\mathcal{V}_I$ and $\mathcal{V}_Q$. A sample $Q$ visibility 
is shown in Figure \ref{fig:sample_Q_sim}.
\figuremacroW{SampleQSim}{0.6}{fig:sample_Q_sim}{
  The real part of a sample $Q$ visibility, given by Equation \ref{eq:sim_vis}, and generated using
  the parameters shown in Figure \ref{fig:sim_params}. 
}{Sample simulated $Q$ visibility.}

We choose not to include the parallactic rotation of $Q$ and $U$ (see Section
\ref{sec:Polarimetry}), implying that the $Q$ we label for this simulation is fixed to topocentric,
azimuth and altitude coordinates. This choice clarifies equations and allows for an ease of
understanding which would be obfuscated by writing equatorially defined $Q$ and $U$.

\begin{table}
  \begin{center}
    \begin{tabular}{c c c c}
      {\bf Label} & {\bf Source Counts} & {\bf $N_{src}$} & {\bf Rotation Measure Distribution}\\ 
      \hline\hline
      A & 6C   & 20,531 & Oppermann \\
      B & NVSS & 11,262 & Oppermann \\
      C & 6C   & 20,531 & 2$\times$Oppermann \\
    \end{tabular}
  \end{center}
  \caption[Simulation treatments]{\label{tab:sims} Three simulation treatments used in this section.}
\end{table}

Finally, we take advantage of our built-in tunable parameters and present three treatments of the
simulation, summarized in Table \ref{tab:sims}. Simulation $A$ serves as a baseline measurement,
with reliable $6C$ source counts and the conservative, low-pass filtered Oppermann rotation measure
distribution. Simulation $B$ uses NVSS source counts, asking is $Q\to I$ leakage is dominated by a
few, bright sources, rather than the forest of dim sources in 6C. Simulation $C$ uses the 6C source
counts, but doubles the Oppermann rotation measures, asking how large rotation measures affect $Q\to
I$ leakage.

\section{Results}
\label{sec:sim_results}

\figuremacroW{SimulationResults}{0.9}{fig:sim_power_spectra}{
  Power spectra for the three treatments of the simulation discussed in Section
  \ref{sec:sim_params}. From top to bottom, the rows are treatments A, B, and C. The left column
  shows the $I$ power spectrum, highlighting $Q\to I$ leakage. The right column shows the $Q$ power
  spectrum. Three redshift bins are shown in all plots: $z=11.13$ (cyan), $z=9.77$ (black), and
  $z=7.05$ (blue). Error bars show 95\% confidence intervals of several iterations of the
  simulation. For a point of reference in the left column, a fiducial EoR model \cite{Lidz2008} is plotted in grey.   
}{Power spectra for three treatments of the simulation}

Figure \ref{fig:sim_power_spectra} the power spectra of several renderings of simulations 
$A$, $B$, and $C$. We interpret the power spectrum of $\mathcal{V}_I$ outside the horizon 
as the amount of polarized leakage corrupting the EoR signal ($Q\to I$ leakage), and the 
power spectrum of $\mathcal{V}_Q$ is our best representation of the polarized signal. These 
plots show the median power in each $k$ bin for 100 realizations of the simulation, with error bars
showing the 1-$\sigma$ extend of the bandpowers for these realizations. These power spectra confirm
the prediction made in Section \ref{sec:onesource} that $\lambda^2$ phase wrapping extends the
foreground cutoff \cite[][e.g.]{PAPERSensitivity, Pober2013} to higher delay bins, corrupting some of the
most sensitive regions of $k$ space for 21cm EoR analysis. They also demonstrate the prediction that
high-redshift bins will be most affected.

\figuremacroW{DeltakGoodK}{0.6}{fig:k3pk_goodk}{
  $A$ (cyan), $B$ (blue), $C$ (black)
}{$\Delta^2(k)$ at $k\approx 0.2\ h{\rm Mpc}^{-1}$ vs. redshift for three simuations}
The severity of the leakage can be inferred from the power in the most EoR-sensitive $k$-bins which lie
outside the horizon for small baselines ($0.2\ h{\rm Mpc}^{-1} \le k \le 0.3\ h{\rm Mpc}^{-1}$). Figure 
\ref{fig:k3pk_goodk} shows $\Delta^2(k)$ in these bins as a function of redshift. The leaked power 
ranges in the hundreds of $\rm{mK}^2$ to thousands, increasing from high frequency/low
redshift to low frequency/high redshift. These simulations are about an order of magnitude above the
level of the expected 21cm signal \cite{Lidz2008}. If we may take this simulation as an accurate
prediction of the low-frequency sky's polarized emission, these results imply that na\"{i}vely
adding $\mathcal{V}_{xx}$ and $\mathcal{V}_{yy}$, formed with an approximately 10\% asymmetric
primary beam, incorporates enough bias from polarized leakage to completely obscure the 21cm signal.
The levels of leakage in our simulations demand a strategy to model and remove point sources.

We note that simply forming $\mathcal{V}_I$ will also remove a negligible component of the EoR
signal via the same mechanism. In a sense, the $Q\to I$ leakage can be thought of as a rotation of
power between the two Stokes parameters. Hence, for precision measurements of the EoR signal, this
simple estimate may not be ideal. However, the effects of the $I\to Q$ leakage is small (compare the
levels of $\mathcal{V}_Q$ and high $k$-modes of $\mathcal{V}_I$) and should not provide a
significant hiderance to detection.

Were a power spectrum computed from only one linearly polarized visibiltity ($xx$, for instance),
all polarized power would corrupt the measurement. We have chosen to suppress the polarized leakage
by adding the linearly-polarized visibilities $xx$ and $yy$. The leakage is dependent on the
difference of the two beams (see discussion in Section \ref{sec:BeamLeakage}), and by having beams
that are at most 10\% different suppresses the signal by around two to three orders of magnitude.
Correcting for the beam-weighting in the image domain can further suppress the leakage, but errors
in the beam model will introduce leakage in much the same manner. Hence, the constraint of having to
suppress polarized leakage by four orders of magnitude causes the need for an accurate primary beam
model to around the 1\% level in the case of imaging, or symmetric at the 1\% level if visibilities
are used directly.

We conlude this discussion by noting the large variance in simulted power. The results shown are the
mean bandpowers  in $\Delta^2(k)$ for several realizations of the simulation. Taking so many
realizations into account essentially maps out the posterior distribution of the $\Delta^2(k)$
bandpowers. The $2\sigma$ width covers nearly two orders of magnitude, which indicates that the
level of $Q\to I$ leakage is highly sensitive to the exact parameters drawn in any realization.
The actual level of leakage measured will thus be highly dependent on a choice of field, and on
the actual distributions of polarized fluxes and rotation measures.

\section{Consistency Tests}
\label{sec:Consistency}

Now that we have presented the power spectra, we must check two
things: first, that the two-dimensional $C_\ell$ power spectrum generated by the simulations agrees
with current measurements \cite{Bernardi2009}, and second, that the assumption that uncorrelated
polarization angles is valid.

\subsection{Two-Dimensional Power Spectrum and Diffuse Emission}

\figuremacroW{SimCell.eps}{0.6}{fig:Cell}{
  Black points show the $C_\ell$ power spectrum from \citet{Bernardi2009}. The blue line shows the
  mean $C_\ell$ of several simulations. The shaded, cyan region shows $2\sigma$ limits of the
  distributions for each bin in $\ell$. This shows the consistency between our simulations and
  recent measurements. Treatment A is used for these simulations.
}{Comparison of simulated $C_\ell$ power spectrum with recent measurements}
Figure \ref{fig:Cell} shows the distribution of two-dimensional power spectrum over several
simulations, plotted alongside the $C_\ell$ measurements from \citet{Bernardi2009}. We see
qualitatively that our simulation well obeys the upper limits imposed by the Bernardi measurement.
This agreement helps validate our results.

The estimates of power in Section \ref{sec:sim_results} are dependent on the relative strengths of
diffuse, polarized emission and polarized point sources. We have taken care to agree with
measurements of all polarized emission, but those measurements are uncertain above $\ell\sim300$. We
interpret them as an upper limit. In the limiting case where diffuse emission is the only component
to the polarized sky, this leakage could be suppressed by measuring with a longer baseline, which in
turn measures a lower $\ell$ or $k_\perp$. We have chosen a 30m baseline, which corresponds to
$\ell\approx200$. This choice of baseline length is relatively short for interferometers at these
wavelengths, but falls at the high end of the \citet{Bernardi2009} measurements.

Including additional diffuse emission in the simulation would certainly increase the total power in the
simulation for low $\ell$, but the frequency structure would remain qualitatively the same as point
sources. As we will show in the following section, the correlation of rotation measures and
polarization angles that could be introduced by an extended structure will not significantly affect
the power spectrum. For this reason, we can consider the polarized sky as having two components with
nearly identical footprints in the line-of-sight direction: diffuse and point-like. Both components
will exhibit similar frequency structure, so choice of baseline length will set the relative
weightings of these components. \citet{Bernardi2009} briefly discuss some of the implications of
their measurement of extended structure to the three-dimensional power spectrum in their conclusion,
which agrees with our analysis of point-like structure. We will discuss the qualitative differences
between diffuse and point-like emission in Section \ref{sec:applicable}.

\subsection{Correlating Polarization Vectors}

The analysis of Section \ref{sec:sim_results} neglects known spatial correlations of the rotation
measure distribution \cite{Kronberg2011}. Furthermore, the random drawing of polarization angles
could have a cancelling effect on the visibilities. This neglect could potentially suppress our
estimation of polarized leakage into the power spectrum. 

To investigate these possible effects, we choose rotation measures from the Oppermann map
\cite{Oppermann2012}, with a pointing center at the Galactic south pole --- a reasonable field for
EoR analysis. We then set all polarization angles to zero, maximally correlating polarization
vectors, while still including information of the polarized sky. All other simulation parameters are
identical to treatment A of Table \ref{tab:sims}.

\figuremacroW{CorrelatePspec.eps}{0.6}{fig:corr_sim}{
  A comparison of power spectrum measurements for a treatment of the simulation with correlated
  polarization angles (black), and those from treatment A (gray). As in Figure
  \ref{fig:sim_power_spectra}, the left panel shows the $I$ power spectrum, and the right panel shows the
  $Q$ power spectrum. Three redshift bins are shown, each denoted with a different line style: 9.73
  (solid), 8.33 (dashed), and 7.25 (dot-dashed). The results of simulation A with this simulation
  show that correlating polarization vectors does not affect the power spectrum.
}{Simulation with correlated polarization vectors}
Figure \ref{fig:corr_sim} compares the results of this treatment  with simulation A from Table
\ref{tab:sims}. The power spectrum of this treatment agreees with simulation A at all redshifts and
values of $k$, for both polarizations. This agreement indicates that the spatial correlation of
polarization vectors do not significantly affect the power spectrum. Thus, the assumption in Section
\ref{sec:sim_params} are spatially uncorrelated does not affect the results of these simulations.

\section{Mitigating Leakage}
\label{sec:sim_mitigation}

Section \ref{sec:sim_results} predicts an excess polarized signal due only to point sources of
around $10^4\ {\rm mK}^2$ at $k\sim 0.15\ h{\rm Mpc}^{-1}$ for most treatments of the simulation.
While the exact levels of these predictions may be subject to some error, the need certainly arises
for some removal scheme. This removal must suppress power from polarized foregrounds by around four
orders of magnitude in the power spectrum.

\figuremacroW{DeltakGoodK_rm.eps}{0.6}{fig:rm_srcs}{
}{Figure \ref{fig:k3pk_goodk}, with sources perfectly removed}
To investigate the effects of modelling and removing polarized sources, we rerun the simulation,
excluding the brightest polarized sources. Figure \ref{fig:rm_srcs} shows the median value of
several simulations of the $k$-bin nearest $0.25\ h{\rm Mpc}^{-1}$, having removed the brightest
1000, 2000, 5000, and 10,000 sources. These limits in numbers of sources correspond to unpolarized 
flux-limits of 1300, 900, 460, and 240 mJy, respectively. Polarized flux limits are roughly $2\%$ of
these. We remove these sources from treatment A of
the simulation, which includes around 21,000 sources. Despite having removed nearly one-third of
the sources, the leaked power still exceeds $10\ {\rm mK}^2$, the expected level of the 21cm EoR
power spectrum.

To remove enough flux to consistently fall below the expected EoR signal, we need to remove a large 
majority of polarized point sources. We recomputed the simulations with a lower minimum flux (60
mJy), expecting a similar result, but found that we increased the power in this $k$-bin by only one
or two ${\rm mK}^2$. For total power to fall below $10\ {\rm mK}^2$, more sources required removal.
We exclude further investigation of this analysis for three reasons. First, current measurements do
not constrain $dN/dS$ to the levels necessary to accurately model such low-flux sources. Second,
including lower-flux sources does not significantly affect the result that the result that the
expected polarized power spectrum will be of the order of $10^4$-$10^6\ {\rm mK}^2$. Third, the
variance in power from one simulation to the next was large enough that the two treatments of the
simulation --- even with 10,000 sources removed --- could not be considered significantly different.

The onerous levels of source-removal suggest that a different mitigation scheme be considered.
Future instruments may take polarization into consideration in their design. Leakage can be
mitigated with more circular beams, and circular feeds avoid the $Q\to I$ leakage entirely. Even
with existing data, rotation measure synthesis \cite{BrentjensDeBruyn} could potentially provide the
ability to separate sources with distinct rotation measure structure to be separated from the EoR
signal.

