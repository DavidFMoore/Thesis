\section{Updated Polarization Fractions}\label{sec:updates}

\subsection{Scaling the Simulations}

\figuremacroW{CompareSim.eps}{0.6}{fig:compare_sim}{
  Top row: measured and simulated power spectra for $I$ (left) and Q (right). Measured power
  spectra are in black, simulated values are in blue (median value), cyan ($68\%$ confidence
  interval), and light cyan ($95\%$ confidence interval). Simulations are generated as in
  Chapter \ref{chap:Simulations}, with a mean polarized fraction of $2.01\%$. Bottom row: Same as
  the top row, for Band II.
}{A comparison of simulations and measurements.}
Figure \ref{fig:compare_sim} compares the measured $Q$ and $I$ power spectra to those simulated
in Chapter \ref{chap:Simulations}. Since the measured values consistently disagree with the
simulations, we can constrain the input parameters to the simulations, beginning with a simple
scaling relation:
\begin{equation}
  P_k = x^2S_k,
  \label{eq:def_x}
\end{equation}
where $P_k$\nomenclature[Rp]{$P_k$}{Power spectrum in the $k^{{\rm th}}$ bin} is the measured $Q$ 
power spectrum in the $k^{{\rm th}}$ bin,
$S_k$\nomenclature[Rs]{$S_k$}{Simulated power spectrum in
the $k^{{\rm th}}$ bin} is the simulated power spectrum in that bin, and
$x^2$\nomenclature[Rx]{$x$}{Scale factor between the simulated and measured power spectra} is the 
scale factor between the two. We choose to use a scale factor of $x^2$ rather than $x$ in order to
facilitate the interpretation of $x$ as an adjustment to the mean polarized fraction of point
sources, as we will soon discuss.

For the duration of this section, we will approximate the measured power spectrum $\hat{P}_k$ as
normal, random variables, 
\begin{equation}
  \hat{P}_k \sim \mathcal{N}(P_k, \sigma_k^2),
\end{equation}
with mean $P_k$ and variance $\sigma_k^2$, derived from the distribution of bootstrapped power
spectra.

We find the scale factor $x$ that best fits Equation \ref{eq:def_x}, and then interpret its physical
meaning. The likelihood of drawing a simulated power spectrum
$\mathcal{S}$\nomenclature[Rs]{$\mathcal{S}$}{Set of all bandpowers in a simulated power spectrum} by
a factor $x^2$ given the measured data
$\mathcal{D}$\nomenclature[Rd]{$\mathcal{D}$}{Set of all
bandpowers in a measured power spectrum} is 
\begin{equation}
  P(x,\mathcal{S}|\mathcal{D}) 
  \propto \exp\left\{-\frac{1}{2}\sum_k\frac{|x^2S_k - P_k|^2}{\sigma_k^2}\right\},
\end{equation}
where the sum extends over all available values of $k$.

By marginalizing over $\mathcal{S}$, we find the likelihood of $x$:
\begin{equation}
  P(x|\mathcal{D}) \propto \int P(x,\mathcal{S}|\mathcal{D})P(\mathcal{S})\D{\mathcal{S}}.
  \label{eq:likelihood}
\end{equation}
Here, $P(\mathcal{S})$ is the joint probability of all $k$-bins of the simulation, i.e. $P(S_0,
\dots, S_n)$ for $k$-bins labelled 0 to $n$, and $\D{\mathcal{S}}$ denotes the $n$ values of
$\mathcal{S}$ over which we integrate. We compute the integral in Equation \ref{eq:likelihood} 
by the Monte Carlo technique, sampling $\mathcal{S}$ from different instances of the simulation. 
This encapsulates both the probability distribution functions of each $S_k$ and the covariances 
between $k$-bins in $\mathcal{S}$. To insulate the result from potentially damaging effects of 
the foreground removal (Section \ref{sec:rm_fg}), we do not consider $k$-bins within the horizon 
in this integral. 

To find the most likely value of $x$ which would produce the measured power spectrum, we turn to
Bayes theorem, $P(\mathcal{D}|x) \propto P(x)P(x|\mathcal{D})$, where $P(\mathcal{D}|x)$ is the
posterior distribution of $\mathcal{D}$ and $P(x)$ is our prior on $x$. Since $x^2$ is a scale
factor, we choose to use Jeffrey's prior in $x^2$, which sets $P(x) \propto 1/x$.

\figuremacroW{Px.eps}{0.6}{fig:Px}{
  Posterior distributions for the data, given scale factor $x$. Blue shows that from Band I; cyan,
  from Band II; and black shows the joint posterior from both bands. Moments of these distributions
  are summarized in Table \ref{tab:x}.
}{Posterior distribution of $\mathcal{D}$ given $x$.}
\begin{table}\begin{center}
  \begin{tabular}{ c c c c }
    Band & $\bar{x}$ & $\sigma_x$ & Implied Mean Polarized Fraction 
    \\
    \hline\hline
    I & 0.99 & 0.002 & $2.0\times10^{-3}$
    \\
    II & 0.247 & 0.002 & $5.0\times10^{-3}$
    \\
    Both & 0.108 & 0.001 & $2.2\times10^{-3}$
  \end{tabular}
  \caption{\label{tab:x}Moments of $P(\mathcal{D}|x)$.}
\end{center}\end{table}
Measurements from the different bands can be summarized into a joint posterior by simply computing
the product of the posterior of each band. This assumes that each band is independent, a reasonable
assumption given the high level of noise in the measured power spectra. Figure \ref{fig:Px} shows
the posterior distributions of $\mathcal{D}$ given $x$ for Bands I and II, alongside the joint posterior. 
The moments of the three distributions are summarized in Table \ref{tab:x}.

\subsection{Why is $x$ Related to the Polarized Fraction?}

As mentioned in the previous section, we interpret the scale factor $x$ as an adjustment to the mean
polarized fraction of point sources. The simulations parameterize each point source with a polarized
fraction $p$, an unpolarized flux $f$, and a rotation measure $\Phi$. All sources are given a
spectrum $pf\exp\{-2i\Phi\lambda^2\}$, where $\lambda^2$ is the squared wavelength. This is a
simplified account of Equation \ref{eq:sim_vis}, but encapsulates the relevant quantities for this
discussion. Since the source counts are well measured at these frequencies \cite{Hales1988}, and
rotation measures are also well-measured \cite{Oppermann2012} and independent of frequency, we
regard the distributions of these two quantities to be fixed. Hence, any constraints we place on
these simulations can be considered as updates to the distribution of polarized fraction.

The amplitude of the power spectrum in the simulations ($S_k$) can be expressed in terms of the
source fluxes $f_i$, and the polarized fractions $p_i$.,
\begin{equation}
  S_k \sim \sum_{i,j}p_ip_jf_if_j \equiv \bar{p}^2\sum_{i,j}\pi_i\pi_jf_if_j,
\end{equation}
where we have defined $\pi_i \equiv p_i/\bar{p}$ as the ratio of a single polarization fraction
$p_i$ to the mean, $\bar{p}$. Hence, the simulated power spectra are proportional to the mean
polarized fraction squared, $\bar{p}^2$, and we can interpret the scale factor $x$ as the fractional
change in the mean polarized fraction.

Table \ref{tab:x} gives the implied mean polarized fraction of point sources for the two bands and
the joint posterior. In these simulations, we drew polarized fractions from a distribution with a
mean of around $2\%$, and now we can set a limit about an order of magnitude lower. 

\figuremacroW{CompareSim_MLE.eps}{0.6}{fig:compare_sim_mle}{
  Same as Figure \ref{fig:compare_sim}, but using a polarized fraction distribution scaled by
  $x=0.108$, the maximum likelihood value of $x$ using both bands.
}{Figure \ref{fig:compare_sim}, with a mean polarized fraction of $2.2\times10^{-3}$.}
Figure \ref{fig:compare_sim} shows updated simulations using the implied polarization fraction of
the joint posterior from Figure \ref{fig:Px}: $2.2\times10^{-3}$. The simulated $Q\to I$ leakage now
lies between 10 and 100 ${\rm mK}^2$ in Band II, around the expected level of the 21cm EoR power
spectrum at redshift 7.

\subsection{On the Applicability of the Simulations}\label{sec:applicable}

We now turn our attention to a qualitative discussion of the diffuse emission found in
\citet[][B13]{Bernardi2013} and \citet[][J14]{Jelic2014}, and how applicable the simulations are to
the polarized emission found in those measurements, which is largely characterized as being diffuse
and as having low rotation measures. We will argue that the simulations may apply to both these two
measurements, and also that the simulations are a valid point of comparison to the measurements made
in Section \ref{sec:results}. This discussion builds on Chapter \ref{chap:Simulations}, and justifies
the use of the simulation as a tool for understanding the power spectra presented in Section
\ref{sec:results}.

Both B13 and J14 show diffuse, weakly polarized emission found at relatively low rotation measure
($|\Phi| \lesssim 25\ {\rm m}^{-2}$. These measurements differ from the input sources of the
simulations in two ways: in the choice of rotation measures included, and in the spatial
correlation of power. We will discuss these in turn.

The simulations sample rotation measures from the entire \citet{Oppermann2012} map, rather than
restricting to a particular field of view. Two effects may arise from such a generality. First, the
inclusion of large rotation measures could scatter power to larger $k$ in the simulations than in
reality, and second, uncorrelated polarization vectors in the simulation could have a depolarizing
effect. This concern was addressed in Section \ref{sec:Consistency} by computing the simulation with
all rotation measures drawn from a pointing at the galactic south pole (coincidentally, the B13
field), with maximally correlated polarization vectors. The result of this test was identical to the
random drawing (Figure \ref{fig:corr_sim}). The reproduction of power spectra between the two
simulations indicates that rotation measure is not a dominant factor in determining the shape or the
amplitude of the power spectrum of polarized emission. 

The rotation measures in B13 and J14 are considerably lower than those sampled in the simulations.
Since there is overlap between our field and the B13 field, our measurements include lower
rotation measures than sampled in the simulation as well. A comparison with our field and the
\citet{Oppermann2012} maps shows that this is due simply to our choice of pointing. Again, the
results of Section \ref{sec:Consistency} show that this does not significantly affect the power
spectrum.

There is another more subtle difference between the Faraday depths of diffuse emission and point
sources. Since we expect diffuse emission mostly to be generated from within our galaxy, we expect
it to be emitting from within the magnetized, ionized plasma that rotates its polarization vector.
As discussed in \citet{Jelic2010}, this creates both a depolarizing effect and structure in the
rotation measure spectrum of the source. 

The simulations account for this by distributing sources on smaller scales than the resolution
element of the array.\footnote{I'm trying very hard not to use the word ``synthesized beam'' here.
  We are looking at the power spectrum on a single baseline --- there is no synthesis, so a
``synthesized beam'' doesn't really make sense.} Since each source is assigned an independent
rotation measure, and many sources are placed within the inverse baseline length ($\theta \sim 1/u$),
then the visibility averages over many rotation measures and polarization angles per pointing. This
has the same effect as a polarized source emitting from within an ionized, magnetized plasma ---
different lines of sight summed within the same resolution element of an array produce a complex
Faraday depth spectrum, and they also add incoherently.

Next, we address the spatial correlation of emission. The simulations assume an isotropic placement
of point sources. Projecting the fringe onto the simulated sky selects the modes correlated on the
baseline scale (in our case, $3^\circ$), so the simulation represents any power correlated on those
$3^\circ$ scales --- that we model it as a series of point sources in many ways is irrelevant, since
it does not affect this fact. Figure \ref{fig:Cell} shows the agreement of these simulations with
existing angular power spectral measurements, which indicates that the level of the power correlated
on $3^\circ$ scales in the simulations agrees with real, diffuse emission. 

Since the simulations produce angular power spectra that agree with measurements, and address
spectral differences between polarized point sources and diffuse emission, we take the simulations
as a good reference for all types of polarized emission. Hence, for the $15\lambda$ baselines we
simulate and measure, any adjustments to the simulation can be considered as adjustments to our
understanding of the polarized sky. 

\subsection{Comparison to Other Measurements}

As mentioned before, most measurements of the polarized sky at meter wavelengths detect large
amounts of diffuse polarized emission, compared to the relatively few point sources.
\citet[][abbreviated in this section as B13]{Bernardi2013}, in a 2400 square degree, detect a single
point source, PMN J0351-2744, whose polarized flux is 320 mJy. Since this was the only source
detected in this survey, they claimed that the polarized fraction of sources must be bounded at
$2\%$.

We ask if our measurements agree with the B13 detection and upper bound. First, we compute the
probability of detecting a source whose polarized fraction is greater than $2\%$ in the context of
our measurements. We scale the log-normal distribution of polarization fractions in \citet{Tucci2012}
by the maximum likelihood value of $x$ from the joint distribution of both bands (Table
\ref{tab:x}). Integrating this distribution above $2\%$  yields the probability of detecting a
polarized point source above $2\%$. That probability is $5\%$. While this implies that $2\%$ cannot
be counted as a strict upper limit, as B13 imply, it does roughly agree with their sstatement that
detecting sources this polarized is unlikely. Thus, we can relax their strict upper limit to a
$2\sigma$ upper bound in the polarization fraction.

\figuremacroW{NgtP.eps}{0.6}{fig:NgtP}{
  Number counts of polarized sources, from a simulation with mean polarized
  fraction of $2.2\times10^{-3}$, derived from the power spectra in Figures \ref{fig:pspec2x2_lo}
  and \ref{fig:pspec2x2_hi}. This is a convolution of the unpolarized source counts
  \cite{Hales1988}, and the polarized fraction distribution \cite{Tucci2012}, scaled by the
  maximum-likelihood value of $x$ (Table \ref{tab:x}).
}{Polarized source counts predicted from our measurement of the mean polarized fraction}
Second, we ask if our updated simulations can produce the occurrence of sources like PMN J0351-2744.
Figure \ref{fig:NgtP} shows the simulated, integrate source counts from the updated simulation.
These source counts imply that one source with a polarized flux of 320 mJy occurs roughly every 1700
square degrees. These number counts are in close agreement with the detection of one source of this
strength in 2400 square degrees.

This data does have the sensitivity to detect sources like PMN J0351-2744, but its location amidst
other polarized emission provides difficulty isolating it, as can be seen in Figure
\ref{fig:pq_vs_t}.

\subsection{Closing Remarks}

Though PAPER in its grid configuration is incapable of creating the high dynamic-range images needed
to isolate polarized point sources, there are several hints in the data indicating the presence of
polarized foregrounds. The power we described in the previous sections is consistent with
the general properties of diffuse, polarized emission described by other measurements
\cite[][e.g.]{Pen2009,Bernardi2009,Bernardi2013,Jelic2014}. Follow-up observations with arrays more
suited for imaging will be necessary to fully detect and characterize this emission.

Even with the much reduced polarized fraction inferred from this emission, the implied level of $Q\to I$ 
leakage exceeds the expected level of the 21cm EoR power spectrum \cite[][e.g.]{Morales2006,
Lidz2008}. This excess presents a challenge for ongoing and future observations. There are two
mitigation strategies. First, polarized point sources may be identified and subtracted, as in
\citet{Geil2011}. Subtracting to the requisite levels will require highly accurate models for an
unreasonable number of sources, as we showed in Section \ref{sec:sim_mitigation}. The second mitigation
strategy will involve the design of future instruments, limiting $Q\to I$ leakage. Engineering a
symmetric primary beam could limit instrumental polarization, and is one of the drivers of the
design of the HERA array.
