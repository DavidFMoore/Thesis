\section{Power Spectra}\label{sec:results}

Now that the data are processed and averaged, we begin computing power spectra by a two-step
process. First, we remove off-diagonal covariances from the correlation matrix of two baselines ---
this results in a power spectrum for each integration time of the fiducial, averaged day and for
each baseline pair. Next, we bootstrap multiple times and baseline pairs to characterize the
statistics of the distributions of power spectra.  

\subsection{Covariance Removal}\label{sec:covariance}

Even after having undergone several layers of RFI excision and crosstalk removal, the data still
show large covariances between delay bins. These covariances dominate the averaged power spectrum
despite varying wildly between baseline pairs. We remove them via the covariance removal strategy
described in Appendix C of \citet{Parsons2014}.

This strategy essentially diagonalizes the average covariance matrix between all baseline pairs.
Since we expect all redundant measurements to see the same sky, and we expect all $k$-bins of the
power spectrum to be independent, then we expect sky signal to appear in the diagonal elements of
the covariance matrix between two delay-transformed visibilities. By measuring the full covariance 
matrix of all baseline pairs, we can estimate the instrumental systematics which would leak signal
from one $k$-bin to another. By inverting the mean covariance matrix of all baseline pairs and
dotting this into a delay-transformed visibility, we remove our best guess at the covariances in
that visibility. 

Once we estimate and remove covariances from the delay-transformed visibilities, we may proceed with
power spectrum estimation. We direct the reader to \citet{Parsons2014} for a more detailed discussion. 

\subsection{Results}

\figuremacroW{PQ_vs_t.eps}{1.0}{fig:pq_vs_t}{
  Power spectra, in units of $mK^2\ (h{\rm Mpc}^{-1})^3$ shown for each $k_{||}$ and LST measured.
  $k_{||}$ modes within the horizon for the 30m baselines used are masked. The left panel show $I$,
  the middle panel, $P_P=P_Q+P_U$, and the right panel shows $P_V$. An excess of power in $P_P$
  below LST 4h30m could indicate polarized emission. The excess at $k_{||} \approx 0.35\ h{\rm
  Mpc}^{-1}$ between LST 6h00m and 8h00m could be generated by a 170 mJy source with rotation measure
  between 42 and 59 ${\rm m}^{-2}$.
}{Power spectra in $I$, $P$, and $V$, vs. LST and $k_{||}$}
The covariance removal described in the previous section projects the delay-transformed
visibilities into a basis in which the covariance between two redundant baselines is diagonal, and
then computes the power spectrum from the projected delay spectra. This procedure produces an
estimate to the power spectrum for each LST bin and baseline type. To measure the uncertainties in
the time-dependent power spectra, we bootstrap over groups of redundant baselines. Figure
\ref{fig:pq_vs_t} shows the linearly polarized power spectra from unpolarized emission ($I$),
linearly polarized emission ($P = Q+iU$), and circularly polarized emission ($V$). as functions of
LST, computed via this bootstrapping.

There are two features in $P_P$ worth noting. First is the excess of emission at $0 \lesssim k_{||}
\lesssim 0.2\ h{\rm Mpc}^{-1}$, between right ascension 1h00m and 4h30m. That the excess survives
the LST averaging over 82 days indicates that it is fixed to the sky, and that it exceeds its
corresponding $k$-bins in $P_I$ indicates that it is polarized. Furthermore, it roughly corresponds
with the diffuse, polarized power shown in \citet{Bernardi2013}, which takes its minimum value at 
around LST of 5h00m.

The second feature of Figure \ref{fig:pq_vs_t} that we will comment on is the track of excess power
from fight ascension 6h00m to 8h00m, at $k_{||} \sim 0.35\ h{\rm Mpc}^{-1}$. This type of excess
power could be generated by a polarized point source. This stripe satisfies the two criteria put forth to
indicate the excess at lower right ascension could be polarized emission, and it appears to be
localized in $k_{||}$, a feature of polarized point sources behind a single Faraday screen.

What properties of a polarized point source would be necessary to generate the excess shown? To
answer this question, we describe the source spectrum with three parameters: a flux $S$, a
geometrical delay $\tau_g$, and a rotation measure $\Phi$. The power spectrum of this point sources
$P_1(k)$, taken from Equation \ref{eq:def_vis2pk} becomes 
\begin{equation}
  P_1(k) \approx \left(\frac{\lambda^2}{2k_B}\right)^2\frac{X^2Y}{\Omega}S^2
    \delta\left(k_{||} - \dfdx{k_{||}}{\eta}\tau_g - k_{leak}(\Phi)\right),
\end{equation}
where $dk/d\eta$ is the linear conversion from delay into $k_{||}$ ($1/Y$ of Equation \ref{eq:def_Y}),
and $k_{leak}(\Phi)$ is the $k_{||}$ mode most infected by rotation measure $\Phi$ (Equation
\ref{eq:bad_k}). All other terms agree with Equation \ref{eq:def_vis2pk}.The bandwidth in the
denominator of Equation \ref{eq:def_vis2pk} is removed to properly normalize the delta function.
Note the important result from Section \ref{sec:onesource} that a rotation measure component to a
spectrum \emph{adds} to the normal position of a source in $k_{||}$.

The potential source peaks in power at right ascension of 6h52m at a $k_{||}$ value of $0.345\ h{\rm
Mpc}^{-1}$. Its power peaks at $4.62\times10^8\ {\rm mK}^2(h^3{\rm Mpc}^{-3})$. A point source whose
right ascension is between 6h00m and 8h00m, with an apparent flux of 173 mJy, whose rotation measure
is between 42 and 59 m$^{-2}$ satisfies the requirements to generate this type of emission. To
confirm the existence of such a source would require follow-up observations with an imaging array,
but the excess power at that location in the $P_P$ spectrum is well-described by such a source.

\figuremacroW{Pspec_2x2_37_70.eps}{1.0}{fig:pspec2x2_lo}{
  Spherically averaged power spectra for the four Stokes parameters: $I$ in the top left panel. $Q$ 
  in the top right, $U$ in the lower left, and $V$ in the lower right. The data from Band I are shown. 
  Error basrs show the 98\% confidence intervals derived from bootstrapping over all samples in LST 
  and all redundant baselines. Dashed, cyan lines show the theoretical level of thermal fluctuations, 
  with $T_{sys}$ calculated in Section \ref{sec:Tsys}. 
}{Power spectra for all Stokes parameters in Band I}
\figuremacroW{Pspec_2x2_110_149.eps}{1.0}{fig:pspec2x2_hi}{
  Same as figure \ref{fig:pspec2x2_lo}, for Band II.
}{Power spectra for all Stokes parameters in Band II}
Figure \ref{fig:pspec2x2_lo} shows the power spectra of the four Stokes parameters  in Band I, and
Figure \ref{fig:pspec2x2_hi} sohws that of Band II. Sensitivity limits using the $T_{sys}$ computed
in Section \ref{sec:Tsys} and the sensitivity calculations of \citet{PAPERSensitivity} and
\citet{Pober2014} are shown in dashed, cyan lines. The noise level of these power spectra are
computed by examining the spread of bootstrapped power spectra, where we bootstrap over both
redundant baselines and LST samples. These noise levels are considerably higher than the predictions
from the $T_{sys}$ of Section \ref{sec:Tsys}. The increase in $T_{sys}$ is most
likely due to calibration errors and systematics not targeted by the covariance removal.

The level of leakage predicted from the arguments of Section \ref{sec:BeamLeakage} show that in the
lowest $k_{||}$ bins, the $I$ power spectrum cannot be dominated by $Q\to I$ leakage. The levels of
polarized leakage in $P_I$, to an order of magnitude, are $10^3\ {\rm mK}^2$ in Band I, and $10^2\
{\rm mK}^2$ in Band II. These levels are well below the systematics which dominate the lowest
$k_{||}$ bins of the $I$ power spectrum, and are also well below the levels of $Q\to I$ leakage 
predicted by the simulations in Section \ref{sec:sim_results}. We will investigate this 
further in Section \ref{sec:updates}. 

\subsection{Ionospheric Effects}
Daily changes in the Faraday depth of the Earth's ionosphere could potentially attenuate polarized
signal. As the total electron content (TEC)\nomenclature[Zt]{TEC}{Total Electron Content of the
ionosphere} varies, it modulates the incoming polarized signal by some Faraday depth that which 
is a function of both the local TEC of that time, and the strength of the Earth's magnetic field. 
Though we assume visibilities are redundant in LST, they do have slight variations due to the 
variable TEC of the ionosphere. Thus, averaging in LST could result in some attenuation of signal.

To quantify this, we first assume that the ionospheric TEC is constant over the PAPER beam, and
assume that from day to day, the Faraday depth of the ionosphere is a random variable, based on the
TEC measurements of \citet{Datta2014}. For now, we neglect the day-to-day correlations, though we
can check the effects of any correlation later.

We begin by writing the LST-averaged visibility as the Faraday depth-weighted sum of otherwise
redundant visibilities:
\begin{equation}
  V' = \frac{1}{N}\sum_i e^{-2\Phi_i\lambda^2}V,
\end{equation}
where $\Phi_i$ is the ionospheric Faraday depth from day $i$ and $V$ is the redundant component of
the visibilities. Using this expression, we compute the magnitude of the rotated visibilities, which 
is proportional to the power spectrum,
\begin{equation}
  P' = |V'|^2 = \frac{1}{N^2}\left(\sum_{i,j}e^{-2i(\Phi_i-\Phi_j)\lambda^2}\right)|V|^2.
\end{equation}
When $i=j$, the term in parentheses becomes one, and the $i,j$ component of thes sum is the
conjugate of the $j,i$ component. This allows us to rewrite the sum in terms of the $i=j$ component,
and the $j>i$ components, now written as cosines:
\begin{equation}
  \sum_{i,j} e^{-2i(\Phi_i-\Phi_j)\lambda^2} 
    = N+2
    \sum_{i>j}\cos\left\{2(\Phi_i-\Phi_j)\lambda^2\right\}.
  \label{eq:ionosphere}  
\end{equation}
In the limit where all values of $\Phi_i$ are equal, the second term becomes $N(N-1)/2$, the number
of $i,j$ pairs with $i>j$. This produces the desired result that with no daily fluctuations in
ionospheric Faraday depth, there is no effect on the signal. In the limit of totally uncorrelated
data (i.e. $\langle\Phi_i\Phi_j\rangle\propto\delta_{ij}$), this term takes its minimum value,
maximally attenuating the measured power spectrum. Hence, considering uncorrelated $\Phi_i$ gives
the worst-case scenario.

To estimate the level of ionospheric attenuation, we estimate the attenuation factor in Equation
\ref{eq:ionosphere}. Using typical TEC values of $6\times10^{16}\ {\rm m}^2$ and a typical value 
for the Earth's magnetic field at the PAPER site\footnote{{\tt
http://www.ngdc.noaa.gov/geomag/magfield.shtml}}, we calculate a rotation measure for each day, and
estimate the attenuation factor, and estimate the attenuation factor for eighty-two days, as the data
was averaged over that period of time.

This procedure yields a distribution of values for the attenuation factor which peaks at $88\%$.
Considering that we have neglected day-to-day correlations of the TEC, which increases this factor,
decreasing the level of attenuation, we assume that attenuation due to the ionosphere is negligible
and do not adjust our results for it.
