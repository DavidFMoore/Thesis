
% Thesis Abstract -----------------------------------------------------

\begin{abstracts}

  As the first luminous objects began to form, they heated their surrounding medium, ionizing it. 
  This event is the most recent cosmic phase transition, and occured during what is called the 
  Epoch of Reionization (EoR). The ionization history of the intergalactic medium can be directly 
  measured by 21cm emission from the hyperfine transition of hydrogen. Measurments of the 21cm signal
  from the EoR can yield information about those first luminous objects and help complete our
  understanding of cosmic history. Today, we measure the 21cm EoR signal in radio frequencies. 

  Excavating the 21cm EoR signal from beneath the bright foregrounds present at meter wavelengths
  requires pristine characterization of all foregrounds. We discuss how spectrally smooth foregrounds
  are isolated to particular regions of the 21cm EoR power spectrum, but Faraday-rotated, polarized
  sources can contaminate all regions, even those typically reserved for the 21cm EoR signal. To
  estimate the level of contamination we can expect from polarized foregrounds, we create a
  physically motivated simulation of the polarized sky at these wavelengths. These simulations imply
  that polarized foregrounds will contaminate the power spectrum at levels much higher than the 21cm
  signal.

  To confirm the theories we develop in simulation, we turn to data taken with the Donald C. Backer
  Precision Array to Probe the Epoch of Reionization (PAPER), an array of antennae operating from 100 to 200 
  MHz in the Karoo desert of South Africa. Using data taken during a six month deployment with PAPER
  elements configured into an $8\times4$ grid, we measure the power spectrum of all four Stokes
  parameters. The measured $Q$ power spectrum exceeds its simulated values, allowing us to constrain the 
  input parameters to the simulations. In particular, we are able to limit the mean polarized
  fraction of sources to $2.2\times10^{-3}$, a factor of ten lower than existing measurements at 1.4
  GHz, on which we based the simulations.

  Finally, we present three new tools for characterizing polarized foregrounds.

\end{abstracts}

% ---------------------------------------------------------------------- 
