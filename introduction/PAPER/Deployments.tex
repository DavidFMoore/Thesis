\section{Deployments of the PAPER Array}\label{sec:deploy}

\begin{table}\begin{center}
  \begin{tabular}{ c || c c c c c}
    & Julian Date & $N_{ant}$ & $N_{corr}$ & Configuration & Publications \\
    \hline\hline
    PSA32   & 2455460-469, 538-544 &  32 &  32 & Imaging                       & \cite{Jacobs2011} \\
    PSA64   & 2455743-747, 817-822 &  64 &  64 & Imaging                       & \cite{Jacobs2013b, Pober2013, Stefan2013} \\
    EoR2011 & 2455903-6006         &  32 &  64 & $8\times4$ Grid               & \cite{Jacobs2014, Moore2014, Parsons2014} \\
    EoR2012 & 2456261-383          &  64 & 128 & $8\times8$ Grid               & --- \\
    EoR2013 & 2456620-637          & 128 & 512 & $16\times7$ Grid, 16 Outliers & --- \\
  \end{tabular}
  \caption[PAPER campaigns]{\label{tab:deploy} PAPER Campaigns from 2009 through 2014. $N_{ant}$ is
  the number of antennae, and $N_{corr}$ is the number of inputs to the correlator.}
\end{center}\end{table}

Table \ref{tab:deploy} summarizes PAPER's five major campaigns since I joined the group in
2009. The first two, PSA32 and PSA64 span an entire year, but due to lack of internet connectivity
to site, consisted of two week-long seasons, separated by about six months. These two campaigns were
designed to characterize foregrounds, so the configuration of antenna was designed to maximize
$uv$-coverage. Data from PSA32 yielded a new catalog of sources in the southern hemisphere
\cite{Jacobs2011}. 

In the PSA64 campaign, we doubled both the number of antennae and the number of
inputs to the correlator. Single-polarization data provided confusion-limited images which yielded
one of the first measurements of the foreground ``wedge'' \cite{Pober2013}, described in Section \ref{sec:StatusQuo},
and shown in Figure \ref{fig:wedge}. It also provided precision measurements of the flux of several calibration sources \cite{Jacobs2013b}, and a detailed analysis of the spectral structure of Centaurus A
\cite{Stefan2013}. All four polarization products were correlated on subarrays of 32 antennae to
give data for full-Stokes images.

In 2011, because we had reached the confusion limit in our images, we reconfigured the array into a
grid pattern. \citet{PAPERSensitivity} show the sensitivity benefits of an antenna configuration
which maximizes the number of redundant baselines. The argument can be summarized by noting that
averaging redundant baselines reduces the variance of $T(\B{k})$ by $1/n$, where $n$ is the number of
measurements, but averaging the power spectrum of baselines in the same annulus of constant $k$
reduces the variance of $P(k)$ by $1/n$. Hence, redundant baselines reduce the uncertainty in the
power spectrum by $1/n$, while non-redundant spacings reduce it by $1/\sqrt{n}$. With this fact in
mind, and also considering the necessity of characterizing the polarized power spectrum
\cite{Moore2013}, we arranged a subset of 32 antennae into an $8\times4$ grid, and took data for
about six months. This campaign was our first long integration designed to measure the 21cm EoR
power spectrum. It yeilded three papers: the first limits on X-ray heating of the IGM
\cite[][Figure \ref{fig:parsons2014}]{Parsons2014}, sensitivity limits on multiple redshift bins \cite{Jacobs2014}, and a
characterization of polarized foregrounds \cite{Moore2014}, the main result of this thesis.

Since the EoR2011 campaign, we have doubled the number of inputs to the correlator each year, and in
2013, we increased the number of antennae to 128. In this last campaign, we placed a small subset of
antennae along the perimeter of a 300m circle, centered on the grid. This allows us to increase the
imaging power of the array, allowing for more accurate foreground characterization and mitigation. 
Work is currently underway to analyze the data from these campaigns, but to date, results have
yet to be published. 
