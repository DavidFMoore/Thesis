\section{Instrument Design}\label{sec:design}

\figuremacroW{RFI.eps}{0.6}{fig:RFI}{
  Comparison of RFI environments in South Africa (dark grey) and Green Bank, West Virginia (light 
  grey). The value shown is the probability that a datum survives RFI excision.
}{Latency of RFI in South Africa and West Virginia}

PAPER is located in the Karoo desert, in the Northern Cape province of South Africa, around 60km west of Carnarvon, at
latitude and longitude of $30^\circ43'17.5''{\rm S}$, $21^\circ25'41.9''{\rm E}$. The array sits
approximately 1km east of the KAT7 array, and roughly 0.5km south of the precursor to the Square
Kilometer Array, MeerKAT.\footnote{www.ska.ac.za} Such a remote location is chosen to isolate PAPER
from human-generated radio frequency interference
(RFI)\nomenclature[Zr]{RFI}{Radio Frequency Interference}.
Figure \ref{fig:RFI} compares the latency of RFI with the site in South Africa to that of the
radio-quiet zone in Green Bank, West Virginia, where we operate a test array. Though the South
African site has a pristine RFI environment, certain frequencies must be discarded at all times ---
most notably, for the Orbcomm constellation of tracking satellites at 137 MHz, and for the
International Space Station at 150 MHz. 

\figuremacroW{SexyDipole.eps}{0.6}{fig:sexydipole}{
  Photograph of a PAPER element, showing the groundscreen and the sleeved dipole.
}{Photograph of a PAPER element}

Celestial signal enters through a dipole hoisted above a wire-mesh groundscreen. Four mesh flaps
positioned at roughly a $45^\circ$ angle above the groundscreen increase the response of the dipoles
towards zenith. The dipole is placed within a sleeve, which broadens the response over a wider range
of frequencies, and allows for better impedance matching of the element over such a wide band.
All electrical elements of the antenna are isolated from one another by plastic fittings, which are 
transparent to radio frequencies. Figure \ref{fig:sexydipole} shows a photograph of a PAPER element.   

\figuremacroW{PAPERBeam.eps}{1.0}{fig:beam}{
  (Left panel) East-west cuts through zenith of the PAPER beam, at 127 MHz (cyan) and 164 MHz (black). In these units,
  isotropic emission would have 0 dBi at all angles. These are the two
  central frequencies of the results in Chapter \ref{chap:PowerSpectra}. (Right panel) Effective
  area of an antenna as a function of frequency.
}{Primary beam}

The groundscreen is designed to have a relatively low effective area so that the first nulls of the
dipole response occur below the horizon. Since the location of these nulls is highly
frequency-dependent, they can introduce systematic errors to the high $k_{||}$ modes reserved for
EoR analysis. Such a small effective area results in a large field of view. Figure \ref{fig:beam} shows an axial cut through
the beam at two frequencies and the effective area $A_{eff}$ of the element, defined as
$A_{eff}\Omega = \lambda^2$\nomenclature[Ra]{$A_{eff}$}{Effective area of an antenna}, where $\Omega$
is the field of view.

\figuremacroW{BandPass.eps}{0.6}{fig:bandpass}{
  Model of the PAPER bandpass, normalized to peak at one.
}{PAPER bandpass}
The first stage of amplification occurs at the dipole, boosting the signal by 60 dB. Signal then
propagates along 75$\Omega$, coaxial cable\footnote{The same coming out of the back of your TV!}. The
length of all coaxial cables is set to be roughly equal to minimize the differences of signal travel
time along these cables, which lessens the calibration burden. These coaxial cables pass into
an RFI-shielded enclosure containing receivers and the digital equipment. A final amplification
stage (40 dB), located inside the RFI-enclosure, corrects for signal loss along the cables, and an analog bandpass filter is applied.
This bandpass is designed to have a flat frequency response from 120 MHz to 180 MHz, and attenuate
signal in the FM band below around 107 MHz and also attenuate signal at 200 MHz, which contains the total
power signal. A plot of the bandpass (gain as a function of frequency) is shown in Figure
\ref{fig:bandpass}.

\figuremacroW{PaperFlowChart}{0.6}{fig:AandD}{
  Flow chart of analog and digital systems.
}{Flow chart of analog and digital systems}
Next, the signal is digitized with a sampling rate of 100 MHz. It is then passed through a F-engine,
which computes the Fourier transform using a four-tap, polyphase filter bank. Since the sampling
rate is below the frequencies allowed by the bandpass filter, we measure an aliased copy of the
frequency spectrum, in the second Nyquist zone. Because the frequency channels are aliased and
reversed, the DC signal is contained in the 200 MHz frequency bin.
The integration time of each Fourier transform is set to be around 10 seconds. The Fourier-transformed 
signal is then distributed over 10GbE to the X-engine, which cross-multiplies each signal and computes 
the visibility for each antenna pair. The visibilities are finally written in MIRIAD format
\cite{MIRIAD} and stored for analysis. Figure \ref{fig:AandD} shows a flowchart which summarizes 
the propagation of signal through the analog and digital systems.

For the EoR2012 and EoR 2013 campaigns described in Table \ref{tab:deploy}, there is a final
process after digitization and correlation. Since the data rate goes as the number of baselines
(roughly, the number of antennae squared) campaigns with many-element arrays will have untenably large data
rates. The extreme case is the EoR2013 campaign with 128 antennae. With 1024 frequency channels per 
10 second integration for each of $10^4$ baselines, this campaign has a data rate of over 200 Mbps.
The total data volume of a 120 day season at this rate would be 127 TB, which, with storage costs of
\$150/TB, costs \$19,000 just to keep.

To mitigate the ``big data'' problems of the latter observing seasons, we implement a data
compression system. Data is piped over 1GbE into a large RAID array, and once the night's data is
taken, a small cluster of compute nodes performs a low-pass filter and decimation on the data,
outputting a smaller, more readily analyzed dataset. The algorithm for the compression process 
is described in detail in Section \ref{sec:data}. This reduces the data rate (and the data volume)
by roughly a factor of twenty.

\figuremacroW{DataFlow.eps}{1.0}{fig:datamap}{
  Map of the data flow. Data is taken on site in the Karoo desert, and shipped to Philadelphia via
  Cape Town (CPT). Data may then be served from Philadelphia to other institutions in the United
  States, such as the University of California (UCB).
}{Map of data flow.}
Once the data is compressed, it is stored on small, portable disks and shipped to its final resting
place, a computer cluster in Philadelphia. From this location, it can be processed using a larger
cluster ({\tt folio.sas.upenn.edu}) and served elsewhere in the United States. Figure 
\ref{fig:datamap} shows a map of this process.
