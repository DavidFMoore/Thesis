\section{Observational Prospects}
\label{sec:StatusQuo}

Now that we have discussed the scientific goals of 21cm EoR observations, we turn to two state of the
art measurements. These two datasets are taken with the PAPER experiment, the primary instrument
used for the result of this thesis, described in detail in Chapter \ref{chap:PAPER}. One of these
measurements is the power spectrum of foregrounds; the other is an upper limit on the EoR power
spectrum, Combined, they provide context for the work in this thesis, and help elucidate some of the observational 
challenges.

\subsection{Foregrounds}

Perhaps the biggest hurdle to measuring the 21cm EoR power spectrum is mitigating the
contributions of foregrounds to the power spectrum. As a general rule, the synchrotron emission 
from our galaxy and extragalactic radio sources are around $10^4$ to $10^5$ times brighter 
than the expected level of the spin temperature at peak reionization. To give representative values, 
the expected brightness temperature of a neutral hydrogen bubble at redshift of 10 is about 30 mK 
(Equation \ref{eq:def_Tcontrast}), but the contributions of foregrounds fall in the hundreds of
degrees Kelvin. 

There are two main strategies for excavating the 21cm signal from underneath the foregrounds,
model subraction, and avoidance. Modelling and subtracting source spectra allows observers to
access the underlying EoR power spectrum, but requires a high level of precision in both the model
spectrum and instrumental calibration terms \cite[][e.g.]{Bowman2009,Datta2010}. While some
strategies require precise imaging of sources, many have simply removed the few brightest principal
components from their spectra \cite[][e.g.]{Liu2009b}. This method has been used with some success
\cite{Dillon2014,Paciga2013}, but as \citet{Paciga2013} demonstrated, blind subtraction of principal
components can also remove components of the EoR power spectrum, severely decreasing sensitivity. 

The second strategy for dealing with foregrounds is through avoidance. Synchrotron emission
generally follows a spectral power law, and thus will accumulate in the lowest bins of $k_{||}$,
while the 21cm EoR signal falls in higher $k_{||}$ modes as well. Hence, foregrounds may be avoided
by focusing analysis only on the highest $k_{||}$ modes.

\figuremacroW{PoberWedge.eps}{1.0}{fig:wedge}{
  Power spectrum of foregrounds, in ${\rm mK}^2\ (h^3{\rm Mpc}^{-3})$. The data presented were taken
  for four hours during the PSA64 season (Table \ref{tab:deploy}). Smooth spectrum foreground
  emission is contained within a foreground ``wedge,'' delimited by the baseline length. A white
  line shows the baseline length, and an orange line shows the baseline length plus a 50 ns buffer.
  This buffer encloses the foreground emission, convolved by the kernel of the power law spectra
  typical to radio sources. Figure taken directly from \citet{Pober2013}.
}{Measurement of foregrounds in the $k_\perp$-$k_{||}$ plane, from \citet{Pober2013}}
A number of studies show evidence of a foreground ``wedge'' in the $k_\perp$-$k_{||}$ plane for
smooth-spectrum foregrounds \cite[][e.g.]{Morales2012, DelaySpectrum, Liu2014}. This wedge 
is due to the spectral response of an interferometer, and as we will show in Section \ref{sec:DDR},
is set by the geometry of an interferometric array. Figure \ref{fig:wedge} shows the first
observational confirmation of the ``wedge.'' The region above and to the left of the orange lines in
Figure \ref{fig:wedge} is relatively free of contamination from foregrounds, and can be designated
as the primary target for observations.

A negative consequence of restricting analysis to high $k_{||}$ modes is the relative levels of the
21cm EoR power spectrum and uncertainty due to thermal noise. Typically, observers target the
spherically averaged power spectrum ($\Delta^2(k)$ in Equation \ref{eq:def_k3pk}), in which the 21cm
EoR signal is relatively flat, and uncertainty due to thermal noise rises as $k^3$. Thus, observers
target the lower $k_{||}$ modes of the upper triangle of Figure \ref{fig:wedge}. This effect
provides the impetus for using short baselines for EoR analysis, since short baselines probe the
lowest values of $k_\perp$, and restrict foregrounds to the smallest region in $k_{||}$.  

Future observations and possible detections will likely use a combination of both strategies.
\citet{Pober2014} show that the most highly sensitive modes to the power spectrum exist within the
foreground wedge, and the best prospects for detection are those in which inter-horizon modes can be
accessed.

\subsection{Current Upper Limits to the Power Spectrum}

To date, there are two prominent upper limits to the 21cm EoR power spectrum: \citet{Parsons2014},
and \citet{Paciga2013}. Both of these measurements have overcome significant hurdles in foreground
removal and avoidance (respectively), and show that great progress is being made to the detection of
the 21cm EoR power spectrum. Since this thesis focuses on data taken with the PAPER array (Chapter
\ref{chap:PAPER}), we will focus on the Parsons result, taken with the PAPER instrument.

\figuremacroW{ParsonsResults.eps}{0.8}{fig:parsons2014}{
  (Left Panel): Power spectrum measurements from the EoR2011 observing season. (Right Panel):
  Spherically averaged power spectrum taken from that same data. Dashed, vertical lines show the
  horizon limit for the $16\lambda$ baselines used for these measurements. Points and error bars
  show covariance-removed data, and 95\% confidence intervals (See Section \ref{sec:data} and 
  \ref{sec:covariance} for details). Cyan lines show the $2\sigma$ upper limit of data without the
  covariance removal applied. Dashed, cyan lines show the level of thermal fluctuations, assuming
  $T_{sys}=550\ {\rm K}$. The magenta line shows a fiducial 21cm EoR model, taken from
  \citet{Lidz2008}.  Yellow triangles show the $2\sigma$ upper limits presented in
  \citet{Paciga2013}. Figure taken directly from \citet{Parsons2014}.
}{Best upper limits to date on the 21cm EoR power spectrum, from \citet{Parsons2014}}
Figure \ref{fig:parsons2014} shows the power spectrum at a redshift of 7.7, measured in the EoR2011 Season 
(Table \ref{tab:deploy}), alongside a fiducial model of the power spectrum \cite{Lidz2008}. While
uncertainties from thermal noise, residual foregrounds, and other instrumental systematics prevent a
detection of the power spectrum, this measurement can constrain the brightness contrast of the spin
temperature to below (41 mK$)^2$ at $k = 0.27\ h{\rm Mpc}^{-1}$. This value allows for one of the
first physical constraints to the history of the IGM, since it can rule out reionization scenarios
in which the IGM cools adiabatically, with no heating from early ionizing sources. Essentially,
restricting the brightness temperature to below $41\ {\rm mK}$ shows that X-ray heating of the IGM
contributed to the bulk of reionization. Scenarios in which the IGM cools adiabatically are highly
unlikely, but this measurement is one of the first observational confirmations of that statement.
