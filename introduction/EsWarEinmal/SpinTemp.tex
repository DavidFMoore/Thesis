\section{Spin Temperature}\label{sec:SpinTemp}

We begin our investigation into the utility of the hyperfine transition of HI as a probe of the EoR
by solving the radiative transfer equation for a simple model of a neutral IGM. This is done in an
attempt to gain intuition about the processes which generate 21cm emission, and to allow for a
discussion of the environments in which it is produced. Much of this discussion will follow
\citet{FOB} and class notes from Adam Lidz, but specific results will be individually cited.

The relative occupancy of hyperfine states in hydrogen gas can be characterized by the spin
temperature $T_{s}$\nomenclature[Rt]{$T_s$}{Spin Temperature}, defined in a Saha-like equation 
\begin{equation}
  \frac{n_1}{n_0} = 3\exp\left\{\frac{T_*}{T_s}\right\}.
\end{equation}
Here, $n_1$ and $n_0$ are the density of hydrogen in the triplet and singlet states, respectively;
the factor of three represents the threefold degeneracy of the triplet state to the singlet state,
and $T_*$ is the line-temperature, defined such that $k_BT_* = h\nu_{21}$. For 21cm emission, $T_* =
68\ {\rm mK}$.

The intensity of 21cm emission can be calculated via the radiative transfer equation,
\begin{equation}
  \dfdx{I_\nu}{\tau} = - I_\nu + s_\nu.
  \label{eq:RTE}
\end{equation}
With the usual notation, $I_\nu$ is the intensity of 21cm emission; $\tau_\nu$ is the optical depth,
here a proxy for distance along the line of sight; and $s_\nu = j_\nu/\alpha_\nu $ is the source function.

By an argument of dimensional analysis (${\rm d}E = j_\nu{\rm d}V{\rm d}\Omega{\rm d}\nu{\rm d}t$),
we can write the emission coefficient in terms of the density of hydrogen atoms in the triplet state
$n_1$, the line-profile of the emission line $\Phi(\nu) \approx \delta(\nu-\nu_{21})$, and the
Einstein coefficient for the transition rate $A_{10}$:
\begin{equation}
  j_\nu = \frac{h\nu}{4\pi}n_1A_{10}\Phi(\nu).
  \label{eq:jnu}
\end{equation}
In a similar fashion, we can write the absorption coefficient in terms of the Einstein coefficient
for photoabsorption, $B_{01}$, and stimulated emission $B_{10}$:
\begin{equation}
  \alpha_\nu = \frac{h\nu}{4\pi}\Phi(\nu)(n_0B_{01} - n_1B_{10}).
\end{equation}
We can write all Einstein coefficients in terms of the emission rate $A_{10}$ to write 
$\alpha_\nu$ in the more managable form,
\begin{equation}
  \alpha_\nu = 
  3n_0A_{10}\frac{\lambda^2}{8\pi}\Phi(\nu)\left(1-e^{T_*/T_s}\right).
\end{equation}
Finally, we write the source function, easing the burden by assuming that $T_s \gg T_*$:
\begin{equation}
  s_\nu = \frac{j_\nu}{\alpha_\nu}
  = \frac{2k_B}{\lambda^2}T_s.
  \label{eq:source_function}
\end{equation}
In the Rayleigh-Jeans limit, the spin temperature is the source function for the temperature of 21cm
emission within a cloud of hydrogen gas.

Solutions to the radiative transfer equation (always in the Rayleigh-Jeans limit) take the form
\begin{equation}
  T = T_{ex}(1-e^{-\tau_\nu}) + T_{bg}e^{-\tau_\nu},
\end{equation}
where $T_{ex}$ is the temperature associated with the source function $s_\nu$, and $T_{bg}$ is the
temperature of background radiation --- in our case, the temperature of the CMB at redshift $z$. In the limit
where the optical depth is small, which we will justify shortly, the solution to the radiative transfer equation
becomes
\begin{equation}
  T = T_s\tau_\nu + T_{bg}(1-\tau_\nu),
  \label{eq:RTE_tau}
\end{equation}
where we take advantage of the source function's being the spin temperature.

The optical depth, $\tau_\nu$, can be calculated by integrating the absorption coefficient along the
line of sight. This can be written in terms of the more cosmologically interesting quantities:
the neutral fraction of hydrogen $x_{HI}$, the matter overdensity field $\delta$, the density of Hydrogen
$n_H$, the Hubble parameter $H(z)$\nomenclature[Rh]{$H(z)$}{Hubble parameter}, and the peculiar velocity 
of the Hydrogen cloud per unit length along the line of sight, $dv_{||}/dr_{||}$.
\begin{equation}
  \tau_\nu = \int \alpha_\nu\D{s} = 9.2\times10^{-3}(1+\delta)(1+z)^{3/2}\frac{x_{HI}}{T_s}
  \left[\frac{H(z)/(1+z)}{dv_{||}/dr_{||}}\right]
  \label{eq:def_opticaldepth}
\end{equation}
The fiducial value of $10^{-2}$ justifies the assertion that $\tau_\nu \ll 1$, which we used to derive
Equation \ref{eq:RTE_tau}. Such a small optical depth also provides a justification for preferring 
the hyperfine transition over Lyman-$\alpha$ as a probe of the EoR.

We can give an approximate solution for the contrast in brightness of 21cm emission to that of the
CMB by inserting Equation \ref{eq:def_opticaldepth} into Equation \ref{eq:RTE_tau} and rearranging
terms. Noting that $T_{bg} = T_{CMB}(1+z)$, where $T_{CMB}$ is the current temperature of the CMB
(2.4 K), we write the brightness contrast as
\begin{align}
  \delta T \equiv& \frac{T_S - T_{CMB}(1+z)}{(1+z)}
  \nonumber \\ \approx&
  9\ {\rm mK}\ x_{HI}(1+\delta)(1+z)^{1/2}
  \left[1 - \frac{T_{CMB}(1+z)}{T_s}\right]
    \left[\frac{H(z)/(1+z)}{dv_{||}/dr_{||}}\right]
  \label{eq:def_Tcontrast}
\end{align}

The point of writing this rather tedious calculation is to elucidate the processes that
generate 21cm radiation, and how we may detect it. There are three salient points: first, we detect the
brightness contrast between 21cm radiation and the CMB; second, the spin temperature provides the
source of that contrast, so the brightness contrast yields information on the astrophysical
processes which drive the spin temerature; and third, the brightness contrast is proportional to the
neutral fraction, so the cosmic ionization history can be derived from measurements of the
brightness contrast over many redshifts. While cosmological parameters may be derived from the
brightness contrast \cite{McQuinn2006}, there is also a wealth of rich astrophysics to be gleaned
from the evolution of the spin temperature. 

Thorough discussions of the evolution of the spin temperature may be found in \citet{FOB} and
\citet{Pritchard2012}, which we summarize here. First, we will discuss the evolution of the spin
temperature's global average, and then discuss the fluctuations.

There are two main physical processes which drive the evolution of the spin temperature. The first
is collisional excitation of HI --- as an HI atom scatters off of another species of particle
(usually another HI atom), some of the kinetic energy of the collision is transferred to excite the
hyperfine structure of the atom. This couples the spin temperature field to the underlying baryon
density field ($\delta$ in Equation \ref{eq:def_Tcontrast}). This coupling depends on the
density of the baryon field, which varies due to the growth of structure and the expansion of the
universe. It also depends on the kinetic energy of that field, typically represented by the kinetic
temperature of the gas.

The second process that drives the evolution of the spin temperature is a coupling to the underlying
UV radiation field. A UV photon will excite the electron in an HI atom from the 1S state in the
hyperfine singlet state into the 2P state. When the electron decays back into the 1S state, it
may not return to the hyperfine singlet state, but to the hyperfine triplet state. This is called
the Wouthuysen-Field effect \cite{Wouthuysen1952, Field1958}. This effect introduces astrophysics
into the spin temperature by allowing it to be affected by UV emitters --- early stars, galaxies,
quasars, and small black holes. Because of this, the spin temperature is an excellent probe of the
early universe. 

\figuremacroW{GlobalSignal.eps}{0.6}{fig:globalsignal}{
  Mean brightness contrast versus redshift and frequency for a fiducial model of reionization. ``Turning
  points'' are labelled and explained in the text. The blue region shows where the brightness
  contrast is seen in absorption; red, in emission. Figure credit: \citet{Harker2012}.
}{Global 21cm signal vs. $z$, from \citet{Harker2012}}
The interplay of these two processes can be summarized by the ``turning points'' of the global
spin temerature \cite[][e.g.]{Harker2012,Mirocha2013,Liu2013}. These are the points in time at which
the time-derivative of the brightness contrast is zero, i.e. the spin temperature transitions from
increasing to decreasing. Figure \ref{fig:globalsignal} shows a sample
model of the evolution of the global spin temperature, to which we refer to explain the ``turning
points.'' (A) At around a redshift of 80, the spin temperature becomes uncoupled to the gas kinetic
temperature, mostly due to the baryonic density growing smaller by cosmic expansion. It recouples to
the CMB temperature, increasing the spin temperature. (B) As the first stars begin emitting UV photons, sometime around redshift 50,
they decrease the spin temperature through the Wouthuysen-Field effect. (C) X-ray heating of the IGM, probably from accreting black holes, raises the spin
temperature at around redshift 20. (D) The spin temperature begins to fall due to the start of
reionization around redshift 13. (E) reionization ends --- since the neutral fraction
is nearly zero by redshift 6, the brightness contrast also falls to zero.

Information from the EoR can be obtained from the global signal between redshifts of 13-6, but a much
richer story can be told by looking at fluctuations in the spin temperatue during the EoR. These
fluctuations are characterized by a power spectrum, described in the next section. The power
spectrum of the spin temperature will clearly depend on the underlying baryonic power spectrum ---
it depends explicitly on the baryon density field (Equation \ref{eq:def_Tcontrast}. But the power
spectrum of spin temperature fluctuations is much more complex, due in part to the growth of ionized
structures surrounding UV emitters. An over-simplified explanation is that ``bubbles'' of ionized
hydrogen within the neutral hydrogen field arise as an X-ray or UV emitter ionizes its surrounding
gas. As these bubbles first form, they introduce small-scale structure to the spin temperature
field, steepening the power spectrum. As they  grow, they flatten the slope of the power spectrum. 
As they merge, they introduce small-scale structure, causing the power spectrum to re-steepen towards 
high wavenumber. Finally, as reionization comes to completion, the amplitude of the power spectrum
goes to zero.

Measuring both the mean spin temperature and fluctuations in the spin temperature field will allow
us to discover both the timing of reionization and the nature of reionizing sources. This will give
us a clearer picture of cosmic history and potentially uncover new physics.
