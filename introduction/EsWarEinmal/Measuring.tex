\section{Characterizing Fluctuations of the Spin Temperature}
\label{sec:measuring}

We begin by defining the correlation function \nomenclature[G]{$\xi(\B{r})$}{Correlation function of
spin temperature field} to a temperature field $T(\B{x})$:
\begin{equation}
  \xi(\B{r}) = \frac{1}{\mathbb{V}} \int T(\B{x})T^*(\B{x}-\B{r}) \D{^3x}, 
  \label{eq:correlation}
\end{equation}
where $\mathbb{V}$ is the cosmological volume over which the field is sampled, and the limits of the
integral extend over $\mathbb{V}$. This is more appropriately viewed as a power spectrum, which is
simply the Fourier transform of the correlation
function\nomenclature[Rp]{$P(\B{k})$}{Power Spectrum of
Spin Temperature}
\begin{equation}
  P(\B{k}) = \int \xi(\B{r})e^{-i\B{k}\cdot\B{r}}\D{^3r}, 
  \label{eq:def_pk}
\end{equation}
where $\B{k}$ is the wave-number, typically measured in $h{\rm Mpc}^{-1}$.\footnote{Note the
  difference in Fourier convention between the theorist's power spectrum and that of an
interferometer.}
Due to the convolution theorem, this is simply
\begin{equation}
  P(\B{k}) = \left|\widetilde{T}(\B{k})\right|^2,
\end{equation}
where we have defined the Fourier-transformed temperature field as 
\begin{equation}
  \widetilde{T}(\B{k}) = \frac{1}{\mathbb{V}}\int T(\B{x})e^{-i\B{k}\cdot\B{x}}\D{^3x}.
  \label{eq:def_Tk}
\end{equation}
The power spectrum defined here is the goal of our measurements and the quantity which most easily
allows us to compare different models of reionization and track the evolution of neutral hydrogen in
the intergalactic medium at different scales.

Isotropy demands that $P(\B{k})$ is rotationally invariant, which implies that the power spectrum is
only dependent on the magnitude of the $k$-mode, i.e. $P(\vec{k}) = P(k)$. Because of this, it is
customary to spherically average the power spectrum with log-spaced bins, defining the quantity
$\Delta^2(k)$\nomenclature[G]{$\Delta^2(k)$}{Spherically-averaged power spectrum}, often called the
``dimensionless" power spectrum\footnote{Theorists tend to normalize the power spectrum by the
global spin temperature, making this a truly dimensionless quantity. Since the global spin 
temperature is as interesting and as unknown as the power spectrum, we do not use this
normalization. Our values of $\Delta^2(k)$ will have units of temperature squared.} as
\begin{equation}
  \Delta^2(k) \equiv \frac{1}{(2\pi)^3}\int P(\B{k}) k^3\D{\log{k}}\D{\Omega} 
    = \frac{k^3}{2\pi^2}P(k)
    \label{eq:def_k3pk}
\end{equation}

It is our task to define a method by which we can detect $P(\B{k})$.  We begin by inspecting the visibility 
(Equation \ref{eq:visdef}) and comparing it with the expressions in Equations \ref{eq:def_pk} and \ref{eq:def_Tk}. 

In the flat sky limit, neglecting calibration terms (including the primary beam which will be
discussed later), the visibility reads
\begin{equation}
  V(u,v,\nu) = \int I(l,m,\nu) e^{-2\pi i (ul+vm)}\D{l}\D{m}.
\end{equation}
As we will discuss in Section \ref{sec:interferometry}, this is the two-dimensional Fourier transform 
over direction cosines $l$ and $m$.

First, we note that the specific intensity of the observation, $I(l,m,\nu)$, is directly
proportional to the spin temperature field $T(\B{x})$. The constant of proportionality and 
the $k$-modes sampled by an observation are set by the limits of the observation, and will be
discussed. 

If we can assume that the measured range in $l$ or $m$, call it $\Delta\theta$ is small, then the
comoving distance subtended by $\Delta\theta$ can be written as a linear scaling, 
$r_{com} \approx X\Delta\theta$. We note that $X$ is simply the comoving distance to redshift $z$, 
and can be found by integrating $c/H(z')$ with respect to $z'$ from redshift 0 to $z$. In general, 
this expression is complicated, but for redshifts 5-15, this can be written in terms of a simple 
power law.
\begin{equation}
  X = 6.5\times10^3\left(\frac{\Omega_m}{0.27}\right)^{-1/2}
  \left(\frac{1+z}{10}\right)^{0.2}\ h^{-1}{\rm Mpc}\ {\rm rad}^{-1},
  \label{eq:def_X}
\end{equation}
where $\Omega_m$ is the cosmic matter density\nomenclature[G]{$\Omega_m$}{Cosmic matter density in
units of the critical density $\rho_c=3H^2/8\pi G$}. We use WMAP7 values for cosmological paramters
\cite{Komatsu2009} to derive this number. This follows the expression in \citet{FOB}.

Having given a prescription to convert a measured quantity into the two transverse cosmological
distances, we now turn to the line of sight distance. Taking advantage of the one-to-one mapping
from the redshift of a 21cm line to the distance to it, we write the distance spanned by bandwidth
$\Delta\nu$ as $\Delta D \approx (dr_{com}/dz)(dz/d\nu)\Delta\nu$. Defining the slope 
$\Delta D \equiv Y\Delta\nu$, we find another simple, linear, scaling relation: 
\begin{equation}
  Y = 1.7\times10^{-2}\left(\frac{1+z}{10}\right)^{1/2}\left(\frac{\Omega_m}{0.27}\right)^{-1/2}\
  h^{-1}{\rm Mpc}\ {\rm GHz}^{-1}.
  \label{eq:def_Y}
\end{equation}

This discussion can be summarized into three main points:
\begin{enumerate}
  \item The sky intensity $I(l,m,\nu)$ is proportional to the spin-temperature field. 
  \item Angles on the sky can be converted into transverse $k$-modes.
  \item The frequency dimension measures line-of-sight $k$-modes.
\end{enumerate}
We have given approximate proportionality constants between measured $l,m,\nu$ coordinates and 
the cosmological $x,y,z$, and a more detailed discussion of power spectral inference will be given
in Section \ref{sec:DelaySpectrum}.
