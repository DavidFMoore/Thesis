\section{The Early Universe}\label{sec:Genesis}

13.7 billion years ago, when the universe was only about 380,000 years old, the cosmic radiation
background (CMB)\nomenclature[Zc]{CMB}{Cosmic Microwave Background Radiation}\footnote{I hesitate to
  call it the Cosmic Microwave Background (which is what CMB stands for) when talking about it
  during recombination. The ``M'' in CMB represents the fact that today, its brightness peaks at
  microwave frequencies. During recombination, it peaked in the UV. Do we call it the CUVB then?
Hence, ``cosmic radiation background.''} cooled to the point where 
its photons could no longer dissociate hydrogen, allowing the first stable atoms to form. This point
in cosmic time is often called the surface of last scattering or the epoch of recombination, and is
the earliest event we can measure, since before this time, the scattering rate of photons with early 
protons and electrons was so great as to render the universe opaque. 

The relic abundance of photons from this period (the CMB) has provided a pristine picture of the
initial conditions of our universe, and has provided much of the evidence for our current
understanding of cosmology. It is truly remarkable how simple these initial conditions were --- the
standard model of cosmology, ``vanilla'' $\Lambda$CDM can be fully characterized by only eight
numbers! Today, such a simplistic representation of our current universe would be absurd --- with the
growth of structure also came the growth of complexity. One of the main goals of the study of 
cosmology is to answer the question, how did the complex universe we see today arise?

To begin answering this question, we turn to the universe as it was just after recombination. Baryonic matter 
in the universe mostly consisted of neutral hydrogen. Quantum fluctuations in the radiation background 
yielded fluctuations in the density of matter. Fluctuations in the matter field allowed the first structures 
to form. The increased gravitational potential in the overdensities drew in the surrounding matter, making 
the overdensities more dense, and the underdensities less dense. Eventually, overdensities above some threshold density
collapsed into galactic haloes --- self-supporting structures held together by their own
gravitational potential.

Within these haloes, the first stars and galaxies formed. It is in these stars and galaxies that we
are interested for this thesis. As we will see in the next section, few Lyman alpha
(Ly$\alpha$)\nomenclature[Zl]{Ly$\alpha$}{Lyman alpha} photons, the most 
abundant source of radiation from HI escaped these early times without being scattered. Without
looking to other tracers of the HI, we know very little about this period of time. Because of this effect, some call
this period ``the cosmic dark ages.'' Others call it ``cosmic dawn,'' because it was at this time
that the first luminous objects originated.

The universe we see today is much different than it was during the cosmic dark ages. Stars have processed
hydrogen and helium into heavier elements, allowing for a more complex chemistry. Galaxies have
merged and evolved into much more complex structures than their relatively simple progenitors.
Notably, UV emission from early luminous objects ionized the field of neutral hydrogen
surrounding them, the intergalactic medium
(IGM)\nomenclature[Zi]{IGM}{Intergalactic Medium}. Today,
we measure a highly ionized IGM, but we know that without a neutral IGM, the CMB could not have
arisen. The period of time when the IGM transitioned from neutral to ionized, called the Epoch of
Reionization (EoR)\nomenclature[Ze]{EoR}{Epoch of Reionization} captivates researchers, as it is the
most recent cosmic phase transition.

By characterizing the ionization history of the IGM during the EoR, we can begin to solve some of 
the mysteries of the early universe. In what environment did the first luminous objects arise? What 
were they? When did the universe begin to look as it does today? These questions allows us to fill
the gaps in our understanding between relatively thorough knowledge of the universe at early and
late times.

\section{How Do We Measure The Earliest Galaxies?}

Now, we turn our attention to observations of the IGM during the EoR, focusing on both its timing
and the nature of reionizing galaxies. Not only will this discussion elucidate the methods of
detecting the signature of the first stars and galaxies, but it will also show the limits of our
understanding of the EoR.

There are three main methods to measure HI during the EoR: the absorption of Ly$\alpha$
emission from high-redshift quasars, the scattering of CMB photons off of free electrons after
reionization, and 21cm emission from the hyperfine transition of HI. We will discuss
these three measurements in turn, focusing on the advantages and limitations of 
each. We will argue that the measurement most likely to detect signatures of reionization is the
21cm power spectrum, on which we will focus for the remainder of this thesis.

\subsection{High Redshift Lyman Alpha Emitters}

Ly$\alpha$ emission from high-redshift quasars is both relatively abundant and relatively
bright, so it would seem to be an excellent candidate for detecting the signature of HI during the 
EoR. Ly$\alpha$ photons have a high cross-section to HI and will quickly become absorbed in a
neutral IGM. As a Ly$\alpha$ photon from a quasar passes through a cloud of HI, it becomes
absorbed, creating a dip in the quasar spectrum. \citet{GunnPeterson} investigated this effect in
their seminal paper, predicting that, for a quasar in a highly neutral interstellar medium, all
emission blue-wards of Ly$\alpha$ in the quasar's rest frame would be nearly completely 
absorbed. Furthermore, they suggest using the optical depth of Ly$\alpha$ to that quasar yields
a measure of the neutral hydrogen fraction, integrated along the line of sight. 

\figuremacroW{Gunn-Peterson.eps}{1.0}{fig:GunnPeterson}{
  Quasar spectra at redshift 5.80 (cyan), 5.82 (green), 5.99 (magenta), and 6.28 (red), showing the
  near-total absorption of Ly-$\alpha$ in the high-redshift IGM. This so-called Gunn-Peterson trough 
  indicates the presence of neutral hydrogen in the quasar environment. Figure taken from
  \citet{Becker2001}.
}{Detection of the Gunn-Peterson trough with high-$z$ quasars, from \citet{Becker2001}}
To date, several measurements have been made of these so-called Gunn-Peterson troughs
\cite[][e.g.]{Fan2006,Becker2001}. Figure \ref{fig:GunnPeterson} shows measurements from
\citet{Becker2001} of several quasar spectra, demonstrating the increasing optical depth of the
Ly$\alpha$ line with redshift. This indicates the increase of the neutral fraction with
increasing redshift. Measurements like this one indicate the presence of HI in the IGM as late as
$z\sim6$. Measuring the global neutral fraction through this method is uncertain, as each
measurement of $x_{HI}$ depends on the line of sight to the quasar, and cannot measure a global
quantity \cite{Taylor2014}.

\citet{GunnPeterson}, in a nearly identical calculation to the one presented in Section 
\ref{sec:SpinTemp}, show that the optical depth of Ly$\alpha$ through neutral hydrogen 
is, to an order of magnitude, $\tau_\nu \sim 10^4x_{HI}$, where $x_{HI}$ is the neutral fraction of 
hydrogen. This poses the difficult problem that nearly all Ly$\alpha$ emission passing 
through an IGM with $x_{HI} \gtrsim 10^{-4}$ is absorbed. Thus, these measurements can only access the
late stages of the EoR, and will prove to be difficult to use in measuring $x_{HI}$ during peak
reionization. 
Ly$\alpha$ emitters can also be measured without spectroscopy. Objects with an excess
brightness in one photometric band which could correspond to highly-redshifted Ly$\alpha$
are called ``Lyman break'' galaxies. Lyman break galaxies can be used to measure the luminosity function 
--- and thus the mass function --- of galaxies as a function of redshift \citet[][e.g.]{Bouwens2011}. 

\figuremacroW{LuminosityFunction.eps}{0.6}{fig:dNdL}{
  Luminosity function of Lyman break galaxies, in several redshift bins. These data indicate a
  steepening of the luminosity function with age. Figure taken from \citet{Bouwens2011}.
}{Luminosity function of Lyman break galaxies, from \citet{Bouwens2011}}
Figure \ref{fig:dNdL} shows the luminosity function for several Lyman break galaxies in a series of
redshift bins. The luminosity function seems to steepen with age, indicating fewer, older luminous 
galaxies; hence, galaxies during the EoR were most likely relatively small. There are two major sources
of uncertainty in this measurement, though. First, since the redshift Lyman break galaxies are 
measured photometrically, there is a high uncertainty in their redshift. This uncertainty may be
caused by a (photometric) degeneracy between redshift 6 Ly$\alpha$ and a redshift 2 OIII line, allowing for
only $\sim85\%$ confidence in the redshifts measured for these galaxies. Second, since only the
brightest galaxies from this epoch can be measured, one must extrapolate measurements of the
luminosity function to lower brightness, creating a high degree of uncertainty. Nonetheless, these
measurements indicate that reionization was most likely driven by low-mass, low-brightness galaxies. 

\subsection{Hints from the CMB}

Free electrons present after reionization will scatter CMB photons, suppressing the overall
amplitude of the temperature power spectrum by a factor of $\exp\{-\tau_{ri}\}$, where 
$\tau_{ri}$ is the optical depth of a CMB photon through those free electrons 
\cite{Zaldarriaga1997-parameters}. The global ionization history can be estimated by $\tau_{ri}$,
and in fact, $\tau_{ri}$ is one of the free parameters of vanilla $\Lambda$CDM 
\cite[][e.g.]{Spergel2003, Komatsu2009}, but there is a clear degeneracy between $\tau_{ri}$ and the
overall scaling of the $TT$ power spectrum.

\figuremacroW{KogutReionization.eps}{0.6}{fig:Kogut}{
  $TE$ cross-correlation from WMAP, showing the first detection of the effects of reionziation 
  on the CMB. Thompson scattering of CMB photons off of free electrons produced during reionization
  creates polarized power on scales of $\ell\sim2\sqrt{\tau_{ri}}$, where $\tau_{ri}$ is the optical
  depth of a CMB photon through reionization. Figure from \citet{Kogut2003}.
}{The first detection of the signature of reionziation in the CMB, from \citet{Kogut2003}}
This degeneracy is broken by analyzing the polarization of the CMB. If a CMB photon Thompson-scatters 
off of an electron from within a quadrupolar temperature anisotropy, the CMB becomes
linearly polarized (this is one generator of E-modes). This type of scattering during reionization
contributes power to the $EE$ spectrum, generating a peak whose amplitude is proportional to
$\tau_{ri}$, and whose location is around $\ell \sim 2\sqrt{\tau_{ri}}$
\cite{Zaldarriaga1997-reionization}. Figure \ref{fig:Kogut} shows the first detection of this
effect, seen in the $TE$ cross-correlation using the Wilkinson Microwave Anisotropy Probe
(WMAP)\nomenclature[Zw]{WMAP}{Wilkinson Microwave Anisotropy Probe}.

Extracting the ionization history from $\tau_{ri}$ is highly model dependent, since $\tau_{ri}$ is
an integral quantity, summing information from all times since last scattering. Typically,
researchers will assume simple models for the ionization history and model the optical depth as 
$\tau_{ri} \propto \int_0^{z_{recomb}}x_{e}(z)(1+z)^{-1}(dl/dz)\D{z}$, where $z_{recomb}$ is the
redshift of recombination, $x_{e}(z)$ is the ionization history, and $dl/dz$ is the cosmological 
line-element at redshift $z$. Popular models for $x_{HI}(z)$ include ``instantaneous reionization,'' 
in which $x_{HI}(z)$ is 1 until the redshift of reionization $z_{reion}$, and 0 afterwards, and a
model including a sustained reionization, $x_{HI}(z) = \tan^{-1}\{(z-z_{reion})/\Delta z\}$, where
$\Delta z$ is the duration of reionization.

\figuremacroW{ZahnCMB.eps}{0.6}{fig:xHIz}{
  Summary of measurements of $x_{HI}(z) \approx 1-\bar{x}_e(z)$ including information from both the CMB data and
Ly$\alpha$ emitters, from \citet{Zahn2012}. Contours show the 1- and $2\sigma$ confidence
intervals of $x_{HI}(z)$ taken from data from WMAP and the South Pole Telescope (SPT). Points,
upper, and lower limits show measurements from Ly$\alpha$ emitters (citations shown in the figure).
}{Summary of measurements of $x_{HI}(z)$ \cite{Zahn2012}}
A final constraint on the ionization history of the IGM that can be drawn from the CMB is due to the
kinetic Sunyaev-Zel'dovic effect. CMB photons will Doppler-shift as they scatter off of moving
electrons \cite{Sunyaev1980}. The flow of free electrons generated by ionized bubbles will generated 
this Doppler shift in the CMB, and will add structure to the $TT$ power spectrum \cite{Zahn2012}.
Measurements of this type are both heavily model dependent and vulnerable to systematic errors due
to imperfect foreground removal. Figure \ref{fig:xHIz} shows inferences of the ionization history 
from measurements of this effect from \citet{Zahn2012}, alongside measurements from CMB polarization 
and Ly$\alpha$ emitters. 

\subsection{Hyperfine Transition of Neutral Hydrogen}

We now turn our attention to the third method of detecting the signature of the EoR, emission from the
hyperfine transition of neutral hydrogen. The hyperfine transition occurs via an
interaction between the spins of the electron and proton in an HI atom. The singlet state, when the
spins are symmetric under interchange, has a slightly lower energy level than the triplet state, when
they are anti-symmetric under interchange. This effect is usually summarized by a spin flip in the
electron, though this only describes two of the three triplet states. This transition
yields a photon with a wavelength of 21cm.

The hyperfine transition may be much more useful for detecting HI during the EoR than the previous 
two measurements. As we will show in Section \ref{sec:SpinTemp}, all lines of sight to
neutral hydrogen at high redshift are optically thin, so emission from all times throughout the EoR
is accessible. It also reaches Earth at radio frequencies, so ground-based experiments are
sufficient to measure it. By contrast, redshift 10 Ly$\alpha$ reaches us in the infrared, to
which the atmosphere is opaque. Also, since 21cm emission directly detects HI,
information about the EoR can be directly inferred, with no dependence on a model, as in CMB
analysis.

There are two downsides to using the 21cm line to measure neutral hydrogen. First, the hyperfine
transition is a forbidden transition, and has a mean lifetime of around $10^7$ years. The effect of
this small transition rate is relatively dim emission from neutral hydrogen clouds. Second, the
21cm line from the EoR redshifts into meter wavelengths at which galactic synchrotron emission
dominates all astrophysical radiation. To give an estimate of the relative strengths of these two
processes, the temperature of 21cm EoR emission is around 30 mK, and the temperature of galactic
synchrotron emission is around 1000 K. That foreground emission is $\sim10^5$ times brighter than the
target signal necessitates a highly accurate foreground removal or avoidance scheme. 

Despite the difficulties facing its measurement, the benefits of 21cm tomogrophy seem to outweigh
the challenges. Again, the universe is optically thin to 21cm emission, and detecting HI during the
EoR requires no model. Hence, for the duration of this thesis, we will focus on 21cm emission from 
neutral hydrogen, and inspect is usefulness for detecting the signature of the first stars and galaxies.
