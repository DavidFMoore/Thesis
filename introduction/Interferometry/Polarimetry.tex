\section{Polarimetry}
\label{sec:Polarimetry}

Here, we add the final complication to our model of the visibility (Equation \ref{eq:visdef}), which
will be the subject of this thesis: polarization. To understand the interferometer's 
response to a polarized signal, we need first to lay some groundwork defining our coordinate systems.

As mentioned in the previous section, the $uv$-plane is defined as being
fixed to the earth, but the polarization vector of a source is obviously fixed to the celestial
sphere. To account for this, we usually define the $uv$-plane in what we call the topocentric
coordinate system (East, North, Up), and we define the source position in the equatorial coordinate
system, fixed to the sky, ($(X,Y,Z)$, where $X^2 + Y^2 + Z^2 = 1$). The three-coordinate equatorial
system can be converted into the more familiar right-ascension
$\alpha$\nomenclature[G]{$\alpha$}{Right
ascension} and declination $\delta$\nomenclature[G]{$\delta$}{Declination}
via a similar transformation to convert from Cartesian to spherical coordinates:\footnote{The
  differences are due to the definitions of the two coordinates --- $\delta$ is defined to be zero
  at the equator, rather than the pole ($\delta=\pi/2-\theta$), and $\alpha$ is defined to be left-handed, to track the
earth's rotation, from the point of view of an observer looking up ($\alpha = -\phi$).}
\begin{equation}
  \tan\alpha = \frac{Y}{X}, \quad
  \sin\delta = Z.
\end{equation}

The projection of the topocentric $uv$-plane into equatorial coordinates is the projection of the
$uv$-plane onto the tangent plane of the celestial sphere. Since the hemisphere
available to an observer is dependent on the observer's location on the earth and the local sidereal
time\nomenclature[Zl]{LST}{Local sidereal time} for the observer, the projection matrix $\B{R}$ is a function of location and time. We write
$\B{R}$ in terms of the declination of the pointing $\delta$ and the hour angle of the pointing $H =
LST-\alpha$\nomenclature[Rh]{$H$}{Hour angle} as  
\begin{equation}
  \B{R} = \begin{pmatrix}
    \sin H & -\sin\delta\cos H &  \cos\delta\cos H \\
    \cos H &  \sin\delta\sin H & -\cos\delta\sin H \\
    0      &  \cos\delta       &  \sin\delta
  \end{pmatrix},
  \label{eq:def_top2eq}
\end{equation}
which is defined to act on topocentric vectors (the baseline $\vec{b}$, for instance) on the right as 
\begin{equation}
  \vec{b}_{eq} = \vec{b}_{top}\cdot\B{R},
\end{equation}
where $\vec{b}_{eq}$ is the baseline vector represented in equatorial coordinates, and $\vec{b}$ is
the baseline vector represented in topocentric coordinates. The projection matrix $\B{R}$ is defined to act
on baselines from the right to facilitate interpretation of the product
$\vec{b}\cdot\hat{s}$.\footnote{Am I standing still and the sky rotates around me, or is the sky
fixed and I'm rotating through it? Either interpretation is correct, depending on your choice of
frame of reference. The source vector $\hat{s}$ can be written in the topocentric frame as
$\B{R}\cdot\hat{s}$, so in either frame of reference, the term $\vec{b}\cdot\B{R}\cdot\hat{s}$, being a scalar, is
coordinate-independent, and thus is always constant. This is highly analogous to the Schr\"{o}dinger
and Heisenberg pictures of quantum mechanics.}

Having defined the equatorial and topocentric coordinate systems, and described how a vector from
one projects onto the other, let us now discuss the physical process of a polarization vector
projecting onto an interferometer's dipole. In doing so, we will introduce three concepts: the 
Kronecker product, the Stokes parameters, and parallactic rotation.

The polarization vector of a source is fixed in the celestial sphere, making equatorial coordinates
a natural choice of basis. Since the propagation direction of the $E$-field of any source is
radially inward, toward the observer, right-ascension and declination are better suited to
describe these vectors. An interferometer, and any telescope in general, projects these vectors onto
its own local frame of reference. If we choose topocentric coordinates $(x,y)$, colinear with the
previously defined $(u,v)$,\footnote{Generally, $u$ and $v$ are the Fourier-dual variables to sky
coordinates $l$ and $m$, measured in units of wavelength. $x$ and $y$ are the physical easting and northing,
measured in units of length.} then we can write
this projection as a simple rotation
\begin{equation}
  \begin{pmatrix}E_x\\E_y\end{pmatrix}
  \equiv \B{P}'\cdot\begin{pmatrix}E_\alpha\\E_\delta\end{pmatrix} =
  \begin{pmatrix}\cos\psi&\sin\psi\\-\sin\psi&\cos\psi\end{pmatrix}
  \begin{pmatrix}E_\alpha\\E_\delta\end{pmatrix},
\end{equation}
where $E_{x,y}$ are the components of the electric field in the $xy$ coordinate system,
$E_{\alpha,\delta}$ are the components in the equatorial frame, and
$\psi$\nomenclature[G]{$\psi$}{Parallactic angle} is the parallactic angle,
defined in terms of latitude $\lambda$, hour angle of observation $H$, and the
declination of observation $\delta$ as 
\begin{equation}
  \tan\psi = \frac{\cos\lambda\sin H}{\sin\lambda\cos\delta - \cos\lambda\sin\delta\cos H}
\end{equation}
Figure \ref{fig:parang} shows the parallactic rotation of a plus sign.

\figuremacroW{introduction/Interferometry/figures/ParallacticAngle.eps}{0.7}{fig:parang}{
  Parallactic rotation of a plus sign. Lines of constant declination at $\delta = 30,\ 0,\ -30,\ -60$ and lines of
constant hour angle $H= 0,\ 3h,\ 6h,\ 12h,\ 15h,\ 18h,\ 21h$ are shown with dotted lines. The latitude 
is chosen to be that of the PAPER array ($-30^\circ43'17.5'$). Hour angle $0$ is chosen to be the meridian.
}{Parallactic rotation of a $+$ sign}

The next step in propagating a celestial signal through an interferometer is applying any
instrumental gains to each component of the electric field. In general, this can be represented by a
$2\times2$, complex matrix ($\B{G}$), all of whose elements are non-zero. There are three major components 
to setting the elements of $\B{G}$: the overall amplification of the signal, the electrical
delay of the signal with respect to an array-wide average, and any instrumental polarization terms,
most of which can arise from improperly aligned feeds or inter-signal crosstalk. Each signal
recieves each of these calibration parameters, so there exist $N_{ant}$ copies of the matrix
$\B{G}$, for an array with $N_{ant}$ antennae. Finally, we separate the direction-dependent terms
and call them the primary beam $A(l,m)$. For the purposes of simplicity\footnote{As we 
  will discuss in Chapter \ref{chap:PowerSpectra}, these cross-terms are measured to be negligibly 
small for our purposes.}, we will omit any discussion of cross-polarizing terms and represent the 
matrix for the $i^{\rm th}$ antenna's matrix $\B{G}_i$ in terms of signal gain for each polarization
$\alpha$, $g_i^\alpha$\nomenclature[Rg]{$g$}{Antenna gain} and an electrical delay for each polarization
$\tau_i^\alpha$\nomenclature[G]{$\tau$}{Delay}, and the
$\alpha$ polarization's primary beam:
\begin{equation}
  \B{G}_i = \begin{pmatrix}
    g_i^xA_x(l,m)e^{-2\pi i \nu \tau_i^x} & 0 \\          
    0 & g_i^yA_y(l,m)e^{-2\pi i \nu \tau_i^y}          
  \end{pmatrix}.
  \label{eq:def_gains}
\end{equation}
This operation is applied to the signal at each antenna, making the electric field measured from the
$i^{\rm th}$ antenna at the point of correlation 
\begin{equation}
  \vec{E}_i = \B{G}_i\cdot\B{P}\cdot\vec{E}_{\alpha\delta}. 
\end{equation}

At last, we have discussed the necessary precursors, and can begin correlation! An interferometer whose dipoles are aligned along the $x$- and $y$-axes (equivalently the $u$- and
$v$-axes) correlates each component of the $E$-field with both itself and the other, totaling to
four polarization products. This operation can be represented by the Kronecker outer
product\nomenclature[X]{$\otimes$}{Kronecker outer product} between two
matrices\footnote{There is another convention, defining an outer product in which
$(\B{A}\otimes\B{B}^\dagger)_{ij} = A_iB^*_j$. Using this notation, the native, linear polarization
products are simply an expansion of the Stokes parameters times the Pauli matrices and unity. While
this definition of an outer product leads to mathematically elegant results, and is in theory
equivalent to our choice of an outer product, I personally find the calculations to be quite
cumbersome and will not use it.}, which takes an $m\times n$ matrix $\B{A}$ and a $p\times q$ matrix $\B{B}$ and computes
the $mp\times nq$ matrix
\begin{equation}
  \B{A}\otimes\B{B} = \begin{pmatrix}
      a_{11}\B{B} & \dots & a_{1n}\B{B} \\ 
      \vdots & \ddots & \vdots \\
      a_{1m}\B{B} & \dots & a_{mn}\B{B}
  \end{pmatrix},
\end{equation}
where $a_{ij}$ is the $(i,j)^{{\rm th}}$ element of the matrix $\B{A}$, and each $p\times q$ block
(represented by $\B{B}$) contains the elements of matrix $\B{B}$. As an example, the
Kronecker product of the $E$ field (in the $(x,y)$ representation) is
\begin{equation}
  \vec{E}_{xy}\otimes\vec{E}_{xy}^* = \begin{pmatrix}E_xE_x^*\\E_xE_y^*\\E_yE_x^*\\E_yE_y^*\end{pmatrix},
    \label{eq:example_kronecker}
\end{equation}
which weighted by the primary beam, and integrated over time and space, is a visibility. A useful
property of the Kronecker product is the mixed-product property, which states that
\begin{equation}
  (\B{A}\B{B})\otimes(\B{C}\B{D}) = (\B{A}\otimes\B{C})\cdot(\B{B}\otimes\B{D}).
\end{equation}
So, if we define the $4\times1$ vector of visibilities measured between antennae $i$ and $j$ as 
$\mathcal{V} \equiv \langle
\vec{E}_i\otimes\vec{E}_j^*\rangle$\nomenclature[Rv]{$\mathcal{V}$}{Visibility, calibration terms
included}, then
\begin{align}
  \mathcal{V} &=
  \left\langle(\B{G}_i\B{P}\vec{E}_{\alpha\delta})\otimes(\B{G}_j\B{P}\vec{E}_{\alpha\delta})^*\right\rangle 
  = (\B{G}_i\otimes\B{G}_j^*)(\B{P}\otimes\B{P})\left\langle \vec{E}_{\alpha\delta}\otimes\vec{E}_{\alpha\delta}\right\rangle
  \label{eq:vis_outerproduct}
\end{align}
We can tackle this expression term-by-term, starting on the right. Equation
\ref{eq:example_kronecker} gives an expression for the outer product of the two linear 
components of an electric field, but a much more useful basis can be found. This basis, whose
components are called the Stokes parameters, is defined for linearly polarized components of 
the electric field thus:
\begin{gather}
    I =  |E_\alpha|^2 + |E_\delta|^2 \\
    Q =  |E_\alpha|^2 - |E_\delta|^2 \\
    U = 2\mathfrak{Re}\left\{E_\alpha E_\delta^*\right\} = E_\alpha E_\delta^* + E_\delta E_\alpha^* \\
    V = 2\mathfrak{Im}\left\{E_\alpha E_\delta^*\right\} = -i(E_\alpha E_\delta^* - E_\delta E_\alpha^*).
\end{gather}
This basis conveniently represents the power of the electric fields in terms of its total intensity
($I$), the power contained in each component of a basis of two, orthogonal, linear polarizations ($Q$ and
$U$), and the power contained in circular polarizations ($V$). The rotation can be represented by
the matrix $\dvec{S}$\nomenclature[Rs]{$\dvec{S}$}{Stokes rotation matrix}, defined as
\begin{equation}
  \begin{pmatrix}I\\Q\\U\\V\end{pmatrix} =
    \begin{pmatrix}
      1 &  0 & 0 &  1 \\
      1 &  0 & 0 & -1 \\
      0 &  1 & 1 &  0 \\
      0 & -i & i &  0
    \end{pmatrix}
  \begin{pmatrix}|E_\alpha|^2\\E_\alpha E_\delta^*\\E_\delta E_\alpha^*\\|E_\delta|^2\end{pmatrix} =
   \dvec{S}
  \begin{pmatrix}|E_\alpha|^2\\E_\alpha E_\delta^*\\E_\delta E_\alpha^*\\|E_\delta|^2\end{pmatrix} 
  \label{eq:stokes_def}
\end{equation}
For completeness, we present the inverse of $\dvec{S}$, and note that were it not for the
normalization which requires $I$ to contain the \emph{total} intensity of the electric field, $\dvec{S}$ 
would be Hermitian:
\begin{equation}
  \dvec{S}^{-1} = \frac{1}{2}\begin{pmatrix}
    1 &  1 & 0 &  0 \\
    0 &  0 & 1 &  i \\
    0 &  0 & 1 & -i \\
    1 & -1 & 0 &  0
  \end{pmatrix}.
\end{equation}
This definition allows us to write Equation \ref{eq:vis_outerproduct} in terms of the Stokes
parameters:
\begin{equation}
  \mathcal{V} =
  (\B{G}_i\otimes\B{G}_j^*)
  (\B{P}\otimes\B{P})\dvec{S}^{-1}\left\langle\begin{pmatrix}I\\Q\\U\\V\end{pmatrix}\right\rangle,
\end{equation}
which allows us to write the parallactic rotation of the Stokes parameters in a convenient form: 

\begin{equation}
  \dvec{P} = (\B{P}\otimes\B{P})\dvec{S}^{-1} 
  =
  \frac{1}{2}\begin{pmatrix}
    1 &  \cos2\psi &  \sin2\psi &  0 \\
    0 & -\sin2\psi &  \cos2\psi &  i \\
    0 & -\sin2\psi &  \cos2\psi & -i \\
    1 & -\cos2\psi & -\sin2\psi &  0 
  \end{pmatrix},
\end{equation}
which looks confusing until we make one observation. The Stokes parameters defined on the celestial
sphere (in the $\alpha,\delta$ basis) are different from those observed (in the $x,y$ basis). If we
define topocentric, observed Stokes parameters, ($I' = |E_x|^2 + |E_y|^2$, etc.), we find that observed 
$Q'$ and $U'$ are rotated from the celestial $Q$ and $U$ by an angle $2\psi$, while $I$ and $V$
remain fixed. We can represent this transformation as 
\begin{align}
\begin{pmatrix}I\\Q\\U\\V\end{pmatrix}
  = \dvec{S}(\B{P}^{-1}\otimes\B{P}^{-1})\dvec{S}^{-1}
\begin{pmatrix}I'\\Q'\\U'\\V'\end{pmatrix}
=  \begin{pmatrix}
    1 & 0 & 0 & 0 \\
    0 & \cos2\psi & -\sin2\psi & 0 \\
    0 & \sin2\psi & \cos2\psi & 0 \\
    0 & 0 & 0 & 1
  \end{pmatrix}
\begin{pmatrix}I'\\Q'\\U'\\V'\end{pmatrix},
\end{align}
which aligns with our intuition that $Q$ and $U$ rotate between frames, but $I$ and $V$ remain
constant. 

Finally, we can write the parallactic rotation from the point of view from the feeds as 
\begin{equation}
  \dvec{P}' =
  (\B{P}\otimes\B{P})\dvec{S}^{-1}\dvec{S}(\B{P}^{-1}\otimes\B{P}^{-1})\dvec{S}^{-1} =
  \dvec{S}^{-1},
\end{equation}
where we have taken advantage of another property of the Kronecker outer product, that
$\B{A}^{-1}\otimes\B{B}^{-1} = (\B{A}\otimes\B{B})^{-1}$. A curious reader at this point will ask
why we bothered defining the visibilities in terms of the equatorially defined Stokes parameters --
since the interferometer measures in topocentrically defined coordinates, why not use those? The
answer is because of the time-dependence of parallactic rotation. A visibility measured at some time
with some pointing is not equal to the same visibility with the same pointing at a later time. We've 
written the matrices and carefully defined each rotation and projection to define visibilities in a
fixed coordinate system, and furthermore, to develop intuition about the rotations between Stokes
parameters and their projection onto the $xy$ plane.

We can use Figure \ref{fig:parang} to visualize the parallactic rotation of $Q$ and $U$, that is,
the rotation from $Q',U'$ to $Q,U$. In this figure, all symbols represent $Q$, since they align with
lines of constant right ascension and declination. However, as the $+$ sign ``rises'' and ``sets,'' increasing or
decreasing in hour angle, it rotates into $\times$ (and back again). The $+$ and $\times$ in Figure \ref{fig:parang}
represent $Q'$ and $U'$, respectively.

To complete this brief discussion of interferometric polarimetry, we present the full equations for
a visibility measured using linear feeds:
\begin{equation}
  \mathcal{V} = \left\langle
  (\B{G}_i\B{P}\vec{E}_{\alpha\delta})
  \otimes
  (\B{G}_j\B{P}\vec{E}_{\alpha\delta})^*
  \right\rangle,
\end{equation}
or, writing each component explicitly in terms of topocentric Stokes parameters:
\begin{gather}
  \mathcal{V}_{xx} = \int A_{xx}(\hat{s})g_i^xg_j^{x*} e^{-2\pi i \nu(\tau_i^x-\tau_j^x)}
  \left[I'(\hat{s}) + Q'(\hat{s})\right]e^{-2\pi i \nu (\vec{b}/c)\cdot\hat{s}}\D{\Omega}
  \label{eq:def_Vxx}
  \\
  \mathcal{V}_{xy} = \int A_{xy}(\hat{s})g_i^xg_j^{y*} e^{-2\pi i \nu(\tau_i^x-\tau_j^y)}
  \left[U'(\hat{s}) + iV'(\hat{s})\right]e^{-2\pi i \nu (\vec{b}/c)\cdot\hat{s}}\D{\Omega}
  \label{eq:def_Vxy}
  \\
  \mathcal{V}_{yx} = \int A_{yx}(\hat{s})g_i^yg_j^{x*} e^{-2\pi i \nu(\tau_i^y-\tau_j^x)}
  \left[U'(\hat{s}) - iV'(\hat{s})\right]e^{-2\pi i \nu (\vec{b}/c)\cdot\hat{s}}\D{\Omega}
  \label{eq:def_Vyx}
  \\
  \mathcal{V}_{yy} = \int A_{yy}(\hat{s})g_i^yg_j^{y*} e^{-2\pi i \nu(\tau_i^y-\tau_j^y)}
  \left[I'(\hat{s}) - Q'(\hat{s})\right]e^{-2\pi i \nu (\vec{b}/c)\cdot\hat{s}}\D{\Omega}
  \label{eq:def_Vyy}
\end{gather}

It is often convenient to rotate the linearly-polarized visibilities as the linearly-polarized
images are rotated into Stokes parameters. This does not exactly represent the true
Fourier-transformed Stokes parameters, but it does provide a useful approximation. Section
\ref{sec:BeamLeakage} will discuss one of the negative consequences of such a rotation. For
completeness, we provide the definition of Stokes visibilities:
\begin{equation}
  \begin{pmatrix}
    \mathcal{V}_I \\
    \mathcal{V}_Q \\
    \mathcal{V}_U \\
    \mathcal{V}_V
  \end{pmatrix} = \begin{pmatrix}
    1 &  0 & 0 &  1 \\
    1 &  0 & 0 & -1 \\
    0 &  1 & 1 &  0 \\
    0 & -i & i &  0
  \end{pmatrix}\begin{pmatrix}
    \mathcal{V}_{xx} \\
    \mathcal{V}_{xy} \\
    \mathcal{V}_{yx} \\
    \mathcal{V}_{yy}
  \end{pmatrix}
\label{eq:def_stokes_vis}
\end{equation}
