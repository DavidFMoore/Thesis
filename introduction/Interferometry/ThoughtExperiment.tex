\section{A Thought Experiment}\label{sec:interferometry}

In this section, we will introduce interferometry, and derive an expression for the visibility
from first principles. While this material is used commonly enough and predates many
references typically given, it is nonetheless useful to provide a reference for a more thorough
discussion of this material, which serve as the sources for it. A more thorough and complete
discussion of interferometry can be found in two books: \citet{WhiteBook} and \citet{TMS}.

Suppose we have two telescopes --- label them $A$ and $B$ --- which are separated by some
baseline vector $\vec{b}$\nomenclature[Rb]{$\vec{b}$}{Baseline (in meters)}. Next, suppose that a single
plane wave at frequency $\nu$\nomenclature[G]{$\nu$}{Frequency} is indendent on
these telescopes $A$ and $B$, with a propagation direction
$-\hat{s}$\nomenclature[Rs]{$\hat{s}$}{Unit
pointing vector}. For now, let's assume that
these two telescopes are evenly sensitive to radiation coming from all directions --- we can add that
complication later. We will also neglect projection effects of the electric field vector on
telescopes $A$ and $B$ until later --- we're only concerned with the phase information for now.

We define the phase of the electric field so that at telescope $A$, it takes a value 
\begin{equation}
  E_A = E_0\exp\left\{-2\pi i\nu t\right\}, 
\end{equation}
which sets the value of the electric field at telescope $B$ to be
\begin{equation}
  E_B = E_0\exp\left\{-2\pi i\nu [t + (\vec{b}/c)\cdot\hat{s}]\right\}.
\end{equation}
The extra term in the phase, $(\vec{b}/c)\cdot\hat{s}$, represents the time-difference of a
wavefront's arrival on telescope $A$ and $B$, and is thus defined as the delay. If we correlate the
signals from the two telescopes in time, we extract the delay thus:
\begin{align}
  \left\langle E_A E_B^* \right\rangle_t &= 
    \lim_{T\to\infty} \frac{1}{2T}\int^T_{-T} E_A(t)E_B(t)\ {\rm d}t \\ 
    &= |E_0|^2\exp\left\{-2\pi i \nu (\vec{b}/c)\cdot\hat{s}\right\}.
\end{align}

The next stage of complication is to allow plane waves to come from all directions. In order to
represent the response of telescopes $A$ and $B$ to the entire celestial sphere, we integrate
over the sphere, and allow $|E_0|^2 \propto I$, the intensity of the incident emission, to vary as a
function of direction:
\begin{equation}
  \left\langle E_A E_B^*\right\rangle_t
  = \int I(\hat{s})\exp\left\{-2\pi i\nu(\vec{b}/c)\cdot\hat{s}\right\}\ {\rm d}\Omega.
  \label{eq:visdef_nobeam}
\end{equation}
We can leave the details of the projection of the sphere onto our antennae for later, but this
equation reveals the two fundamental aspects of interferometry:
\begin{enumerate}
  \item There is a Fourier relationship between the intensity of celestial emission, $I(\hat{s})$,
    and the inteferometric response $\langle E_A E_B^*\rangle_t$.
  \item The dual variable to sky position $\hat{s}$ (in the Fourier sense) is the baseline vector
    between two antennae, measured in units of wavelength.
\end{enumerate}

Before attempting to create an image from the interferometer's response, we will add two final
complications. First, we define a coordinate system. The antenna locations are defined in a
topocentric coordinate system $(u,v,w)$, with $u$ the local easting, $v$ is the local northing, 
and $w$ is pointed towards zenith. This coordinate system is fixed to the earth at the location of
the observer. The upper half of the celestial sphere is characterized by the
coordinates $(l,m)$, where at zenith, $l$ points in the direction of $u$. The relevant terms of
equation \ref{eq:visdef_nobeam} are the measure, 
\begin{equation}
  {\rm d}\Omega = \frac{{\rm d}l{\rm d}m}{\sqrt{1-l^2-m^2}}, 
\end{equation}
and the delay,
\begin{equation}
  (\vec{b}/\lambda)\cdot\hat{s} = ul + vm + w(1-\sqrt{1-l^2-m^2}). 
\end{equation}
It is typical to assume that the array of antennae is coplanar, so we can set $w=0$, and define
baselines as being in the $uv$-plane. It is also typical to use the flat-sky approximation for these
coordinates, setting the measure ${\rm d}\Omega \simeq \D{l}\D{m}$, but we will forgo this
approximation.

The penultimate complication we will add (the final complication, polarization, gets its own
section) is the spatial response pattern of an antenna, $A(l,m)$\nomenclature[Ra]{$A(l,m)$}{Primary
beam}, called the primary beam. The
primary beam attenuates the signal received by the electric field, modifying $E_i \to A_iE_i$.
We assume that all antennae are identical, so we can combine the product $A_iA_j$ into a single
primary beam $A$, which attenuates the intensity of incident radiation, rather than the electric
field.

Finally, we can name the interferometer's response, and define the visibility
\nomenclature[Rv]{$V(u,v,\nu)$}{Visibility}(a word studiously
avoided until now) as
\begin{equation}
  V(u,v,\nu) = \int A(l,m)I(l,m)e^{-2\pi i (ul + vm)}\frac{\D{l}\D{m}}{\sqrt{1-l^2-m^2}}.
  \label{eq:visdef}
\end{equation}

Images may be recreated from the visibilities by an inverse Fourier transforms over all visibilties
in an array:
\begin{equation}
  \frac{A(l,m)\widetilde{I}(l,m)}{\sqrt{1-l^2-m^2}} 
  = \int \Sha(u,v)V(u,v)e^{+2\pi i\nu(ul+vm)}\D{u}\D{v}.
  \label{eq:sha_def}
\end{equation}
A couple of items to note in Equation \ref{eq:sha_def}\nomenclature[Gz]{$\Sha(u,v)$}{Sampling function in the
$uv$-plane}: first, the reconstructed image
$\widetilde{I}$ is attenuated by the beam-response pattern and the measure\footnote{Typically, 
the factor of $(1-l^2-m^2)^{-1/2}$ is absorbed into the beam response $A(l,m)$. Henceforth, we will
follow this convention.} Second (and more importantly) the sampling function $\Sha(u,v)$, which is
defined as being 1 in the $uv$ points that an array samples and 0 elsewhere, prohibits the full
Fourier spectrum from being included in the reconstructed image. The effect is that the true image
$I$ is convolved with what is defined as the dirty beam, the Fourier-transform of the sampling
function. Because $\Sha(u,v)$ contains zeros, a complete deconvolution of the reconstructed image
from the dirty beam is impossible.


