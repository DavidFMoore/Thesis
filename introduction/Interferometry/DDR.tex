\section{The Time and Frequency Dependence of Visibilities}
\label{sec:DDR}

In this section, we go through a thorough investigation of the $\vec{b}\cdot\hat{s}$ term in
Equation \ref{eq:visdef}. The discussion here will largely follow Parsons and Backer 2009 \cite{Parsons2009} 
and Appendix $A$ of Parsons, et al. 2014 \cite{Parsons2014}. It will also serve as a precursor to Section
\ref{sec:DelaySpectrum} and be used as a reference for discussing the compression
techniques in Section \ref{sec:data}.

We begin by expanding $(\vec{b}/c)\cdot\hat{s}$ in the bases defined in Section
\ref{sec:Polarimetry}, using the equatorial to topocentric rotation matrix $\B{R}$. We note that in
topocentric coordinates, $\hat{s} = \begin{pmatrix}0&0&1\end{pmatrix}^T$, since the $uv$-plane is
defined to be tangent to the celestial sphere at $\hat{s}$. Written in this basis,
\begin{gather}
\vec{b}\cdot\hat{s} 
  = \begin{pmatrix}b_x&b_y&b_z\end{pmatrix}
    \begin{pmatrix}
      \sin H & -\sin\delta\cos H &  \cos\delta\cos H \\
      \cos H &  \sin\delta\sin H & -\cos\delta\sin H \\
      0      &  \cos\delta       &  \sin\delta
    \end{pmatrix}
  \begin{pmatrix}0\\0\\1\end{pmatrix}
   \nonumber \\ = 
 b_x\cos\delta\cos H -b_y\cos\delta\sin H +b_z\sin\delta,
\end{gather}
where $H$ is the hour angle of the pointing, and $\delta$ is the declination. We notice that the
phase of a visibility is $2\pi i \nu(\vec{b}/c)\cdot\hat{s}$, and define a time-constant of the
visibility, $\tau_g =
(\vec{b}/c)\cdot\hat{s}$\nomenclature[G]{$\tau_g$}{Geometric delay}, called the geometric delay. This is simply the time
delay of the signal between the two antennae in a basline. This argument indicates that the
geometric delay is bound between the positive and negative baseline lengths, $-|\vec{b}| \le \tau_g
\le |\vec{b}|$, whose values are realized when $\hat{s}$ aligns with the basline vector, i.e. the
interferometer is pointing along the bore sight of its baseline.

Neglecting calibration terms, we may write the visibility (Equation \ref{eq:visdef}) in terms of
$\tau_g$:
\begin{equation}
   V = \int A(\hat{s},\nu)I(\hat{s},\nu)e^{-2\pi i \nu\tau_g}\D{\Omega}.
  \label{eq:vis_taug}  
\end{equation}
Noting the similariry of Equation \ref{eq:vis_taug} to a Fourier transform with respect to $\nu$, we
define the delay transform as the inverse Fourier transform of a visibility with respect to
frequency\nomenclature[Rv]{$\widetilde{V}(\tau)$}{Delay-transformed visibility}:
\begin{equation}
  \widetilde{V}(\tau) = \int V e^{+2\pi i\nu\tau}\D{\nu}
    = \int A(\hat{s},\nu)I(\hat{s},\nu)e^{-2\pi i \nu(\tau_g-\tau)}\D{\Omega}\D{\nu},
  \label{eq:def_DelayTransform}
\end{equation}
the convolution of the beam-weighted image with a delta-function kernel, peaked at $\tau_g$. Written
more explicitly in terms of the convolution operation $\star$, we have 
\begin{equation}
  \widetilde{V} = \widetilde{A}(\tau) \star \widetilde{I}(\tau) \star \delta\left(\tau-\tau_g\right).
\end{equation}
To build intuition about this tranform, we can neglect the primary beam $A$, and assume that the
only incident radiation is due to a flat-spectrum point source, i.e. 
$I(\hat{s},\nu) = I_o\delta(\hat{s}-\hat{s}_0).$ This assumption reduces Equation
\ref{eq:def_DelayTransform} simply to $\widetilde{V}(\tau) = I_0\delta(\tau-\tau_g(\hat{s}_0))$, exhibiting
the important property of the delay spectrum, that it isolates smooth-spectrum, point sources in a
space accessible to all baslines individually. The isolation of smooth-spectrum point soures is
demonstrated in Figure \ref{fig:src_delay} --- this figure also demonstrates the restriction of
smooth-spectrum emission to within the horizon limit of a baseline.

\figuremacroW{delay_wfall.eps}{1.0}{fig:src_delay}{
  Delay spectra of baslines of different lengths of five simulated sources. Four of these sources
  have smooth, power law spectra, and one (which ``turns on'' at 0.8 days) with a non-smooth
  spectrum. Horizon limits for the four baslines are shown by black, dashed lines --- these
  correspond to the baseline lengths of 32 m (top left), 64m (top right), 128 m (bottom left), and
  256 m (bottom right). Visibilities are computed over the PAPER band, spanning frequencies of 100
  MHz to 200 MHz. Color scale denotes the flux, with red showing the brightest sources
  and blue showing the dimmest, but the absolute scaling of the flux scale is arbitrary. Figure credit: \citet{DelaySpectrum}.
}{Delay spectra of sources for several baseline lengths, from \citet{DelaySpectrum}}

Once we understand that the geometric delay of a source changes with time,\footnote{Many sources
(\cite{TMS}, e.g.) call this the fringe. We will accept this nomenclature, reserving the term
``delay rate" for the time-dependence of the geometric delay, rather than the visibility. Hence, a
fringe-rate transform will be the Fourier transform w.r.t. time of a visibility, but the delay-rate
transform will be that of a delay-transformed visiblity. The difference is subtle, and won't appear
in this work, but nonetheless, it is great enough to warrant two different names for the two
different transforms.} we can apply a similar technique in the time-domain. We
define the fringe-rate transform of a visibility thus\nomenclature[Rv]{$\widehat{V}(\nu,f)$}{Fringe-rate
transformed visibility}:
\begin{equation}
  \widehat{V}(\nu,f) = \int V(\nu,t) e^{-2\pi i ft}\D{t}
  \label{eq:def_DR}
\end{equation}
Assuming again that we may neglect calibration terms and may only focus on a single,
smooth-spectrum, point source, we find the fringe-rate transform isolates sources in the similar manner
as a delay transform:
\begin{align}
  \widehat{V}(\nu,f) &= 
  \int \left[\int I_0\delta(\hat{s}-\hat{s}_0)e^{-2\pi i \tau_g\nu}\D{\Omega}\right]e^{-2\pi i ft}\D{t}
  \nonumber \\
  &\propto \int e^{-2\pi i(\tau_g\nu + ft)}\D{t}
  \approx \delta\left(f+\dfdx{\tau_g}{t}\nu\right),
  \label{eq:prep_fringe_rate}
\end{align}
where $\tau_g$ is evaluated at $\hat{s}_0$ after the integration with respect to the sky coordinates
is performed. This allows us to immediately read off the fringe-rate of a source,
\begin{equation}
  f_0(\hat{s}) = \nu\dfdx{\tau_g}{t} = 
  -\omega_\oplus\cos\delta\left((\nu/c)b_x\cos H + (\nu/c)b_y\sin H\right),
  \label{eq:def_fringe_rate}
\end{equation}
where $\omega_\oplus = -dH/dt$\nomenclature[G]{$\omega_\oplus$}{Angular frequency of the Earth's
rotation}, the angular frequency of the earth's rotation.

Notice that $b_x\cos H + b_y \sin H$ is simply the east-west portion of the
basline (in topocentric coordinates), and the linear velocity of the earth's rotation at latitude
$\delta$ is proportional to $\omega_\oplus\cos\delta$ --- the fringe-rate is simply the dot product
of the basline with the angular velocity of the earth!

To complete the analogy with the delay transform, we note that the fringe rate of a source is
limited to $-(\nu/c)b_{E}\cos\delta \le f_0/\omega_\oplus \le (\nu/c)b_E$, where $b_E$ is the
topocentric, east-west component of the basline. 

\figuremacroW{introduction/Interferometry/figures/DelayDelayRate.eps}{0.8}{fig:ddr}{
  (Left Panel) Delay, in nanoseconds, of a 32m (106ns), east-west baseline at the latitude $-31^\circ43'17.5''$. The
  delay ranges from $\pm|\vec{b}|$ and is constant along the north-south axis. (Right panel) Fringe
  rate of that same baseline, in mHz.
}{Delay and delay rate}
\figuremacroW{DDR_limits.eps}{1.0}{fig:ddr_limits}{
  (Left Panel) Delay spectrum of one day's worth of PAPER data vs. time. A solid black lines showing the
  horizon limit of the 30 m baseline used to take this data. (Right Panel) Delay rate spectrum of
  the same data vs. frequency. Again, black lines denote the horizon limits of the fringe rate. In
  both plot, the colorscale denotes flux, with red being the brightest, but the absolute flux scale
  is arbitrary. 
}{Limits of delay and delay rate from PAPER data}
Figures \ref{fig:ddr} and \ref{fig:ddr_limits} show the limits of the delay and fringe rate
transform. Figure \ref{fig:ddr} shows the mapping of delay and fringe rate onto the sky, and Figure
\ref{fig:ddr_limits} shows evidence for the horizon limits in PAPER data.
