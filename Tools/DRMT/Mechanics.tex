\section{The Mechanics of the DRMT}\label{sec:mechanics}

Now, we will discuss the computational details of estimating the integral \ref{eq:def_dtrmt}. This
discussion builds upon and is intended to complement the work done in
BdB, in
which rotation measure synthesis is described, and a prescription for its computation is given. We
will review the discussion of BdB and add a discussion of the effects of
discretizing the RMT, creating (in an analogy to the DFT) a discrete
rotation measure transform (DRMT)\nomenclature[Zd]{DRMT}{Discrete Rotation Measure Transform}

We begin the prescription for a DRMT by assigning $\lambda^2$ values to each frequency bin of data.
Following BdB, we estimate the average value of $\lambda^2$ in the $i^{th}$ frequency bin ($\nu_i$)
in a spectrum with evenly spaced frequencies, whose spacing is $\Delta\nu$. We define the discrete
values of the squared wavelength $\lambda^2_i$ as
\begin{equation}
  \lambda_i^2 = 
  \frac{1}{\Delta\nu}\int_{\nu_i-\Delta\nu/2}^{\nu_i+\Delta\nu/2} \frac{c^2}{\nu^2}\D{\nu}
  = \frac{c^2}{\nu_i\Delta\nu}\left[
    \left(1 - \frac{1}{2}\frac{\Delta\nu}{\nu_i}\right)^{-1} 
  - \left(1 + \frac{1}{2}\frac{\Delta\nu}{\nu_i}\right)^{-1} \right]
  \approx
  \left(\frac{c}{\nu_i}\right)^2
  \label{eq:def_lambda2}
\end{equation}
where $c$ is the speed of light. To approximate the band-averaged value of $\lambda^2$, we have used 
the assumption that $\Delta\nu \ll \nu_i$.\footnote{That this is different than the value cited in
BdB, $\lambda^2_i = (c/\nu)^2(1 + 3\Delta\nu/4\nu_i)$, due to our only approximating to first order
in $\Delta\nu/\nu$. I personally haven't found any justification for going to a higher order.} Since frequencies are spaced evenly, $\lambda_i^2$ cannot
be; ergo, there is no set of rotation measures $\Phi_j$ which allow the DRMT to evenly sample the
unit circle, a crucial property of the choice of frequencies in the DFT. Formally, there are no
solutions to the equation 
\begin{equation}
  2(\lambda_i^2 - \lambda_0^2)\Phi_j = 2\pi\frac{ij}{N}
  \label{eq:choose_Phiij}
\end{equation}
which are independent of $i$.\footnote{As a reminder, the frequencies of a DFT are chosen such that
  $2\pi(t_i-t_0)\nu_j = 2\pi(ij/N)$. This allows for an even sampling of the unit circle on
  $[-\pi,\pi)$ and allows for a complete basis of frequencies.} 
Here $i$ runs on the interval $[0,N)$ and $j\in[-N/2,N/2)$. We will discuss the implications of the
lack of solutions to Equation \ref{eq:choose_Phiij} throughout this section, but the first and
most obvious of which prohibits a straightforward set of rotation measures to sample.

To find an optimal set of rotation measures to sample, we must first approximate the spacing between
adjacent frequency bins and then choose a set of rotation measures accordingly. To do this, we
simply Taylor expand the spacing in $\lambda^2$ as 
\begin{equation}
  \Delta\lambda_i^2 \approx \Delta\nu\left|\pdfdx{\lambda^2}{\nu}\right|_{\nu=\nu_i}
  = 2\left(\frac{c}{\nu_i}\right)^2\left(\frac{\Delta\nu}{\nu_i}\right),
  \label{eq:def_dlam2}
\end{equation}
and the approximating Equation \ref{eq:choose_Phiij} for appropriate values of $\Phi$.

\figuremacroW{oversampled_kernel}{0.6}{fig:bad_kernel}{
  Magnitude of the kernel of the DRMT $K(\Phi)$, sampled to a maximum value
  $\Phi_{max}=(2\Delta\lambda^2_{min})^{-1}$. Vertical lines correspond to the harmonics of
  $\Phi_{min} = (2\Delta\lambda^2_{max})^{-1}$, with the dashed vertical line corresponding to
  $\Phi_{min}$ and the dotted lines corresponding to higher harmonics. This plot demonstrates the
  pitfalls of choosing $\Delta\Phi$ too large, since it shows the kernel structure induced from
  oversampling the lowest frequencies in a band.
}{Oversampled kernel of DRMT}

Next, we choose spacings in $\Delta\Phi$ to fully sample the spectrum in rotation measure. Setting
$\Delta\Phi < 1/2N\Delta\lambda^2_{min}$ prohibits complete reconstruction (i.e. aliasing) of the 
high-frequency channels, since this choice of $\Delta\Phi$ is less than the maximum Nyquist frequency
of the spectrum. On the other hand, setting $\Delta\Phi > 1/2N\Delta\lambda_{max}^2$ over-samples the
$\Phi$ spectrum at the lowest frequencies, adding structure to the kernel of the DRMT at the
harmonics of $1/2N\Delta\lambda_i^2$, demonstrated in Figure \ref{fig:bad_kernel}.

\figuremacroW{sampling_cartoon}{0.6}{fig:sampling_cartoon}{
  A cartoon to demonstrate the range of acceptable choices for $\Delta\Phi$. The region on the
  $\Delta\Phi$-axis hashed to the upper-right represents the range $\Delta\Phi <
  (2N\Delta\lambda^2_{min})^{-1}$ which under-samples the high frequencies. The region hashed to the
  lower-right depicts the region $\Delta\Phi > (2N\Delta\lambda_{max}^2)^{-1}$, which over-samples
  the high frequencies.
}{Cartoon representation of arguments for sampling $\Delta\Phi$}

Figure \ref{fig:sampling_cartoon} gives a graphic depiction of the argment for choosing
$\Delta\Phi$. We avoid the doubly-excluded region $(2N\Delta\lambda^2_{max})^{-1} < \Delta\Phi <
(2N\Delta\lambda^2_{min})^{-1}$, and choose to undersample rather than to introduce unwanted
structure. Finally, we choose a sampling in rotation measure
\begin{equation}
  \Delta\Phi = \frac{1}{N\Delta\lambda^2_{max}}.
  \label{eq:def_dphi}
\end{equation}

We use this choice of spacing in rotation measure to construct the rotation measure spectrum which
we sample:
\begin{equation}
  \Phi_j = j\Delta\Phi,
\end{equation}
where the index $j\in[N/2, N/2)$ mirrors that of the DFT.

Finally, we set the phase-offset
\begin{equation}
  \lambda_0^2 \equiv \frac{1}{N}\sum_i\lambda_i^2
\end{equation}
to reduce the phase error between the $i=0$ and $i=N$ frequency bins.

Now, we can approximate the integral in Equation \ref{eq:rmt_continuum} as a matrix equation:
\begin{align}
  \widetilde{F}_j &\approx \frac{1}{\sum_i \Delta\lambda^2_i}
  \sum_i F_ie^{-2i\Phi_j(\lambda_i-\lambda_0^2)}\Delta\lambda_i^2
  \nonumber \\ &\equiv
  \frac{1}{N}\sum_iF_ie^{-2i\Phi_j(\lambda_i^2-\lambda_0^2)}\omega_i,
  \label{eq:def_DRMT}
\end{align}
where $\widetilde{F}_j$ is the DRMT of $F$ at $\Phi_j$, $F_i$ are the spectrum values, and
$\lambda^2_i$ are the bin-averaged values of the square wavelength (Equation \ref{eq:def_lambda2}).
Equation \ref{eq:def_DRMT} also defines the weights
\begin{equation}
  \omega_i = \frac{N\Delta\lambda_i^2}{\sum_i\Delta\lambda^2_i}
  \label{eq:def_omega}
\end{equation}
which are the ratios of the bin-widths in $\lambda^2$ to the total bandwidth of the spectrum. We
notat that this is a particular case of Equation \ref{eq:def_dtrmt} whose weights are chosen to most
accurately approximate the measure of the rotation measure transform (Equation \ref{eq:rmt_continuum}).

The matrix $\B{W}$, defined as
\begin{equation}
  W_{ij} = e^{-2i\Phi_j(\lambda_i^2-\lambda_0^2)}\omega_i
  \label{eq:w_ij}
\end{equation}
allows us to write Equation \ref{eq:def_DRMT} in matrix form as $\widetilde{\B{F}} =
\B{W}\cdot\B{F}$. This highlights the computational benefit that we need only calculate the matrix
$\B{W}$ one for each class of spectra $\B{F}$.

\subsection{Example Spectra}

\figuremacroW{5sources}{0.6}{fig:example_drmt}{
  (Top Panel) Simulated spectrum with five ``sources" with randomly chosen fluxes and rotation
  measures. Each channel is injected with 10Jy noise. (Bottom Panel) The magnitude of the DRMT of
  the spectrum in the top panel in black, with red triangles corresponding to the simulated $\Phi$ /
  flux pairs overlaid. The range of $\Phi$ values shown is restricted to $|\Phi| \le 50\ {\rm
  m}^{-2}$ to better show the kernel of the DRMT.
}{Example rotation measure spectrum calculated with the DRMT}
As a proof of concept, we simulate a mock spectrum containing five ``point sources,'' each with a
randomly selected flux and rotation measure, shown in the top panel of Figure
\ref{fig:example_drmt}. The simulated spectrum $S_i$ is calculated as
\begin{equation}
  S_i = \sum_{j=1}^5 A_je^{-2i\Phi_j\lambda_i^2} + n_i,
\end{equation}
where $A_j$ are the simulated fluxes, drawn from a chi-squared distribution with two degrees of
freedom, $\Phi_j$ are the simulated rotation measures, drawn from a normal distribution whose mean
is zero and width is $25\ {\rm m}^{-2}$, $n_i$ is the injected noise, drawn from a mean-zero normal
distribution with a width of 10 Jy, and the spectrum is calculated on frequencies $\nu_i$ (and their
corresponding square wavelengths $\lambda_i^2$), which are evenly spaced on 0.1 to 0.2 GHz, with
1000 channels.

Figure \ref{fig:example_drmt} displays excellent isolation of each $\Phi$ component and demonstrates
the recovery of an input model. Normally, a physical source will not exhibit such a complex rotation
measure structure, but this simulation is designed to highlight the strengths of the DRMT, rather
than be physically representative.

\subsection{Inverse Transform}

In general, the transformation matrix $\B{W}$ (Equation \ref{eq:w_ij}) is singular, prohibiting an
exact expression for the inverse DRMT. Hence, we must supply an approximation for the inverse
rotation measure transform,
\begin{equation}
  F(\lambda^2) = \frac{1}{\pi}\int \widetilde{F}(\Phi)e^{+2i\Phi\lambda^2}\D{\Phi}
  \label{eq:def_iRMT}
\end{equation}

which should follow the same conventions as the rotation measure transform. The matrix
\begin{equation}
  \widetilde{\B{W}}_{ij} = \frac{1}{\pi}e^{+2i\Phi_i\lambda_j^2}\Delta\Phi
  = 
  \frac{1}{\pi}\frac{\sum_k\Delta\lambda_k^2}{N\Delta\lambda^2_{max}}e^{+2i\Phi_i(\lambda_j^2-\lambda_0^2)},
  \label{eq:w_ij-1}
\end{equation}
defined such that $\B{F} \approx \widetilde{\B{W}}\cdot\widetilde{\B{F}}$ is an approximation to
Equation \ref{eq:def_iRMT}. A direct consequence of the singularity of $\B{W}$ is that the product 
\begin{equation}
  \sum_j W_{ij}\widetilde{W}_{jk} \approx \frac{\omega_i}{\omega_{max}}\delta_{ik}
\end{equation}
is not the identity matrix, but converges to the identity matrix in the limit where the spacing in $\lambda^2$
becomes even. In this special case, the DRMT reduces to the DFT, as expected. We also note that in
this special case, the DRMT is unitary.

\subsection{Noise Characteristics}

Parseval's theorem states that power must be conserved between a function and its Fourier transform
--- the rotation measure transform is no different. Rewritten to adhere to the conventions of
rotation measure synthesis, Parseval's theorem requires 
\begin{equation}
  \int|F(\lambda^2)|^2\D{\lambda^2} = \frac{1}{\pi}\int|\widetilde{F}(\Phi)|^2\D{\Phi},
  \label{eq:parseval}
\end{equation}
where the factor of $\pi^{-1}$ accommodates the abnormal Fourier convention of rotation measure
synthesis. With the substitutions $\lambda^2\to x$ and $\Phi\to\pi k$, Equation \ref{eq:parseval}
reduces to the conventional notation for a Fourier transform defined as $\tilde{f}(k) = \int
f(x)\exp\{-2\pi ikx\}\D{x}$.

Because the transfer matrix for the DRMT in Equation \ref{eq:w_ij} is non-unitary ($\B{W}^\dagger
\ne \B{W}^{-1}$), the DRMT will not satisfy Parseval's theorem exactly. However, just as we
approximate $\B{W}^{-1}$ in Equation \ref{eq:w_ij-1}, we can approximate the noise characteristics
of the DRMT by enforcing Parseval's theorem in the limit where the $\lambda_i^2$ are uniformly
spaced.

We begin our approximation of Parseval's theorem by discretizing the right-hand side of Equation
\ref{eq:parseval}, and expanding $\widetilde{F}$ in terms of the DRMT of the original spectrum:
\begin{equation}
  \sum_i|\widetilde{F}_i|^2 = N \sum_i \left|F_i\right|^2\omega_i^2.
  \label{eq:parseval_discrete}
\end{equation}
In the limit $\omega_i\to1$, this expression behaves as expected and gives the usual statement of
Parseval's theorem for a DFT. Otherwise, we can approximate Equation \ref{eq:parseval_discrete} by
providing an upper limit,
\begin{equation}
  \sum_i|\widetilde{F}_i|^2 \le N\omega_{max}^2\sum_i\left|F_i\right|^2,
  \label{eq:noise}
\end{equation}
where $\omega_{max}$ is the maximum fractional bin width for $\lambda^2$ spectrum, corresponding
with the lowest frequency bin.

The relationship in Equation \ref{eq:noise} allows us to estimate the noise properties of the DRMT
in terms of the original spectrum. If a spectrum has uniform noise with variance
$\sigma^2_{\lambda^2}$, then the DRMT of that wignal will also have uniform noise with variance
$\sigma^2_\Phi \le N\omega_{max}^2\sigma_{\lambda^2}^2$.

Though the noise level of the DRMT is uniform, the noise in each $\Phi$ sample will not necessarily
be independent. Since the kernel of the DRMT, 
\begin{equation}
  K_j = \sum_i e^{-2i\Phi_j\lambda_i^2}\omega_i
  \label{eq:kernel}
\end{equation}
only reduces to a delta function when the $\lambda_i^2$ are uniformly spaced, the noise will have
some covariance given by the matrix $C_{ij} = K_iK_j^*$. As the spacings in $\lambda^2$ become
uniform $(\omega_i\to1)$, the covariance will vanish in all but its off-diagonal terms
$(C_{ij}\to\delta_{ij})$, since the DRMT reduces to a DFT.
