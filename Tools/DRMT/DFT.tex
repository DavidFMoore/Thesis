\section{Comparison with the Discrete Fourier Transform}

Using the tools developed in the previous section, we investigate the validity of setting $w_i=1$ in
Equation \ref{eq:def_dtrmt}. This is equivalent to examining the validity of the approximation that
the spacings in $\lambda^2$ may be approximated as even.

There are two effects that may arise from setting $w_i=1$. First, the samples in $\Phi$ may not be
independent, that is, the uneven sampling of $\lambda^2$ widens the native sinc window of the DFT.
Second, because the data are improperly weighted, the low-frequency samples, where the $\lambda^2$
sampling function is least dense is weighted unnaturally low. This causes the DFT to underestimate
the noise in the RMT.

The effects of these errors may be parameterized by the measure of the DRMT, $\omega_i$, defined in
Equation \ref{eq:def_omega}. This quantity is maximized in the lowest frequency bins (Equation
\ref{eq:def_dlam2}). We can use this value, $\omega_{max}$ as a measure of the uniformity of
spacings in $\lambda^2$. This value can be parameterized by the fractional bandwidth of the 
measurement ($f_\nu$)\nomenclature[Rf]{$f_\nu$}{Fractional bandwidth}, defined for spectra ranging in
frequency from $\nu_0(1-f_\nu/2)$ to $\nu_0(1+f_\nu/2)$. The maximum deviation from even spacings
can be written in terms of $f_\nu$ as
\begin{equation}
  N\omega_{max} = \frac{N\Delta\lambda^2_{max}}{\sum_i\Delta\lambda^2_i}
    = \frac{(1-f_\nu/2)^2}{(1-f_\nu/2)},
\end{equation}
where we employ Equation \ref{eq:def_dlam2} to approximate the spacings $\Delta\lambda^2$ and
noticed that $\sum_i\Delta\lambda^2_i = \lambda_{max}^2 - \lambda_{min}^2$. As $f_\nu\to0$,
$N\omega_{max}\to 1$, and the DRMT will reduce to a DFT, validating the assumption made to set
$w_i$ in Equation \ref{eq:def_dtrmt} to 1. Allowing ten percent tolerance on $N\omega_{max}$ allows
all bands with $f_\nu < 0.07$ to be well approximated by the DFT.

\figuremacroW{kernels.eps}{1.0}{fig:kernels}{
  A comparison of the two methods of the RMT. Each plot shows the squared magnitude of the kernel of
  the DRMT (cyan) and the approximate DFT (black). The three panels show three representative bands
  with central frequency $\nu_0=150 {\rm MHz}$ and different fractional bandwidths $f_\nu$. 
    (Top) $f_\nu=1$
    (Middle) $f_\nu=0.8$
    (Bottom) $f_\nu=0.5$.
    Only 100 bins of the kernel are shown to better show structure near the central peak.
}{Kernels of the DRMT for a few values of $f_\nu$}

Figure \ref{fig:kernels} shows the kernel (Equation \ref{eq:kernel}) for the transform over three
bands with varying fractional bandwidths. Each spectrum is computed with $\nu_0 = 150\ {\rm MHz}$
and $N=1000$ channels, though these numbers do not affect the level of disagreement between the DRMT
and the standard method. As $f_\nu\to0$, the kernels of these two methods converge to a delta
function, the kernel of a true DFT.

We will compare the use of $\omega_i$ set by Equation \ref{eq:def_omega} to setting $w_i=1$ by 
two metrics --- the with of the kernel, and the noise equivalent bandwidth. We will compute these as
functions of the fractional bandwidth $f_\nu$, noting that for the limiting case, when $f_\nu=0$,
the DRMT reduces to the DFT. The first metric, the FWHM, demonstrates the transforms ability to
isolate distinct rotation measure structures. In a true DFT, the FWHM of a transform is always one
bin --- this is the simple statement that the kernel of a DFT in frequency space is a delta
function. Away from that limit, the lower the FWHM, the less covariance between neighboring $\Phi$
modes a transform allows. The second is a measure of the statistical uncertainty in the transformed 
measurement. For a true DFT, this is always one, but in the regime with unevenly spaced $\lambda^2$
samples, its meaning is more complicated. In general, higher values of the noise-equivalent
bandwidth indicate lower levels of uncertainty in the measurement. 

\figuremacroW{FWHM_vs_fnu.eps}{0.6}{fig:fwhm}{
  Full width at half maximum (FWHM) of the kernel of the DRMT (cyan) and the kernel of the approximate DFT
  (black). The FWHM is found by interpolating the value where kernel approaches 0.5 --- this
  interpolation accounts for both the non-integer values, and the deviation from one as $f_\nu\to0$.
}{Full width at half maximum of the DMRT kernel vs. $f_\nu$}
To further examine the kernels of these two transforms, we plot the full width at half maximum
(FWHM) of the kernels as functions of $f_\nu$. We interpolate the kernel to find the precise value
of the FWHM --- this produces non-integer values. As $f_\nu\to0$, we approach one, the value taken
by a DFT. The slight offset from one is due to the interpolation. As $f_\nu\to1$, the FWHM of the DRMT
becomes $2/3$ that of the approximate DFT. 

\figuremacroW{EffectiveBandwidth.eps}{0.6}{fig:neqbw}{
  Noise equivalent bandwidth (Equation \ref{eq:neqbw} of the DRMT is shown in cyan, of the approximate DFT in black. 
  That the bandwidth exceeds one reflects the choice of $\Delta\Phi$ from the largest $\lambda^2$ bin. Dashed, 
  black lines show the allowable range in $\eta$, for any choice of $\Delta\Phi$.
}{Effective bandwidth of the DRMT vs. $f_\nu$}
Another measure of the transform is the noise-equivalent bandwidth of the kernel, defined for a
kernel $K_i$ as
\begin{equation}
  \eta = \frac{\left|\sum_i K_i\right|^2 }{\sum_i \left|K_i\right|^2}
  \label{eq:neqbw}
\end{equation}
This quantity depends on the choice of $\Phi_j$. As we discussed in Section \ref{sec:mechanics}, the
available range in $\Phi$ is well determined in the limit of evenly spaced $\lambda^2$, but outside of 
that limit, the available range is not constrained. We choose to sample $\Phi$ based on the largest
spacings in $\lambda^2$, but this neglects the higher $\Phi$ modes which are measured in the highest
frequency bins. This effect allows the noise-equivalent bandwidth $\eta$ to exceed one, which
typically is forbidden. Figure \ref{fig:neqbw} shows $\eta$ as a function of $f_\nu$, alongside the
allowed ranges of that quantity. The DRMT shows a factor of 1.7 increase in this quantity when compared to the 
approximate DFT.

The DRMT outperforms the approximate DFT in both metrics, indicating that the proper choice of a
metric ($w_i$ in Equation \ref{eq:def_dtrmt}) can lead to more precise measurements of the rotation
measure structures of polarized spectra.
