\section{Polarization Calibration in Redundant Arrays}\label{sec:redcal}

Since the field of view of PAPER cannot be dominated by a single source, we turn to broader, more
statistical measures to calibrate against. The first is that a handful of calibrator sources,
Pictor A, and Fornax A to name two, are only weakly polarized, allowing their constituent
unpolarized signals to dominate the emission in $\mathcal{V}_{xx}$ and $\mathcal{V}_{yy}$. This
allows us to calibrate the $xx$ and $yy$ polarizations with only an unpolarized model. Once the
antenna-dependent gains are solved for, we apply those to $\mathcal{V}_{xy}$ and $\mathcal{V}_{yx}$.

This calibration scheme is incomplete, leaving an uncalibrated phase between the two
cross-polarized visibilities. To solve for this phase difference, we make a final assumption, that
$V=0$, and solve for the phase which minimizes $\mathcal{V}_V \equiv
\mathcal{V}_{xy}-\mathcal{V}_{yx}$. In other words, the assumption that $V=0$ allows us to assume
that $\mathcal{V}_{xy}$ and $\mathcal{V}_{yx}$ are redundant.

To understand the process of calibrating for the cross-polarization phase difference, we review
redundant calibration in general, focusing on PAPER's implementation in particular. We extend this
into polarization, explaining the impetus for solving for a single cross-polarization calibration
term.

If the signal in two visibilities labeled $12$ and $34$ is redundant and calibration terms are 
antenna-based, then we can solve the equation 
\begin{equation}
  g_1g^*_2\mathcal{V}_{12} = g_3g_4^*\mathcal{V}_{34}
\end{equation}
by minimizing the value
\begin{equation}
  \chi^2 = \sum_{(i,j),(k,l)\in\mathcal{R}}
    \left|g_ig_j^*\mathcal{V}_{ij} - g_kg_l^*\mathcal{V}_{kl}\right|^2,
    \label{eq:def_redcal}
\end{equation}
where \nomenclature[Rc]{$\mathcal{R}$}{Set of redundant baselines}$\mathcal{R}$ is the set of redundant baselines 
in an array. A full description of redundant calibration can be found in \citet{Liu2010}, and a
clever implementation of it can be found in \citet{Zheng2014}.

\figuremacroW{AntennaGrid.eps}{0.6}{fig:red_array}{
}{Map of redundant spacings in PSA32}

A full minimization of Equation \ref{eq:def_redcal} requires using all redundant baselines in an
array, a computationally costly task. We restrict ourselves to three redundant types, a subset of
$\mathcal{R}$, labelled by their grid-spacings as $(0,1)$,$(\pm1,1)$ in Figure \ref{fig:red_array},
which shows a cartoon map of the PAPER array.

We are restricted in our choice of a subspace of $\mathcal{R}$. Consider the subspace of all
baselines with grid spacing $(0,1)$. Each row of the array may be calibrated relative to itself, but
there are no terms in \label{eq:def_redcal} which link different rows. If we add all baselines with
$(1,1)$ to this subset, then we can link rows, but the solutions for the top-left and bottom-left
antennae are under-constrained. Adding the $(-1,1)$ baseline type fixes this problem and illuminates
the two related criteria for the subspace of $\mathcal{R}$ necessary to fully solve for redundant
calibration: the subset $\mathcal{R}'$ must extend through all antennae twice, and it also must 
allow for closure quantities to be calculated. As a reminder, the phase of all baselines in a closed
loop of antennae must add to zero ---  this is called a closure quantity. A general rule is to draw
closed loops with elements of $\mathcal{R}'$, shown in Figure \ref{fig:red_array} as a cyan triangle
made of two $(0,1)$ baselines, and one each of $(1,1)$ and $(-1,1)$. 

Having chosen a subset of $\mathcal{R}$, we further simplify the procedure by linearizing Equation 
\ref{eq:def_redcal}, computing the logarithm of the visibilities. Finally, we model the phase as a 
line (an electrical delay $\tau_i$), representing Equation \ref{eq:def_redcal} as
\begin{equation}
  \log\left(\frac{\mathcal{V}_{ij}}{\mathcal{V}_{kl}}\right) 
  = 
  \log g_i + \log g_j + \log g_k + \log g_l 
  +
  2\pi i\nu(\tau_i-\tau_j-\tau_k+\tau_l).
  \label{eq:redcal_vis}
\end{equation}
This equation sacrifices an unbiased, optimal solution for computational ease. The real part of the
log of the ratio $\mathcal{V}_{ij}/\mathcal{V}_{kl}$ is simply a function of the antenna gains, and
the imaginary part is simply a function of the electrical delays.

As a final simplifying step, we choose a fiducial baseline from each type to serve as the denominator 
of the ratio in Equation \ref{eq:redcal_vis}. This eases the computational burden from order $N_{ant}^4$, the
number of baselines squared, to order $N_{ant}^2$. This step essentially reduces the size of the 
matrix representing the addition of gains (or delays) from $N_{ant}\times N_{ant}^4$ to 
$N_{ant}\times N_{ant}^2$ --- the system of equations is still over-constrained, but now we have
less work to do.

Thus we solve for calibration parameters which force the array to be redundant. There are two terms
per polarization which cannot be calibrated in this way, though. The first is obvious: an array-wide
flux scale, setting the calibration of the $xx$-polarization, say, to the sky. We solve for these by
assuming that $I$ dominates the signal in $xx$ and $yy$, and setting the flux to a model of Pictor A
\cite{Jacobs2013b}. The second, less obvious terms remaining are the delays of the fiducial
baselines we chose, which we decompose into a fiducial east-west and a fiducial north-south
baseline. These final terms are also found by fitting a model of Pictor A to all of the now-redundant visibilities.

Now, both the $xx$ and the $yy$ polarizations are calibrated. If we assume that the gains and delays
are antenna-based, we can apply them to the $xy$ and $yx$ polarizations. This leaves one final term:
a cross polarization delay. Up until this point in the redundant calibration process, each polarization 
of the array has been treated independently. To set the $x$ and $y$ delays to the same reference, we
must add information to our calibration schema.

To solve for the cross-polarization delay, we assume that $V=0$ at these frequencies, so we can
treat the $xy$ and $yx$ visibilities as redundant. Then, as before, we solve for the delay which
minimizes $\mathcal{V}_V$, 
\begin{equation}
  \mathfrak{Im}\left\{\log\mathcal{V}_{xy} - \log\mathcal{V}_{yx}\right\} = 2\pi i \tau_{xy}\nu.
  \label{eq:def_crosspol}
\end{equation}
Since all baselines are calibrated to be redundant at this point, we use all available data to solve
for $\tau_{xy}$ and apply it to fully calibrate the redundant array.

It should be noted that this method is similar to that presented in \citet{Cotton2012}, with two
differences. First, \citet{Cotton2012} suggests maximizing the sum on the left-hand side of Equation
\ref{eq:def_crosspol}, which pushes all available signal into $\mathcal{V}_U$, allowing some to
remain in $\mathcal{V}_V$. In general, one cannot make the assumption that $V=0$, but at the low
frequencies measured by PAPER, no circularly polarized emission has been measured to date. Second,
\citet{Cotton2012} uses this method on a per-baseline basis, not assuming a redundant array, which
we clearly do. The use of multiple baselines to solve for a single calibration term increases the
signal-to-noise of the measurements of that calibration term.
