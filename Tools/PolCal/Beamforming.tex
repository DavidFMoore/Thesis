\section{Beamforming}\label{sec:Beamforming}

We now discuss a method for calibrating the off-diagonal gain terms. Typical measurements require a
single calibration source to dominate emission in a field of view. With a wide field imager, this
requirement can never be met. We discuss a method to artificially restrict the field of view, only
imaging a few points on the sky at once. 

We begin by modelling a calibrator as a point source at position $\hat{s}$ with Stokes parameters
$I$, $Q$, $U$, and $V$. We employ Equations \ref{eq:def_Vxx}, \ref{eq:def_Vxy}, \ref{eq:def_Vyx}, 
and \ref{eq:def_Vyy} to represent a visibility containing only this source as a matrix equation,
\begin{align}
  \begin{pmatrix}\mathcal{V}_{xx}\\\mathcal{V}_{xy}\\\mathcal{V}_{yx}\\\mathcal{V}_{yy}\end{pmatrix}
    &= e^{-2\pi i(\vec{b}_{ij}\cdot\hat{s})} \times
  \nonumber \\
  &\begin{pmatrix}A_{xx}&&&\\&A_{xy}&&\\&&A_{yx}&\\&&&A_{yy}\end{pmatrix}
  \begin{pmatrix}
    1 &  1 & 0 &  0 \\
    0 &  0 & 1 & -i \\
    0 &  0 & 1 &  i \\
    1 & -1 & 0 &  0 
  \end{pmatrix}
  \begin{pmatrix}
    1 &         0 &          0 & 0 \\
    0 & \cos2\psi & -\sin2\psi & 0 \\
    0 & \sin2\psi &  \cos2\psi & 0 \\
    0 &         0 &          0 & 1 
  \end{pmatrix}
\begin{pmatrix}I\\Q\\U\\V\end{pmatrix}.
  \label{eq:def_vismat_full}
\end{align}
This equation has neglected gain terms not associated with the primary beam. In contrast with the
discussion in Section \ref{sec:Polarimetry}, we allow the gain matrix $\dvec{G}_{ij}$ to have
sixteen, independent components, rather than forcing it to be diagonal. Furthermore, we write
Equation \ref{eq:def_vismat_full} in terms of a vector the source's stokes parameters $\vec{M}$, a
vector of visibilities $\vec{\mathcal{V}}_{ij,t}$, a diagonal matrix containing the elements of the
beam, $\dvec{A}_t$, and a transfer matrix $\dvec{W}_t$, and the fringe, 
$\exp\{-2\pi i(\vec{b}\cdot\hat{s})\}$:
\begin{equation}
  \vec{\mathcal{V}}_{ij,t} 
  = e^{-2\pi i(\vec{b}_{ij}\cdot\hat{s})}
  \dvec{G}_{ij}\cdot\dvec{A}_t\cdot\dvec{W}_t\cdot\vec{M}. 
  \label{eq:def_beamformmodel}
\end{equation}
We explicitly label the time-dependent quantities with subscript $t$. Modelling multiple point
sources is as simple as summing over different models $M$, with different transfer matrices and
beams,
\begin{equation}
  \vec{\mathcal{V}}_{ij,t} 
  = \dvec{G}_{ij} \cdot
  \sum_s e^{-2\pi i(\vec{b}_{ij}\cdot\hat{s})}
  \dvec{A}_{t,s}\cdot\dvec{W}_{t,s}\cdot\vec{M}_s. 
  \label{eq:def_beamformmodel}
\end{equation}

Our task is to solve for the sixteen components of $\dvec{G}_{ij}$. We do this by assuming that
$\vec{\mathcal{V}}_{ij}$ is comprised of only our model, a sum of point sources, and thermal noise.
This allows us to solve for the components of $\dvec{G}_{ij}$ in the least squares sense, minimizing
\begin{equation}
  \chi^2 = \sum_t \left|
    \vec{\mathcal{V}}_{ij} 
    - \dvec{G}_{ij}\sum_s e^{-2\pi i(\vec{b}_{ij}\cdot\hat{s})}
    \dvec{A}_{t,s}\cdot\dvec{W}_{t,s}\cdot\vec{M}_s
  \right|^2.
  \label{eq:beamform_chi2}
\end{equation}
For simplicity, we will drop the subscripts $i$ and $j$, which until now have denoted the
visibility's baseline label. We do this noting that this calibration solution is
baseline-independent. In an additional measure of notational simplicity, we redefine the model
visibility $\vec{M}'_t \equiv \sum_s \exp\{-2\pi
i(\vec{b}\cdot\hat{s})\}\dvec{A}_{t,s}\cdot\dvec{W}_{t,s}\cdot\vec{M}_s$ as the time-dependent model
visibility. These simplifications reduce the chi-squared expression to
\begin{equation}
  \chi^2 = \sum_t \left|\vec{\mathcal{V}}_t - \dvec{G}\cdot\vec{M}_t'\right|^2.
\end{equation}

The values of $\dvec{G}$ which minimize $\chi^2$, found by setting $d\chi^2/dG_{\alpha\beta}^* = 0$ is 
\begin{equation}
  \dvec{G} = \left(\sum_t \vec{M}_t'\otimes\vec{M}_t'^\dagger\right)^{-1}
             \cdot  
             \left(\sum_t \vec{V}_t \otimes\vec{M}_t'^\dagger\right),
\end{equation}
where $\otimes$ represents the Kronecker outer product, defined in Section \ref{sec:Polarimetry}.

In order for $\sum_t \vec{M}_t'\otimes\vec{M}_t'^\dagger$ to be non-singular, a full Stokes model of
a polarized source must be included. Otherwise, one may approximate the inverse of that matrix by
assuming that model Stokes parameters $Q$, $U$, and $V$ are much smaller than $I$.

Including multiple sources in the model $\vec{M}_t'$ will increase the accuracy of the calibration.
By including many sources, we model both the sources in question and sidelobes from nearby sources
--- this provides accurate source spectra for each pointing. With redundant arrays like PAPER, a
single pointing on the sky may include emission from multiple sources. Modelling this effect will
clearly increase the accuracy of both the calibration and the measurement of source spectra. 

To date, no calibrators are sufficiently accurately measured for use in this method, so this method has yet to be
implemented. Once a more accurate model of the polarized sky in the southern hemisphere is made,
future instruments will be able to use this method to calibrate off-diagonal polarization
calibration terms without the computationally costly and oftentimes uncertain imaging deconvolution.
