\chapter{New Methods for Polarization Calibration}\label{chap:PolCal}

This chapter presents two novel approaches to polarization calibration, designed mainly for
redundant, widefield arrays like PAPER. The two authoritative sources in interferometric
polarization calibration are the series of papers by Hamacker, Bregman, and Sault \cite{HBS1,HBS2,HBS3,HBS4,HBS5} and
Chapter three of \citet{TMS}. They present a calibration scheme in which instrumental
polarization is represented in terms of a rotation matrix $\B{D}_i$, comprised of off-diagonal
terms of the gain equation presented in Section \ref{sec:Polarimetry}, in Equation
\ref{eq:def_gains}. In the Jones formalism, the measured electric field can be described by
the equation 
\begin{equation}
  \begin{pmatrix}E_x\\E_y\end{pmatrix}
= \begin{pmatrix}g_{ix} & 0 \\ 0 &  g_{iy}\end{pmatrix}
  \begin{pmatrix}1&d_{iy}\\-d_{ix}&1\end{pmatrix}
  \begin{pmatrix}\cos\psi & -\sin\psi \\ \sin\psi & \cos\psi\end{pmatrix}
  \begin{pmatrix}E_\alpha\\E_\delta\end{pmatrix}
\equiv
  \B{G}_i\cdot\B{D}_i\cdot\B{P}\cdot\vec{E},
  \label{eq:def_D}
\end{equation}
where all terms agree with Equation \ref{eq:def_gains}. The new, off-diagonal terms $d_{ix}$, e.g.
may be written in term of the three Euler angles of the feed of an antenna \cite{TMS}, but in
general, instrumental polarization arises in the analog electronics, and cannot be represented so
simply.

The cited sources give prescriptions for solving for the $4N_{ant}$ complex calibration terms as
functions of both time and frequency, but these methods require two criteria which PAPER does not meet:
\begin{enumerate}
  \item The primary beam must be dominated by a single or few calibration sources. With PAPER's
    nearly 1 sr beam, there will almost certainly be a plethora of sources with calibration-level
    sensitivities.
  \item The $uv$-coverage of the array must be filled to sufficiently isolate point sources,
    creating a clarity of image which PAPER in its redundant configuration cannot provide.
\end{enumerate}

With these two criteria in mind, we must find non-traditional calibration methods. We present two:
the first of which relies on the redundancy of identical baselines, and the second synthesizes images of
only a few pointings on the sphere.

\section{Polarization Calibration in Redundant Arrays}\label{sec:redcal}

Since the field of view of PAPER cannot be dominated by a single source, we turn to broader, more
statistical measures to calibrate against. The first is that a handful of calibrator sources,
Pictor A, and Fornax A to name two, are only weakly polarized, allowing their constituent
unpolarized signals to dominate the emission in $\mathcal{V}_{xx}$ and $\mathcal{V}_{yy}$. This
allows us to calibrate the $xx$ and $yy$ polarizations with only an unpolarized model. Once the
antenna-dependent gains are solved for, we apply those to $\mathcal{V}_{xy}$ and $\mathcal{V}_{yx}$.

This calibration scheme is incomplete, leaving an uncalibrated phase between the two
cross-polarized visibilities. To solve for this phase difference, we make a final assumption, that
$V=0$, and solve for the phase which minimizes $\mathcal{V}_V \equiv
\mathcal{V}_{xy}-\mathcal{V}_{yx}$. In other words, the assumption that $V=0$ allows us to assume
that $\mathcal{V}_{xy}$ and $\mathcal{V}_{yx}$ are redundant.

To understand the process of calibrating for the cross-polarization phase difference, we review
redundant calibration in general, focusing on PAPER's implementation in particular. We extend this
into polarization, explaining the impetus for solving for a single cross-polarization calibration
term.

If the signal in two visibilities labeled $12$ and $34$ is redundant and calibration terms are 
antenna-based, then we can solve the equation 
\begin{equation}
  g_1g^*_2\mathcal{V}_{12} = g_3g_4^*\mathcal{V}_{34}
\end{equation}
by minimizing the value
\begin{equation}
  \chi^2 = \sum_{(i,j),(k,l)\in\mathcal{R}}
    \left|g_ig_j^*\mathcal{V}_{ij} - g_kg_l^*\mathcal{V}_{kl}\right|^2,
    \label{eq:def_redcal}
\end{equation}
where \nomenclature[Rc]{$\mathcal{R}$}{Set of redundant baselines}$\mathcal{R}$ is the set of redundant baselines 
in an array. A full description of redundant calibration can be found in \citet{Liu2010}, and a
clever implementation of it can be found in \citet{Zheng2014}.

\figuremacroW{AntennaGrid.eps}{0.6}{fig:red_array}{
}{Map of redundant spacings in PSA32}

A full minimization of Equation \ref{eq:def_redcal} requires using all redundant baselines in an
array, a computationally costly task. We restrict ourselves to three redundant types, a subset of
$\mathcal{R}$, labelled by their grid-spacings as $(0,1)$,$(\pm1,1)$ in Figure \ref{fig:red_array},
which shows a cartoon map of the PAPER array.

We are restricted in our choice of a subspace of $\mathcal{R}$. Consider the subspace of all
baselines with grid spacing $(0,1)$. Each row of the array may be calibrated relative to itself, but
there are no terms in \label{eq:def_redcal} which link different rows. If we add all baselines with
$(1,1)$ to this subset, then we can link rows, but the solutions for the top-left and bottom-left
antennae are under-constrained. Adding the $(-1,1)$ baseline type fixes this problem and illuminates
the two related criteria for the subspace of $\mathcal{R}$ necessary to fully solve for redundant
calibration: the subset $\mathcal{R}'$ must extend through all antennae twice, and it also must 
allow for closure quantities to be calculated. As a reminder, the phase of all baselines in a closed
loop of antennae must add to zero ---  this is called a closure quantity. A general rule is to draw
closed loops with elements of $\mathcal{R}'$, shown in Figure \ref{fig:red_array} as a cyan triangle
made of two $(0,1)$ baselines, and one each of $(1,1)$ and $(-1,1)$. 

Having chosen a subset of $\mathcal{R}$, we further simplify the procedure by linearizing Equation 
\ref{eq:def_redcal}, computing the logarithm of the visibilities. Finally, we model the phase as a 
line (an electrical delay $\tau_i$), representing Equation \ref{eq:def_redcal} as
\begin{equation}
  \log\left(\frac{\mathcal{V}_{ij}}{\mathcal{V}_{kl}}\right) 
  = 
  \log g_i + \log g_j + \log g_k + \log g_l 
  +
  2\pi i\nu(\tau_i-\tau_j-\tau_k+\tau_l).
  \label{eq:redcal_vis}
\end{equation}
This equation sacrifices an unbiased, optimal solution for computational ease. The real part of the
log of the ratio $\mathcal{V}_{ij}/\mathcal{V}_{kl}$ is simply a function of the antenna gains, and
the imaginary part is simply a function of the electrical delays.

As a final simplifying step, we choose a fiducial baseline from each type to serve as the denominator 
of the ratio in Equation \ref{eq:redcal_vis}. This eases the computational burden from order $N_{ant}^4$, the
number of baselines squared, to order $N_{ant}^2$. This step essentially reduces the size of the 
matrix representing the addition of gains (or delays) from $N_{ant}\times N_{ant}^4$ to 
$N_{ant}\times N_{ant}^2$ --- the system of equations is still over-constrained, but now we have
less work to do.

Thus we solve for calibration parameters which force the array to be redundant. There are two terms
per polarization which cannot be calibrated in this way, though. The first is obvious: an array-wide
flux scale, setting the calibration of the $xx$-polarization, say, to the sky. We solve for these by
assuming that $I$ dominates the signal in $xx$ and $yy$, and setting the flux to a model of Pictor A
\cite{Jacobs2013b}. The second, less obvious terms remaining are the delays of the fiducial
baselines we chose, which we decompose into a fiducial east-west and a fiducial north-south
baseline. These final terms are also found by fitting a model of Pictor A to all of the now-redundant visibilities.

Now, both the $xx$ and the $yy$ polarizations are calibrated. If we assume that the gains and delays
are antenna-based, we can apply them to the $xy$ and $yx$ polarizations. This leaves one final term:
a cross polarization delay. Up until this point in the redundant calibration process, each polarization 
of the array has been treated independently. To set the $x$ and $y$ delays to the same reference, we
must add information to our calibration schema.

To solve for the cross-polarization delay, we assume that $V=0$ at these frequencies, so we can
treat the $xy$ and $yx$ visibilities as redundant. Then, as before, we solve for the delay which
minimizes $\mathcal{V}_V$, 
\begin{equation}
  \mathfrak{Im}\left\{\log\mathcal{V}_{xy} - \log\mathcal{V}_{yx}\right\} = 2\pi i \tau_{xy}\nu.
  \label{eq:def_crosspol}
\end{equation}
Since all baselines are calibrated to be redundant at this point, we use all available data to solve
for $\tau_{xy}$ and apply it to fully calibrate the redundant array.

It should be noted that this method is similar to that presented in \citet{Cotton2012}, with two
differences. First, \citet{Cotton2012} suggests maximizing the sum on the left-hand side of Equation
\ref{eq:def_crosspol}, which pushes all available signal into $\mathcal{V}_U$, allowing some to
remain in $\mathcal{V}_V$. In general, one cannot make the assumption that $V=0$, but at the low
frequencies measured by PAPER, no circularly polarized emission has been measured to date. Second,
\citet{Cotton2012} uses this method on a per-baseline basis, not assuming a redundant array, which
we clearly do. The use of multiple baselines to solve for a single calibration term increases the
signal-to-noise of the measurements of that calibration term.

\section{Beamforming}\label{sec:Beamforming}

We now discuss a method for calibrating the off-diagonal gain terms. Typical measurements require a
single calibration source to dominate emission in a field of view. With a wide field imager, this
requirement can never be met. We discuss a method to artificially restrict the field of view, only
imaging a few points on the sky at once. 

We begin by modelling a calibrator as a point source at position $\hat{s}$ with Stokes parameters
$I$, $Q$, $U$, and $V$. We employ Equations \ref{eq:def_Vxx}, \ref{eq:def_Vxy}, \ref{eq:def_Vyx}, 
and \ref{eq:def_Vyy} to represent a visibility containing only this source as a matrix equation,
\begin{align}
  \begin{pmatrix}\mathcal{V}_{xx}\\\mathcal{V}_{xy}\\\mathcal{V}_{yx}\\\mathcal{V}_{yy}\end{pmatrix}
    &= e^{-2\pi i(\vec{b}_{ij}\cdot\hat{s})} \times
  \nonumber \\
  &\begin{pmatrix}A_{xx}&&&\\&A_{xy}&&\\&&A_{yx}&\\&&&A_{yy}\end{pmatrix}
  \begin{pmatrix}
    1 &  1 & 0 &  0 \\
    0 &  0 & 1 & -i \\
    0 &  0 & 1 &  i \\
    1 & -1 & 0 &  0 
  \end{pmatrix}
  \begin{pmatrix}
    1 &         0 &          0 & 0 \\
    0 & \cos2\psi & -\sin2\psi & 0 \\
    0 & \sin2\psi &  \cos2\psi & 0 \\
    0 &         0 &          0 & 1 
  \end{pmatrix}
\begin{pmatrix}I\\Q\\U\\V\end{pmatrix}.
  \label{eq:def_vismat_full}
\end{align}
This equation has neglected gain terms not associated with the primary beam. In contrast with the
discussion in Section \ref{sec:Polarimetry}, we allow the gain matrix $\dvec{G}_{ij}$ to have
sixteen, independent components, rather than forcing it to be diagonal. Furthermore, we write
Equation \ref{eq:def_vismat_full} in terms of a vector the source's stokes parameters $\vec{M}$, a
vector of visibilities $\vec{\mathcal{V}}_{ij,t}$, a diagonal matrix containing the elements of the
beam, $\dvec{A}_t$, and a transfer matrix $\dvec{W}_t$, and the fringe, 
$\exp\{-2\pi i(\vec{b}\cdot\hat{s})\}$:
\begin{equation}
  \vec{\mathcal{V}}_{ij,t} 
  = e^{-2\pi i(\vec{b}_{ij}\cdot\hat{s})}
  \dvec{G}_{ij}\cdot\dvec{A}_t\cdot\dvec{W}_t\cdot\vec{M}. 
  \label{eq:def_beamformmodel}
\end{equation}
We explicitly label the time-dependent quantities with subscript $t$. Modelling multiple point
sources is as simple as summing over different models $M$, with different transfer matrices and
beams,
\begin{equation}
  \vec{\mathcal{V}}_{ij,t} 
  = \dvec{G}_{ij} \cdot
  \sum_s e^{-2\pi i(\vec{b}_{ij}\cdot\hat{s})}
  \dvec{A}_{t,s}\cdot\dvec{W}_{t,s}\cdot\vec{M}_s. 
  \label{eq:def_beamformmodel}
\end{equation}

Our task is to solve for the sixteen components of $\dvec{G}_{ij}$. We do this by assuming that
$\vec{\mathcal{V}}_{ij}$ is comprised of only our model, a sum of point sources, and thermal noise.
This allows us to solve for the components of $\dvec{G}_{ij}$ in the least squares sense, minimizing
\begin{equation}
  \chi^2 = \sum_t \left|
    \vec{\mathcal{V}}_{ij} 
    - \dvec{G}_{ij}\sum_s e^{-2\pi i(\vec{b}_{ij}\cdot\hat{s})}
    \dvec{A}_{t,s}\cdot\dvec{W}_{t,s}\cdot\vec{M}_s
  \right|^2.
  \label{eq:beamform_chi2}
\end{equation}
For simplicity, we will drop the subscripts $i$ and $j$, which until now have denoted the
visibility's baseline label. We do this noting that this calibration solution is
baseline-independent. In an additional measure of notational simplicity, we redefine the model
visibility $\vec{M}'_t \equiv \sum_s \exp\{-2\pi
i(\vec{b}\cdot\hat{s})\}\dvec{A}_{t,s}\cdot\dvec{W}_{t,s}\cdot\vec{M}_s$ as the time-dependent model
visibility. These simplifications reduce the chi-squared expression to
\begin{equation}
  \chi^2 = \sum_t \left|\vec{\mathcal{V}}_t - \dvec{G}\cdot\vec{M}_t'\right|^2.
\end{equation}

The values of $\dvec{G}$ which minimize $\chi^2$, found by setting $d\chi^2/dG_{\alpha\beta}^* = 0$ is 
\begin{equation}
  \dvec{G} = \left(\sum_t \vec{M}_t'\otimes\vec{M}_t'^\dagger\right)^{-1}
             \cdot  
             \left(\sum_t \vec{V}_t \otimes\vec{M}_t'^\dagger\right),
\end{equation}
where $\otimes$ represents the Kronecker outer product, defined in Section \ref{sec:Polarimetry}.

In order for $\sum_t \vec{M}_t'\otimes\vec{M}_t'^\dagger$ to be non-singular, a full Stokes model of
a polarized source must be included. Otherwise, one may approximate the inverse of that matrix by
assuming that model Stokes parameters $Q$, $U$, and $V$ are much smaller than $I$.

Including multiple sources in the model $\vec{M}_t'$ will increase the accuracy of the calibration.
By including many sources, we model both the sources in question and sidelobes from nearby sources
--- this provides accurate source spectra for each pointing. With redundant arrays like PAPER, a
single pointing on the sky may include emission from multiple sources. Modelling this effect will
clearly increase the accuracy of both the calibration and the measurement of source spectra. 

To date, no calibrators are sufficiently accurately measured for use in this method, so this method has yet to be
implemented. Once a more accurate model of the polarized sky in the southern hemisphere is made,
future instruments will be able to use this method to calibrate off-diagonal polarization
calibration terms without the computationally costly and oftentimes uncertain imaging deconvolution.


