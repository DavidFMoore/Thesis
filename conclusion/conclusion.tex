\begin{partChapter}
\chapter{Conclusion}

% ----------------------------------------------------------------------
% ----------------------- conclusion content ----------------------- 
% ----------------------------------------------------------------------

We have demonstrated how crucial foreground characterization will be for uncovering signal from the
21cm EoR. For smooth-spectrum sources --- the overwhelming majority of radio sources --- power is
isolated within a ``wedge'' in the $k_\perp$-$k_{||}$ plane, which allows us to undertake foreground
avoidance. Faraday-rotated, polarized point sources disobey this rule, and generally will not be
restricted to low values of $k_{||}$. 

We synthesized state-of-the-art measurements of the polarized sky in radio frequencies to construct
a simulation of polarized point sources. This allows us to simulate the level of contamination of
polarized sources to the 21cm EoR power spectrum. This simulation predicted that the expected level
of polarized foregrounds, leaked into the unpolarized power spectrum, will far exceed the levels of
reasonable models for the 21cm EoR power spectrum. The simulations presented were based mostly on
measurements within the PAPER band, but some extrapolations from higher frequencies were required.
In particular, no comprehensive measurements of the polarized fraction of point sources had been
measured at meter wavelengths, so we drew on measurements at 1.4 GHz. 

To test the predictions made with the simulations, we constructed the most sensitive polarized power
spectra at these frequencies ever made. These power spectra were made from about six months of data
from the PAPER array. Comparing the measured $Q$ power spectrum to the simulated, we find that the
simulations overestimate the levels of power. This yields a revised estimate of polarized leakage
into the unpolarized, 21cm EoR power spectrum --- the revision is lower, but still at the level of
leading models. 

Because of the disagreement between the measured power spectra and the simulated
ones, we update the input parameters to the simulation. Since the distributions of source counts 
and rotation measures are relatively accurately measured, we focus on the distribution of polarized
fractions. By a Bayesian analysis, we find that the data exclude our assumptions to high
significance, and prefer a mean polarized fraction of $2.2\times10^{-3}$, a factor of ten lower than
measured at 1.4 GHz. This new distribution of polarized fractions qualitatively agrees with the few
recent measurements of the polarized sky at meter wavelengths. 

These new measurements show the importance of characterizing polarized foregrounds. Future
observations will require mitigation strategies for instrumental, polarized leakage in order to
detect and characterize the 21cm EoR power spectrum. Mitigation strategies will incorporate both
source detection and the updated design of instruments. Measuring polarized sources at these
frequencies will require new observations from arrays more suited for imaging, as well as updated
techniques for polarimetry. Instruments will also need to limit beam leakage by uniformly
illuminating their dishes, creating a primary beam which is symmetric about 90$^\circ$ rotations.
These tactics will limit polarized leakage and more smoothly pave the way to measuring the
ionization history of the IGM through the 21cm EoR power spectrum. 

\end{partChapter}
